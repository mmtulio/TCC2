% ==============================================================================
% TCC2 - TULIO MULLER
% Capítulo 4 - Resultados Computacionais
% ==============================================================================
\chapter{Resultados Computacionais}
\label{sec-resultados}

Este capítulo apresenta os experimentos computacionais e os resultados alcançados neste trabalho, constando as análises comparativas com outro trabalho, bem como as soluções manuais elaboradas pelos coordenadores de curso do DCOMP-CCENS-UFES. Mostra também as escolhas dos parâmetros do GRASP, os parâmetros específicos da busca local \textit{Simulated Annealing} e os movimentos utilizados para a geração de soluções vizinhas.

\section{Escolha de Parâmetros}
\label{sec-res-param}

O algoritmo GRASP possui dois parâmetros: o número máximo de iterações \(Max_{Iter}\) e o valor \(\alpha\), que define a forma como a construção da solução inicial será conduzida. Se o parâmetro \(\alpha\) tem seu valor muito próximo a 0 (zero), o algoritmo de construção inicial tem comportamento predominantemente guloso, e produz soluções de boa qualidade, porém pouco diversificadas. Caso \(\alpha\) é mais próximo de 1 (um), as soluções possuem características mais aleatórias, porém com a desvantagem de possuir um valor de função objetivo mais alto. Para o fator de aleatoriedade \(\alpha\) foram escolhidos 5 valores distintos: 0, 0.25, 0.5, 0.75 e 1, para serem usados nos experimentos. Além desses valores, em uma das execuções do GRASP o valor usado não foi fixo, ou seja, em cada iteração um valor aleatório de \(\alpha\) no intervalo [0,1] é escolhido e usado. Para efeitos de comparação com o trabalho de \citeonline{vital2015grasp}, o valor de  \(\alpha = 0.15\) também foi testado. E da mesma forma também que \citeonline{vital2015grasp}, o parâmetro \(Max_{Iter}\) foi definido pelo tempo de execução: 500 segundos. 

Para o \textit{Simulated Annealing}, os parâmetros foram calibrados de forma a permitir um certo grau de diversificação no início da busca e maior intensificação no final do processo. Para cada valor de \(\alpha\) e para cada instância (2013/2 e 2016/1) o \textit{SA} foi testado com dois conjuntos de movimentos: \textit{Move} e \textit{Swap}, estes, com 50\% de probabilidade cada e \textit{Lecture Move}.

Na tabela \ref{tbl_param} são apresentados os parâmetros definidos durante a calibração para o algoritmo SA.


\begin{table}[!htbp]
\centering
\begin{tabular}{c|c|c}
\hline \hline
\textbf{Parâmetro} & \textbf{Descrição} & \textbf{Valor} \\ \hline
\(T_{i}\) & Temperatura inicial para o SA & 1000 \\
\(T_{f}\) & Temperatura de congelamento para o SA & 0.01 \\
\(\beta\) & Taxa de resfriamento para o SA & 0.975 \\
\(N_{v}\) & Número máximo de iterações para o SA & 500 \\ \hline \hline
\end{tabular}
\caption{Parâmetros e valores dos algoritmos}
\label{tbl_param}
\end{table}

Para comparações dos resultados, os valores das penalizações (vetor \(\omega\)) das restrições fortes e fracas da função objetivo, usadas neste trabalho, foram as mesmas adotadas por \citeonline{vital2015grasp}.

\begin{table}[!htbp]
\centering
\begin{tabular}{c|c}
\hline \hline
\textbf{Penalização} & \textbf{Valor} \\ \hline
\(\omega_{1}\) & 5000 \\
\(\omega_{2}\) & 5000 \\
\(\omega_{3}\) & 5000 \\
\(\omega_{4}\) & 5000 \\
\(\omega_{5}\) & 5000 \\
\(\omega_{6}\) & 300 \\
\(\omega_{7}\) & 5000 \\
\(\omega_{8}\) & 10 \\
\(\omega_{9}\) & 20 \\
\(\omega_{10}\) & 4 \\
\(\omega_{11}\) & 600 \\
\(\omega_{12}\) & 10 \\
\(\omega_{13}\) & 10 \\
\(\omega_{14}\) & 10 \\
\(\omega_{15}\) & 500 \\
\(\omega_{16}\) & 200 \\
\(\omega_{17}\) & 150 \\ \hline \hline
\end{tabular}
\caption{Penalizações e valores do PTHU}
\label{tbl_penal}
\end{table}


\section{Detalhes de Implementação}
\label{sec-res-impl}

Os algoritmos descritos neste trabalho foram implementados na linguagem C++ e os testes computacionais foram realizados em uma máquina com CPU Intel Core i5-7200U CPU @ 2.50GHz com 12,0 GB de mémória RAM e sistema operacional Microsoft Windows 10.

As ofertas das disciplinas \textit{Trabalho de Conclusão de Curso I}, \textit{Trabalho de Conclusão de Curso II} e \textit{Estágio em Informática} não foram consideradas para o problema, visto que são disciplinas que não necessitam serem consideradas no horário e não requerem espaço físico.

\section{Análise dos Resultados}
\label{sec-res-anal}

O algoritmo foi executado 10 vezes para cada instância (2013/2 e 2016/1), para cada valor de \(\alpha\) (0, 0.15, 0.25. 0.5, 0.75, 1 e Aleatório) e para cada conjunto de movimentos da busca local (\textit{Move}/\textit{Swap} e \textit{Lecture Move}). Ao final das execuções, todas as soluções obtidas apresentaram-se viáveis, ou seja, não violaram nenhuma restrição forte para o problema. É importante notar que, apesar da execução ocorrer até \(Max_{Iter}\), os tempos exibidos são os tempos em que foi alcançada a melhor solução.

Nas Tabelas \ref{tbl_resultados_alfa152013} e \ref{tbl_resultados_alfa152016}, pode-se verificar que, mantendo \(\alpha = 0.15\), mesmo valor usado por \citeonline{vital2015grasp}, para ambos os semestres, a solução com o movimento \textit{Lecture Move} apresentou melhores resultados. Para 2013/2, a média dos resultados foi \(18,37\%\) melhor do que a solução com os movimentos \textit{Move/Swap}, enquanto para o semestre 2016/1, a melhora foi de \(20,08\%\).

A melhoria da solução com o \textit{Lecture Move} em relação ao  \textit{Move/Swap} pode ser explicada pela eliminação da escolha aleatória de qual movimento seria realizado. Com a introdução do \textit{Lecture Move}, primeiro são escolhidos os horários, se os dois horários escolhidos já possuem aulas alocadas, tenta-se realizar o \textit{Swap}, mas se somente um horário já está alocado, tenta-se fazer o \textit{Move}.

\begin{table}[!htbp]
\centering
\begin{tabular}{|c|c|c|}
\hline
 & Move/Swap & Lecture Move \\ \hline
Melhor Solução & 406 & 306 \\ \hline
Média & 463,6 & 331,4 \\ \hline
Tempo & 267,012 & 255,710 \\ \hline
\end{tabular}
\caption{Melhores Soluções, médias das soluções e tempo para 2013/2 com \(\alpha\) = 0.15}
\label{tbl_resultados_alfa152013}
\end{table}

\begin{table}[!htbp]
\centering
\begin{tabular}{|c|c|c|}
\hline
 & Move/Swap & Lecture Move \\ \hline
Melhor Solução & 278 & 208 \\ \hline
Média & 298,8 & 238,0 \\ \hline
Tempo & 231,525 & 301,577 \\ \hline
\end{tabular}
\caption{Melhores Soluções, médias das soluções e tempo para 2016/1 com \(\alpha\) = 0.15}
\label{tbl_resultados_alfa152016}
\end{table}


Na Tabela \ref{tbl_resultados_tulio} são mostrados as melhores resultados, a média dos valores e a média de tempo utilizando o movimento de busca local \textit{Lecture Move} e variando o valor de \(\alpha\). Pode-se observar que a melhor média de soluções para o semestre 2013/2 foi obtida com \(\alpha\) = 0, com a média dos resultados de 320,8, enquanto que para 2016/1 foi com \(\alpha\) = 0.75, com a média dos resultados de 233,4.

Para o semestre 2013/2, a melhor solução encontrada foi com  \(\alpha\) = 0, isso significa que uma estratégia totalmente gulosa foi melhor aproveitada para essa instância.
Já para o semestre 2016/1, com a melhor solução possuindo  \(\alpha\) = 0.75, percebe-se que uma maior aleatoriedade foi mais benéfica para essa instância.

\begin{table}[!htbp]
\begin{adjustbox}{width=1.2\textwidth,center=\textwidth}
\centering
\begin{tabular}{|cc|c|c|c|c|c|c|c|}
\hline
 &  & 0 & 0.15 & 0.25 & 0.5 & 0.75 & 1.0 & Aleatório \\ \hline
\multicolumn{1}{|c|}{} & Melhor Solução & 268 & 306 & 324 & 306 & 296 & 302 & 282 \\ \cline{2-9} 
\multicolumn{1}{|c|}{\begin{tabular}[c]{@{}c@{}}\(2013/\)2\\ \textit{Lecture Move}\end{tabular}} & Média & 320,8 & 331,4 & 351,8 & 347,8 & 347,0 & 350,0 & 337,8 \\ \cline{2-9} 
\multicolumn{1}{|c|}{} & Tempo & 183,860 & 255,710 & 261,245 & 231,892 & 363,247 & 197,625 & 336,832 \\ \hline
\multicolumn{1}{|c|}{} & Melhor Solução & 214 & 208 & 200 & 220 & 198 & 214 & 222 \\ \cline{2-9} 
\multicolumn{1}{|c|}{\begin{tabular}[c]{@{}c@{}}\(2016/1\)\\ \textit{Lecture Move}\end{tabular}} & Média & 247,4 & 238,0 & 234,4 & 242,8 & 233,4 & 243,8 & 234,8 \\ \cline{2-9} 
\multicolumn{1}{|c|}{} & Tempo & 342,737 & 301,577 & 215,537 & 271,353 & 185,639 & 261,776 & 285,848 \\ \hline
\end{tabular}
\end{adjustbox}
\caption{Melhores soluções, média das soluções e tempo para o Movimento Lecture Move com variação de alfa}
\label{tbl_resultados_tulio}
\end{table}

Na Tabela \ref{tbl_resultados_comp} pode-se verificar a melhor solução, os tempos médios de execução e a média da FO obtidas neste trabalho, bem como da solução proposta por \citeonline{vital2015grasp} e o valor da FO da solução construída manualmente pelos coordenadores de curso.

Para o semestre de 2013/2 a solução manual obtida pelo DCOMP resultou em uma FO = 760, já para o trabalho de \citeonline{vital2015grasp} a melhor solução encontrada foi FO = 466. Assim é possível verificar que a solução apresentada por este trabalho alcançou uma melhora percentual de \(57,79\%\) em relação à solução manual 2013/2 e uma melhora de \(20.98\%\) quando comparada à solução de \citeonline{vital2015grasp}.

Já para o semestre de 2016/1, a solução manual obtida pelo DCOMP teve sua FO calculada em 2374. Assim, pode-se verificar que a solução obtida por este trabalho apresenta uma melhora de \(90.17\%\) comparada com a solução manual e de \(21.63\%\) quando comparada à solução de \citeonline{vital2015grasp}.

\begin{table}[!htbp]
% \begin{adjustbox}{width=1.2\textwidth,center=\textwidth}
\centering
\begin{tabular}{|cc|c|c|c|}
\hline
 &  & Manual & \citeonline{vital2015grasp} & GRASP+SA+LM \\ \hline
\multicolumn{1}{|c|}{} & Melhor Solução & 760 & 406 & 268 \\ \cline{2-5} 
\multicolumn{1}{|c|}{2013/2} & Média & - & 463,6 & 320,8 \\ \cline{2-5} 
\multicolumn{1}{|c|}{} & Tempo & - & 267,012 & 206,482 \\ \hline
\multicolumn{1}{|c|}{} & Melhor Solução & 2374 & 278 & 198 \\ \cline{2-5} 
\multicolumn{1}{|c|}{2016/1} & Média & - & 297,8 & 233,4 \\ \cline{2-5} 
\multicolumn{1}{|c|}{} & Tempo & - & 231,525 & 185,639 \\ \hline
\end{tabular}
% \end{adjustbox}
\caption{Melhores soluções, média das soluções e tempo para o Movimento Lecture Move}
\label{tbl_resultados_comp}
\end{table}


% Na Tabela \ref{tbl_medias_2016} pode-se verificar os tempos médios de execução e a média da FO dos outros fatores de aleatoriedade, bem como da solução proposta por \citeonline{vital2015grasp} e valor da FO da solução construída manualmente pelos coordenadores de curso.

% \begin{table}[!htbp]
% \begin{adjustbox}{width=1.2\textwidth,center=\textwidth}
% \centering
% \begin{tabular}{c|c|c|c|c|c|c|c|c|c}
% \hline \hline
%  & Manual & Vital & 0 & 0,15 & 0,25 & 0,50 & 0,75 & 1 & Aleatório \\ \hline
% Função Objetivo & 760 & 406 & 320,8 & 331,4 & 351,8 & 347,8 & 347 & 350 & 337,8 \\
% Tempo (segundos) & - & 116,85 & 206,482 & 255,71 & 261,245 & 231,892 & 363,247 & 197,625 & 336,832\\
% \hline \hline
% \end{tabular}
% \end{adjustbox}
% \caption{Média das Funções objetivo e tempo para 2013/2 com fatores de aleatoriedade diferentes}
% \label{tbl_medias_2013}
% \end{table}

% \begin{table}[!htbp]
% \begin{adjustbox}{width=1.2\textwidth,center=\textwidth}
% \centering
% \begin{tabular}{c|c|c|c|c|c|c|c|c|c}
% \hline \hline
%  & Manual & Vital & 0 & 0,15 & 0,25 & 0,50 & 0,75 & 1 & Aleatório \\ \hline
% Função Objetivo & 2374 & 297,8 & 247,4 & 238 & 234,4 & 242,8 & 233,4 & 243,8 & 234,8 \\
% Tempo (segundos) & - & 231,525 & 342,737 & 301,577 & 215,537 & 271,353 & 185,639 & 261,776 & 285,848\\
% \hline \hline
% \end{tabular}
% \end{adjustbox}
% \caption{Média das Funções objetivo e tempo para 2016/1 com fatores de aleatoriedade diferentes}
% \label{tbl_medias_2016}
% \end{table}



% \begin{table}[!htbp]
% \begin{adjustbox}{width=1.2\textwidth,center=\textwidth}
% \begin{tabular}{c|c|c|c|c|c|c|c|c|c}
% \hline \hline
% Penalização & \begin{tabular}[c]{@{}c@{}}Melhor Solução \\ Manual 2016/1\end{tabular} & \begin{tabular}[c]{@{}c@{}}Melhor Solução \\ Vital\end{tabular} & \begin{tabular}[c]{@{}c@{}}Melhor solução\\ alfa = 0\end{tabular} & \begin{tabular}[c]{@{}c@{}}Melhor solução\\ alfa = 0.15\end{tabular} & \begin{tabular}[c]{@{}c@{}}Melhor solução\\ alfa = 0.25\end{tabular} & \begin{tabular}[c]{@{}c@{}}Melhor solução\\ alfa = 0.50\end{tabular} & \begin{tabular}[c]{@{}c@{}}Melhor solução\\ alfa = 0.75\end{tabular} & \begin{tabular}[c]{@{}c@{}}Melhor solução\\ alfa = 1.00\end{tabular} & \begin{tabular}[c]{@{}c@{}}Melhor solução\\ alfa aleatório\end{tabular} \\
% \hline
% \(CP_p\) &  0 & 0 & 0 & 0 & 0 & 0 & 0 & 0 & 0 \\
% \(CT_t\) &  0 & 0 & 0 & 0 & 0 & 0 & 0 & 0 & 0 \\
% \(CS_s\) &  0 & 0 & 0 & 0 & 0 & 0 & 0 & 0 & 0 \\
% \(OFT\) &  0 & 0 & 0 & 0 & 0 & 0 & 0 & 0 & 0 \\
% \(VS_s\) &  0 & 0 & 0 & 0 & 0 & 0 & 0 & 0 & 0 \\
% \(TSI\) &  0 & 0 & 0 & 0 & 0 & 0 & 0 & 0 & 0 \\
% \(DH3\) &  0 & 0 & 0 & 0 & 0 & 0 & 0 & 0 & 0 \\
% \(IT_p\) &  10 & 10 & 8 & 7 & 7 & 5 & 7 & 7 & 7 \\
% \(JH_t\) &  6 & 0 & 0 & 0 & 0 & 0 & 0 & 0 & 0 \\
% \(PP_t\) &  0 & 24 & 17 & 14 & 16 & 14 & 19 & 13 & 18 \\
% \(AS_d\) &  0 & 0 & 0 & 0 & 0 & 0 & 0 & 0 & 0 \\
% \(ND_p\) &  1 & 0 & 0 & 0 & 0 & 0 & 0 & 0 & 0 \\
% \(ASD\) &  34 & 10 & 5 & 8 & 9 & 8 & 7 & 9 & 6 \\
% \(ADU\) &  19 & 11 & 7 & 10 & 10 & 12 & 8 & 9 & 8 \\
% \(DHP\) &  0 & 0 & 0 & 0 & 0 & 0 & 0 & 0 & 0 \\
% \(AHA\) &  0 & 0 & 0 & 0 & 0 & 0 & 0 & 0 & 0 \\
% \(AHFP\) &  0 & 0 & 0 & 0 & 0 & 0 & 0 & 0 & 0 \\\hline
% Função Objetivo &  760 & 406 & 268 & 306 & 324 & 306 & 296 & 302 & 282 \\
% \(Tempo\ (segundos)\) &  - & 116,85 & 183,86 & 400,17 & 304,2 & 24,94 & 78,82 & 81,53 & 397,98 \\
% \(Alfa\) & - & 0,15 & 0,00 & 0,15 & 0,25 & 0,50 & 0,75 & 1,00 & 0,972076 \\
% \hline \hline
% \end{tabular}
% \end{adjustbox}
% \caption{Comparação das melhores soluções para 2013/2 com fatores de aleatoriedade diferentes}
% \label{tbl_melhores_2013}
% \end{table}

% \begin{table}[!htbp]
% \begin{adjustbox}{width=1.2\textwidth,center=\textwidth}
% \begin{tabular}{c|c|c|c|c|c|c|c|c|c}
% \hline \hline
% Penalização & \begin{tabular}[c]{@{}c@{}}Melhor Solução \\ Manual 2013/2\end{tabular} & \begin{tabular}[c]{@{}c@{}}Melhor Solução \\ Vital\end{tabular} & \begin{tabular}[c]{@{}c@{}}Melhor solução\\ alfa = 0\end{tabular} & \begin{tabular}[c]{@{}c@{}}Melhor solução\\ alfa = 0.15\end{tabular} & \begin{tabular}[c]{@{}c@{}}Melhor solução\\ alfa = 0.25\end{tabular} & \begin{tabular}[c]{@{}c@{}}Melhor solução\\ alfa = 0.50\end{tabular} & \begin{tabular}[c]{@{}c@{}}Melhor solução\\ alfa = 0.75\end{tabular} & \begin{tabular}[c]{@{}c@{}}Melhor solução\\ alfa = 1.00\end{tabular} & \begin{tabular}[c]{@{}c@{}}Melhor solução\\ alfa aleatório\end{tabular} \\ \hline
% \(CP_p\) & 0 & 0 & 0 & 0 & 0 & 0 & 0 & 0 & 0 \\
% \(CT_t\) & 0 & 0 & 0 & 0 & 0 & 0 & 0 & 0 & 0 \\
% \(CS_s\) & 0 & 0 & 0 & 0 & 0 & 0 & 0 & 0 & 0 \\
% \(OFT\) & 0 & 0 & 0 & 0 & 0 & 0 & 0 & 0 & 0 \\
% \(VS_s\) & 0 & 0 & 0 & 0 & 0 & 0 & 0 & 0 & 0 \\
% \(TSI\) & 0 & 0 & 0 & 0 & 0 & 0 & 0 & 0 & 0 \\
% \(DH3\) & 0 & 0 & 0 & 0 & 0 & 0 & 0 & 0 & 0 \\
% \(IT_p\) & 3 & 5 & 4 & 3 & 3 & 4 & 5 & 3 & 4 \\
% \(JH_t\) & 8 & 0 & 0 & 0 & 0 & 0 & 0 & 0 & 0 \\
% \(PP_t\) & 16 & 17 & 11 & 12 & 10 & 15 & 12 & 11 & 13 \\
% \(AS_d\) & 3 & 0 & 0 & 0 & 0 & 0 & 0 & 0 & 0 \\
% \(ND_p\) & 1 & 0 & 0 & 0 & 0 & 0 & 0 & 0 & 0 \\
% \(ASD\) & 14 & 10 & 8 & 7 & 8 & 9 & 6 & 9 & 9 \\
% \(ADU\) & 18 & 6 & 5 & 6 & 5 & 3 & 4 & 5 & 4 \\
% \(DHP\) & 0 & 0 & 0 & 0 & 0 & 0 & 0 & 0 & 0 \\
% \(AHA\) & 0 & 0 & 0 & 0 & 0 & 0 & 0 & 0 & 0 \\
% \(AHFP\) & 0 & 0 & 0 & 0 & 0 & 0 & 0 & 0 & 0 \\ \hline
% Função Objetivo & 2374 & 278 & 214 & 208 & 200 & 220 & 198 & 214 & 222 \\
% \(Tempo\ (segundos)\) & - & 82,28 & 15,12 & 88,98 & 182,18 & 71,92 & 341,34 & 57,12 & 454,48 \\
% \(Alfa\) & - & 0,15 & 0,00 & 0,15 & 0,25 & 0,50 & 0,75 & 1,00 & 0.440718 \\
% \hline \hline
% \end{tabular}
% \end{adjustbox}
% \caption{Comparação das melhores soluções para 2016/1 com fatores de aleatoriedade diferentes}
% \label{tbl_melhores_2016}
% \end{table}