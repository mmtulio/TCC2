% ==============================================================================
% Modelo para Monografia de Projeto de Graduação (PG)
% Agradecimentos: Prof. Vítor E. Silva Souza - Nemo / DI / UFES
%
% Baseado em abtex2-modelo-trabalho-academico.tex, v-1.9.2 laurocesar
% Copyright 2012-2014 by abnTeX2 group at http://abntex2.googlecode.com/ 
%
% This work may be distributed and/or modified under the conditions of the LaTeX 
% Project Public License, either version 1.3 of this license or (at your option) 
% any later version. The latest version of this license is in
% http://www.latex-project.org/lppl.txt.
%
% IMPORTANTE:
% Instruções encontram-se espalhadas pelo documento. Para facilitar sua leitura,
% tais instruções são precedidas por (*) -- utilize a função localizar do seu
% editor para passar por todas elas.
% ==============================================================================

% Usa o estilo abntex2, configurando detalhes de formatação e hifenização.
\documentclass[
	12pt,				% Tamanho da fonte.
	openright,			% Capítulos começam em página ímpar (insere página vazia caso preciso).
	oneside,			% Para impressão em verso e anverso. Oposto a oneside.
	a4paper,			% Tamanho do papel.
	english,			% Idioma adicional para hifenização.
	french,				% Idioma adicional para hifenização.
	spanish,			% Idioma adicional para hifenização.
	brazil				% O último idioma é o principal do documento.
	]{abntex2}



%%% Importação de pacotes. %%%

% Conserta o erro "No room for a new \count"
\usepackage{etex}
%\reserveinserts{28}

% Usa a fonte Latin Modern.
\usepackage{lmodern}

% Seleção de códigos de fonte.
\usepackage[T1]{fontenc}

% Codificação do documento em Unicode.
\usepackage[utf8]{inputenc}

% Usado pela ficha catalográfica.
\usepackage{lastpage}

% Indenta o primeiro parágrafo de cada seção.
\usepackage{indentfirst}

% Controle das cores.
\usepackage[usenames,dvipsnames, xcdraw, table]{xcolor}

% Inclusão de gráficos.
\usepackage{graphicx}

% Inclusão de páginas em PDF diretamente no documento (para uso nos apêndices).
\usepackage{pdfpages}

% Para melhorias de justificação.
\usepackage{microtype}

\usepackage[portuguesekw,ruled,lined,linesnumbered]{algorithm2e}
\usepackage{algorithmic}

\usepackage[dvipsnames]{xcolor}

\usepackage{adjustbox}

% Citações padrão ABNT.
\usepackage[brazilian,hyperpageref]{backref}
\usepackage[alf]{abntex2cite}	
\renewcommand{\backrefpagesname}{Citado na(s) página(s):~}		% Usado sem a opção hyperpageref de backref.
\renewcommand{\backref}{}										% Texto padrão antes do número das páginas.
\renewcommand*{\backrefalt}[4]{									% Define os textos da citação.
	\ifcase #1
		Nenhuma citação no texto.
	\or
		Citado na página #2.
	\else
		Citado #1 vezes nas páginas #2.
	\fi}

% \rm is deprecated and should not be used in a LaTeX2e document
% http://tex.stackexchange.com/questions/151897/always-textrm-never-rm-a-counterexample
\renewcommand{\rm}{\textrm}

% Pacotes não incluídos no template abntex2. 
% Podem ser comentados caso não queira utilizá-los.

% Inclusão de símbolos não padrão.
\usepackage{amssymb}
\usepackage{eurosym}

% Para utilizar \eqref para referenciar equações.
\usepackage{amsmath}

% Permite mostrar figuras muito largas em modo paisagem com \begin{sidewaysfigure} ao invés de \begin{figure}.
\usepackage{rotating}

% Permite customizar listas enumeradas/com marcadores.
\usepackage{enumitem}

% Permite inserir hiperlinks com \url{}.
\usepackage{bigfoot}
\usepackage{hyperref}

% Permite usar o comando \hl{} para evidenciar texto com fundo amarelo. Útil para chamar atenção a itens a fazer.
\usepackage{soulutf8}

% Colorinlistoftodos package: to insert colored comments so authors can collaborate on the content.
\usepackage[colorinlistoftodos, textwidth=20mm, textsize=footnotesize]{todonotes}
\newcommand{\aluno}[1]{\todo[author=\textbf{Aluno},color=green!30,caption={},inline]{#1}}
\newcommand{\professor}[1]{\todo[author=\textbf{Professor},color=red!30,caption={},inline]{#1}}

\newcommand{\listalgorithmname}{Lista de Algoritimos}
\renewcommand{\listalgorithmname}{Lista de Algoritimos}


% Permite inserir espaço em branco condicional (incluído no texto final só se necessário) em macros.
\usepackage{xspace}

% Permite incluir listagens de código com o comando \lstinputlisting{}.
\usepackage{listings}
\usepackage{caption}
\usepackage{subcaption}
\usepackage{array}
\newcolumntype{M}[1]{>{\centering\arraybackslash}m{#1}}

\usepackage{longtable}
\usepackage{float}
\DeclareCaptionFont{white}{\color{white}}
\DeclareCaptionFormat{listing}{\colorbox{gray}{\parbox{\textwidth}{#1#2#3}}}
% \captionsetup[lstlisting]{format=listing,labelfont=white,textfont=white}
\renewcommand{\lstlistingname}{Listagem}
\definecolor{mygray}{rgb}{0.5,0.5,0.5}
\lstset{
	basicstyle=\scriptsize,
	breaklines=true,
	numbers=left,
	numbersep=5pt,
	numberstyle=\tiny\color{mygray}, 
	rulecolor=\color{black},
	showstringspaces=false,
	tabsize=2,
    inputencoding=utf8,
    extendedchars=true,
    literate=%
    {é}{{\'{e}}}1
    {è}{{\`{e}}}1
    {ê}{{\^{e}}}1
    {ë}{{\¨{e}}}1
    {É}{{\'{E}}}1
    {Ê}{{\^{E}}}1
    {û}{{\^{u}}}1
    {ù}{{\`{u}}}1
    {â}{{\^{a}}}1
    {à}{{\`{a}}}1
    {á}{{\'{a}}}1
    {ã}{{\~{a}}}1
    {Á}{{\'{A}}}1
    {Â}{{\^{A}}}1
    {Ã}{{\~{A}}}1
    {ç}{{\c{c}}}1
    {Ç}{{\c{C}}}1
    {õ}{{\~{o}}}1
    {ó}{{\'{o}}}1
    {ô}{{\^{o}}}1
    {Õ}{{\~{O}}}1
    {Ó}{{\'{O}}}1
    {Ô}{{\^{O}}}1
    {î}{{\^{i}}}1
    {Î}{{\^{I}}}1
    {í}{{\'{i}}}1
    {Í}{{\~{Í}}}1
}

%%% Definição de variáveis. %%%

% (*) Substituir os textos abaixo com as informações apropriadas.
\titulo{ESTUDO DO IMPACTO DA ALEATORIEDADE E MOVIMENTOS NA META-HEURÍSTICA GRASP COM SIMULATED ANNEALING PARA O PROBLEMA DE TABELA-HORÁRIO DO DCOMP-CCENS-UFES}
\autor{Tulio Machado Müller}
\local{Alegre - ES}
\data{2020}
\orientador{Prof. Dr. Edmar Hell Kampke}
\instituicao{
  Universidade Federal do Espírito Santo
  \par
  Centro de Ciências Exatas, Naturais e da Saúde
  \par
  Departamento de Computação}
\tipotrabalho{Trabalho de Conclusão de Curso 2 (TCC2)}

% Preâmbulo (tipo do trabalho, objetivo, nome da instituição, área de concentração, etc.).
% (*) Verificar se está correto (ex.: substituir por Engenharia de Computação se for o caso).
\preambulo{Trabalho de conclusão de curso apresentado ao Departamento de Computação do Centro de Ciências Exatas, Naturais e da Saúde da Universidade Federal do Espírito Santo, como requisito parcial para obtenção do Grau de Bacharel em Ciência da Computação.}

% Macros específicas do trabalho.
% (*) Inclua aqui termos que são utilizados muitas vezes e que demandam formatação especial.
% Os exemplos abaixo incluem i* (substituindo o asterisco por uma estrela) e Java com TM em superscript.
% Use sempre \xspace para que o LaTeX inclua espaço em branco após a macro somente quando necessário.
\newcommand{\istar}{\textit{i}$^\star$\xspace}
\newcommand{\java}{Java\texttrademark\xspace}
\newcommand{\latex}{\LaTeX\xspace}




%%% Configurações finais de aparência. %%%

% Altera o aspecto da cor azul.
\definecolor{blue}{RGB}{41,5,195}

% Informações do PDF.
\makeatletter
\hypersetup{
	pdftitle={\@title}, 
	pdfauthor={\@author},
	pdfsubject={\imprimirpreambulo},
	pdfcreator={LaTeX with abnTeX2},
	pdfkeywords={abnt}{latex}{abntex}{abntex2}{trabalho acadêmico}, 
	colorlinks=true,				% Colore os links (ao invés de usar caixas).
	linkcolor=blue,					% Cor dos links.
	citecolor=blue,					% Cor dos links na bibliografia.
	filecolor=magenta,				% Cor dos links de arquivo.
	urlcolor=blue,					% Cor das URLs.
	bookmarksdepth=4
}
\makeatother

% Espaçamentos entre linhas e parágrafos.
\setlength{\parindent}{1.3cm}
\setlength{\parskip}{0.2cm}



%%% Páginas iniciais do documento: capa, folha de rosto, ficha, resumo, tabelas, etc. %%%

% Compila o índice.
\makeindex

% Inicia o documento.
\begin{document}


\begin{figure}
\centering
\includegraphics[width=.20\textwidth]{figuras/brasao.jpg} 
\label{fig-brasao}
\end{figure}

\begin{center}
\textbf{\textsf{UNIVERSIDADE FEDERAL DO ESPÍRITO SANTO}}

\textbf{\textsf{CENTRO DE CIÊNCIAS EXATAS, NATURAIS E DA SAÚDE}}

\textbf{\textsf{DEPARTAMENTO DE COMPUTAÇÃO}}

\large{\textbf{\textsf{  }}}

\large{\textbf{\textsf{  }}}
\end{center}

% Retira espaço extra obsoleto entre as frases.
\frenchspacing

% Capa do trabalho.
\imprimircapa

% Folha de rosto (o * indica que haverá a ficha bibliográfica).
\imprimirfolhaderosto*


% Ficha catalográfica.
% (*) Escolher entre as versões de ficha catalográfica abaixo (comente aquela que não quiser usar).

% Versão 1: caso a biblioteca da sua universidade lhe forneça um PDF (adequar o nome do arquivo).
% \begin{fichacatalografica}
%     \includepdf{include-fichacatalografica.pdf}
% \end{fichacatalografica}

% Versão 2: caso você tenha que inserir sua própria ficha catalográfica.
% (*) Neste caso, preencher palavras-chave e adicione co-orientador (se houver).
% \begin{fichacatalografica}
% 	\vspace*{\fill}
% 	\hrule
% 	\begin{center}
% 	\begin{minipage}[c]{12.5cm}
	
% 	\imprimirautor
	
% 	\hspace{0.5cm} \imprimirtitulo  / \imprimirautor. --
% 	\imprimirlocal, \imprimirdata-
	
% 	\hspace{0.5cm} \pageref{LastPage} p. : il. (algumas color.) ; 30 cm.\\
	
% 	\hspace{0.5cm} \imprimirorientadorRotulo~\imprimirorientador\\
	
% 	\hspace{0.5cm}
% 	\parbox[t]{\textwidth}{\imprimirtipotrabalho~--~\imprimirinstituicao,
% 	\imprimirdata.}\\
	
% 	\hspace{0.5cm}
% 		1. Problema de Tabela-Horário em Universidades.
% 		2. Metaheurística.
% 		I. Tulio Machado Müller.
% 		II. Universidade Federal do Espírito Santo.
%         III. Departamento de Computação.
% 		IV. \imprimirtitulo \\ 			
	
% 	\hspace{8.75cm} CDU 02:141:005.7\\
	
% 	\end{minipage}
% 	\end{center}
% 	\hrule
% \end{fichacatalografica}


% Folha de aprovação.
% (*) Escolher entre as versões de ficha catalográfica abaixo (comente aquela que não quiser usar).

% Versão 1: cópia digitalizada da folha de aprovação assinada pela banca.
% \includepdf{include-folhadeaprovacao.pdf}

% Versão 2: folha de aprovação em branco.
% (*) Ajustar a data e os nomes dos participantes da banca.
\begin{folhadeaprovacao}
  \begin{center}
    {\ABNTEXchapterfont\large\imprimirautor}
    \vspace*{\fill}\vspace*{\fill}
    \begin{center}
      \ABNTEXchapterfont\bfseries\Large\imprimirtitulo
    \end{center}
    \vspace*{\fill}
    \hspace{.45\textwidth}
    \begin{minipage}{.5\textwidth}
        \imprimirpreambulo
    \end{minipage}%
    \vspace*{\fill}
   \end{center}
   Trabalho Aprovado. \imprimirlocal, 06 de Novembro de 2020:
   \assinatura{\textbf{\imprimirorientador} \\ Orientador} 
   \assinatura{\textbf{Prof. M.Sc. Valéria Alves da Silva} \\ Universidade Federal do Espírito Santo}
   \assinatura{\textbf{Prof. Dr. Rodrigo Freitas Silva} \\ Universidade Federal do Espírito Santo}
   %\assinatura{\textbf{Professor} \\ Convidado 3}
   %\assinatura{\textbf{Professor} \\ Convidado 4}
   \begin{center}
    \vspace*{0.5cm}
    {\large\imprimirlocal}
    \par
    {\large\imprimirdata}
    \vspace*{1cm}
  \end{center}  
\end{folhadeaprovacao}


% Dedicatória.
% (*) Escrever dedicatória ou remover/comentar seção.
%\begin{dedicatoria}
%   \vspace*{\fill}
%   \centering
%   \noindent
%   \textit{Lorem ipsum dolor sit amet, consectetur adipiscing elit. Duis malesuada laoreet leo at interdum. Nullam neque eros, dignissim sed ipsum sed, sagittis laoreet nisi.} \vspace*{\fill}
%\end{dedicatoria}


% Agradecimentos.
% (*) Escrever agradecimentos ou remover/comentar seção.
\begin{agradecimentos}
Agradeço a todos que me apoiaram até aqui e não me deixaram desistir.
\end{agradecimentos}


% Epígrafe.
% (*) Escrever epígrafe ou remover/comentar seção.
%\begin{epigrafe}
%    \vspace*{\fill}
%	\begin{flushright}
%		\textit{``Lorem ipsum dolor sit amet, consectetur adipiscing elit. \\
%		Duis malesuada laoreet leo at interdum. Nullam neque eros, dignissim \\
%		sed ipsum sed, sagittis laoreet nisi.\\
%		(Lipsum generator)}
%	\end{flushright}
%\end{epigrafe}


% Resumo em português.
% (*) Escrever resumo e palavras-chave.
\setlength{\absparsep}{18pt}
\begin{resumo}
O Problema de Tabela-Horário em Universidades (PTHU) possui grande relevância no âmbito acadêmico, por ser um processo complexo e desgastante que deve ser realizado todo semestre. O problema consiste em alocar um conjunto de aulas em salas disponíveis e em períodos pré-determinados, de tal forma que a solução encontrada atenda os anseios dos envolvidos da melhor forma possível. Sendo assim, este trabalho propõe o estudo do impacto da escolha do fator de aleatoriedade e de movimentos na meta-heurística híbrida \textit{Greedy Randomized Adaptive Search Procedure} (GRASP) com \textit{Simulated Annealing} (SA), que em outro trabalho já mostrou ser eficiente na resolução do PTHU do Departamento de Computação (DCOMP) do Centro de Ciências Exatas, Naturais e da Saúde (CCENS) da Universidade Federal do Espírito Santo (UFES). Dessa maneira, um novo movimento, denominado \textit{Lecture Move}, foi implementado e o PTHU do DCOMP-CCENS-UFES, que atualmente é elaborado manualmente, foi resolvido combinando diferentes fatores de aleatoriedade e movimentos. Os resultados obtidos são comparados com as melhores soluções disponíveis na literatura e também com as soluções manuais elaboradas pelos coordenadores de curso. Esses resultados indicam que a escolha do fator de aleatoriedade e dos movimentos usados na busca local impactam diretamente na qualidade das soluções obtidas.

\textbf{Palavras-chaves:} Problema de Tabela-Horário em Universidades; Meta-heurística; GRASP; Simulated Annealing; Lecture Move.

\end{resumo}

% Resumo em inglês.
% (*) Escrever resumo e palavras-chave.
\setlength{\absparsep}{18pt}
\begin{resumo}[Abstract]

The University Timetabling Problem (PTHU) has great relevance in the academic field, being a complex and exhausting process that must be done every semester. The problem consists in allocating a set of classes in rooms available at predetermined intervals, in such a way that a found solution meets the wishes of those involved in the best possible way. Therefore, this work proposes the study of the impact of the choice of the factor of randomness and movements in the hybrid metaheuristic Greedy Randomized Adaptive Search Procedure (GRASP) with Simulated Annealing (SA), which in another work has already proved to be efficient in solving the PTHU of the Computing Department (DCOMP) of the Center for Exact, Natural and Health Sciences (CCENS) of the Federal University of Espirito Santo (UFES). In this way, a new movement, called Lecture Move, was implemented and the PTHU of the DCOMP-CCENS-UFES, which is currently elaborated manually, was solved by combining different factors of randomness and movements. The results obtained are compared with the best solutions available in the literature and also with the manual solutions developed by the course coordinators. These results indicate that the choice of the randomness factor and the movements used in the local search directly impact the quality of the solutions obtained.

\textbf{Keywords:} University timetabling problem; Metaheuristic; GRASP; Simulated Annealing; Lecture Move.
\end{resumo}


% Insere lista de ilustrações.
\pdfbookmark[0]{\listfigurename}{lof}
\listoffigures*
\cleardoublepage

% Insere lista de tabelas.
\pdfbookmark[0]{\listtablename}{lot}
\listoftables*
\cleardoublepage

\pdfbookmark[0]{\listalgorithmname}{loa}
\listofalgorithms
\cleardoublepage

% Lista de abreviaturas e siglas.
% (*) Preencher com as siglas usadas ao longo do texto e seus significados.
\begin{siglas}
  \item[ALNS] Adaptive Large Neighborhood Search
  \item[CCENS] Centro de Ciências Exatas, Naturais e da Saúde
  \item[DCOMP] Departamento de Computação
  \item[FO] Função Objetivo
  \item[GRASP] Greedy Randomized Adaptive Search Procedure
  \item[ITC] International Timetabling Competition
  \item[LC] Lista de Candidatos
  \item[LM] Lecture Move
  \item[LRC] Lista Restrita de Candidatos
  \item[PTHU] Problema de Tabela-Horário em Universidades
  \item[SA] Simulated Annealing
  \item[UFES] Universidade Federal do Espírito Santo


\end{siglas}

% Insere o sumário.
\pdfbookmark[0]{\contentsname}{toc}
\tableofcontents*
\cleardoublepage



%%% Início da parte de conteúdo do documento. %%%

% Marca o início dos elementos textuais.
\textual

% Inclusão dos capítulos.
% (*) Para facilitar a organização, os capítulos foram divididos em arquivo separados e colocados dentro da.
% pasta capitulos/. Caso o aluno prefira trabalhar com um só arquivo, basta substituir os comandos \include 
% pelos conteúdos dos arquivos que estão sendo incluídos, excluindo a pasta capitulos/ em seguida.
% ==============================================================================
% TCC2 - TULIO MULLER
% Capítulo 1 - Introdução
% ==============================================================================
\chapter{Introdução}
\label{sec-intro}

%\hl{Texto.}

%\hrulefill

O problema de tabela-horário é um problema de Otimização Combinatória (OC), que deriva do problema de escalonamento e foi definido por \citeonline{wren-1995} como a alocação, submetida a restrições, de eventos em um número limitado de períodos de tempo e locais, de forma a satisfazer, tanto quanto possível, um conjunto de objetivos estabelecidos.

\section{O problema e sua importância}
\label{sec-intro-problema}

O problema de tabela-horário pode ser aplicado a diversos tipos de situações, entre os quais pode-se citar: escalas de trabalhadores, escalas de condutores de veículos de transporte, escalas de competições esportivas, e tabelas de horário educacionais, sendo este último o foco deste trabalho, por ser um dos mais estudados na área, conforme pode ser observado em \citeonline{schaerf1999survey}.

Os problemas de tabela-horário educacionais abordam, por exemplo, a geração de tabela-horário para escolas de ensino médio e universidades. O problema de tabela-horário de universidades possui diversas formulações. Isso ocorre pois cada instituição de ensino possui diferentes restrições do problema.

A complexidade do Problema de Tabela-Horário de Universidades (PTHU) é uma das maiores da área de otimização combinatória, e aumenta à medida que são adicionadas restrições, ou seja, quanto mais restrições o problema tiver, mais difícil será encontrar uma solução que atenda todas elas. Segundo \citeonline{schaerf1999survey}, esse problema é classificado como NP-completo para grande parte das formulações, isso significa que a solução ótima só pode ser encontrada rapidamente para instâncias muito pequenas, o que não é a realidade da maioria das universidades brasileiras. Desta forma, busca-se minimizar, através de uma solução automática, o esforço manual na geração de tabela-horário, bem como isentar o processo de um possível viés na alocação dos horários, por parte dos docentes envolvidos.

Apesar das diferentes formulações, segundo \citeonline{santos2007programaccao}, os problemas de tabela-horário educacionais possuem uma característica em comum: a separação das restrições em dois grupos, denominados de restrições fortes e restrições fracas. Isso é feito dessa maneira, pois geralmente não é possível encontrar uma solução que atenda todas as restrições impostas.

As restrições fortes são aquelas que não podem ser violadas. Este grupo restringe o conjunto de soluções para impedir situações irreais. Se uma tabela-horário não viola nenhuma restrição forte, ela é considerada uma solução viável.

As restrições fracas são aquelas cuja satisfação é desejável, mas caso não seja possível atendê-las, a solução não é inviabilizada. Essas restrições possuem pesos para refletir sua importância na qualidade da solução.

Este trabalho leva em consideração o PTHU do caso real do Departamento de Computação (DCOMP) do Centro de Ciências Exatas, Naturais e da Saúde (CCENS) da Universidade Federal do Espírito Santo (UFES), no qual foram inicialmente identificadas 15 restrições por \citeonline{mariano2014alns}, e posteriormente adicionadas mais 2 restrições por \citeonline{vital2015grasp}. A seguir, as 17 restrições consideradas neste trabalho são apresentadas.

\begin{itemize}
    \item Restrições Fortes:
    \begin{enumerate}
        \item Conflitos de professor: um professor não poderá ministrar mais de uma disciplina no mesmo dia e horário;
        \item Conflitos de turmas: uma turma não poderá assistir a mais de uma aula no mesmo dia e horário;
        \item Conflitos de salas: uma sala de aula não poderá estar reservada para mais de uma disciplina no mesmo dia e horário;
        \item Aulas fora do turno: uma aula não poderá ser alocada fora do turno da oferta (diurno ou noturno).
        \item Capacidade da sala: uma turma não poderá ser alocada em uma sala cuja capacidade seja inferior ao número de alunos da turma;
        \item Tipo incompatível de sala: as aulas não poderão ser alocadas em uma determinada sala que não é compatível ao tipo solicitado, por exemplo, aulas que deveriam ser realizadas em laboratórios e foram alocadas em salas de aula;
        \item “Disciplinas especiais”: disciplinas com 3 horas aulas semanais deverão ser alocadas nos três primeiros horários do turno diurno, e nos três primeiros ou três últimos horários do turno noturno, permitindo assim que outras disciplinas possam ser alocadas;
    \end{enumerate}
    \item Restrições Fracas:
    \begin{enumerate}[resume]
		\item Intervalo de trabalho do professor: o intervalo entre o primeiro e o último dia da semana em que um professor ministrará as aulas deverá ser minimizado;
		\item Janelas de horário: intervalos na grade de horários de cada turma, entre duas aulas, deverão ser reduzidos;
		\item Período preferencial: as turmas diurnas deverão ter suas disciplinas concentradas no período da manhã ou da tarde. Assim, a quantidade de disciplinas ofertadas fora do turno “preferencial” de cada turma deverá ser minimizada;
		\item Aulas seguidas: aulas repetidas de uma disciplina ministradas para uma turma no mesmo dia devem ser evitadas;
		\item Intervalo entre períodos: a ocorrência de professores que ministram aula em um dia à noite e no dia seguinte pela manhã deverá ser minimizada;
		\item Aulas seguidas de nível “difícil”: as aulas de complexidade “difícil” ministradas em horários sequenciais devem ser evitadas;
		\item Aulas de nível “difícil” no último horário: aulas de complexidade “difícil” ministradas no último horário de cada dia deverão ser evitadas.
		\item Aulas de carga horária par: aulas com 2 ou 4 horas do turno diurno deverão ser alocadas fora do primeiro horário do dia.
		\item Aulas alocadas imediatamente antes, ou imediatamente depois, do horário de almoço devem ser evitadas. Ex: Uma aula com carga horária de 2 horas, sendo alocada de 11:00 às 13:00 ou 12:00 às 14:00
		\item Aulas com alocação iniciada fora do horário padrão. Ex: Uma aula sendo alocada de 13:00 às 14:00 ou 15:00 às 16:00
    \end{enumerate}
\end{itemize}

\section{Objetivos}
\label{sec-intro-objetivos}
Os objetivos deste trabalho são divididos em objetivo geral e objetivos específicos.

\subsection{Objetivo Geral}
\label{sec-intro-obg-geral}
Estudar o  impacto da escolha do fator de aleatoriedade e de movimentos, entre eles o \textit{Lecture Move}, na meta-heurística híbrida \textit{Greedy Randomized Adaptive Search Procedure} (GRASP) com \textit{Simulated Annealing} (SA), quando aplicado ao PTHU do DCOMP-CCENS-UFES

\subsection{Objetivos Específicos}
\label{sec-intro-obg-esp}
\begin{enumerate}[label=(\alph*)]
    \item Estudar novos movimentos a serem usados na busca local do problema abordado;
    \item Implementar um novo movimento na etapa de busca local;
    \item Realizar experimentos computacionais;
    \item Avaliar o desempenho do GRASP com SA usando o novo movimento implementado e diferentes fatores de aleatoriedade;
    \item Comparar os resultados obtidos com os resultados apresentados por \citeonline{vital2015grasp}, bem como as soluções construídas manualmente pelos coordenadores de curso.
    
\end{enumerate}
% ==============================================================================
% TCC2 - TULIO MULLER
% Capítulo 2 - Revisão de Literatura
% ==============================================================================
\chapter{Revisão de Literatura}
\label{sec-revisao}
Na década de 60, \citeonline{gotlieb1962construction}, através de resoluções a partir de análise combinatória, iniciou estudos relacionados ao problema de tabela-horário de instituições de ensino. Desde então, o tema ganhou atenção de pesquisadores \cite{schaerf1999survey, lewis2008survey}.

Devido a grande variedade de restrições específicas a cada instituição de ensino, o problema de tabela-horário não possui uma formulação única.

De acordo com \citeonline{schmidt1980timetable} e \citeonline{souza2000programaccao}, os primeiros trabalhos utilizavam heurísticas construtivas, mas desde então, pesquisadores passaram a utilizar outras técnicas para a resolução do problema, como exemplo, representando o problema com grafos e resolvendo com algoritmos de fluxo \cite{ostermann1982some}, ou coloração \cite{wood1969technique, neufeld1975generalized, cangolovic1991exact}. Outros trabalhos também usaram programação inteira mista \cite{tripathy1984school, ferland1985timetabling}.

Em \citeonline{lewis2008survey} pode ser observado que grande parte dos trabalhos recentes tem utilizado meta-heurísticas, tanto pela simplicidade, quanto pelos bons resultados alcançados. \textit{Simulated Annealing} \cite{mariano2014alns}, Busca Tabu \cite{machado2009proposta}, Algoritmos  Genéticos \cite{burke1994genetic} e Algoritmos Meméticos \cite{burke1995memetic} são as meta-heurísticas mais utilizadas. Em alguns trabalhos se observa também  combinação de meta-heurísticas \cite{vital2015grasp}.

Na primeira edição do campeonato internacional de tabela-horário (\textit{International Timetabling Competition} - ITC), realizado em 2002 (ITC-2002), \citeonline{kostuch2004university} desenvolve um algoritmo que constrói a tabela-horário em três etapas. Na primeira etapa é usado um algoritmo de coloração de grafos, que obtém uma solução inicial viável. Na segunda e terceira etapas aplica-se o \textit{Simulated Annealing}, sendo que em cada etapa é utilizada uma estrutura diferente de vizinhança.

\citeonline{muller2009itc2007} resolve o problema de tabela-horário de universidades da terceira formulação da segunda edição do ITC, realizado em 2007 (ITC-2007), usando \textit{Conflict-based Statistics} para gerar a solução inicial e \textit{Hill Climbing} combinado com \textit{Great Deluge} e \textit{Simulated Annealing} para refinamento da solução. \citeonline{rocha2013algoritmo} trata do mesmo problema aplicando a meta-heurística \textit{Greedy Randomized Adaptive Search Procedure} (GRASP), sendo testados os métodos \textit{Hill Climbing} e \textit{Simulated Annealing} como métodos de busca local, e o método \textit{Path-Relinking} também é aplicado, mas para intensificar a busca por soluções de boa qualidade. \citeonline{segatto2017} expande o trabalho de \citeonline{rocha2013algoritmo} e implementa novas vizinhanças. Dessa forma, proporcionou um maior entendimento do processo de busca no espaço de soluções e conseguiu melhorar o desempenho do algoritmo.

\citeonline{elmohamed1997comparison} investigaram diversas formas de aplicar o \textit{Simulated Annealing} no PTHU da Universidade de Syracuse. Dentre as configurações investigadas, os melhores resultados foram obtidos com resfriamento adaptativo, reaquecimento e um algoritmo baseado em regras, que é usado para gerar uma boa solução inicial. \citeonline{ceschia-gaspero-shaerf-2012} usam \textit{Simulated Annealing} para resolver a terceira formulação do ITC-2007. Os autores conseguiram boas respostas para as instâncias usadas no ITC-2007, e em algumas instâncias foram obtidas melhores soluções que as conhecidas até aquele momento.

\citeonline{mariano2014alns} utilizou a meta-heurística \textit{Adaptive Large Neighborhood Search} (ALNS), aplicada ao PTHU de um caso real do DCOMP-CCENS-UFES.

Ainda para o PTHU do DCOMP-CCENS-UFES, \citeonline{vital2015grasp} apresentou uma proposta de utilizar a meta-heurística \textit{Greedy Randomized Adaptive Search Procedure} (GRASP), com o \textit{Simulated Annealing} na fase de busca local, tendo encontrado soluções melhores do que as feitas manualmente pelos coordenadores de curso. \citeonline{carvalho2016etal} resolve o mesmo problema que \citeonline{mariano2014alns} e \citeonline{vital2015grasp}, porém usa um método guloso para construir uma solução inicial que é refinada através da meta-heurística \textit{Simulated Annealing}.
% ==============================================================================
% TCC2 - TULIO MULLER
% Capítulo 3 - Metodologia
% ==============================================================================
\chapter{Metodologia}
\label{sec-metodologia}
Nesse capítulo é apresentada a metodologia proposta para o desenvolvimento deste trabalho, em que é descrita a forma como o problema foi modelado, bem como as instâncias utilizadas nos testes do estudo proposto. 

\section{Modelagem do Problema}
\label{sec-met-modelagem}
A modelagem usada para representar o PTHU do DCOMP-CCENS-UFES foi baseada nos trabalhos de \citeonline{mariano2014alns} e \citeonline{vital2015grasp}, além das instâncias que estes trabalhos utilizaram. O problema é modelado com estruturas definidas pelo identificador de cada elemento relevante e das características seguintes ao identificador.

As estruturas são armazenadas em vetores que possibilitam estabelecer as relações entre os conjuntos de elementos que são trabalhados. Desse modo, as estruturas mais simples são: salas, professores, disciplinas, turmas e horários. Além dessas, é estabelecida uma estrutura denominada oferta, que promove a relação entre todas as outras. O levantamento de todas as entidades que referem-se aos períodos 2013/2 e 2016/1, usados nos testes deste trabalho, constam no Apêndice \ref{ap-dados}.

Para a entidade sala é definida uma estrutura capaz de armazenar as informações necessárias, de modo que a coluna ‘Prédio’ possui a sigla da localização, a coluna ‘Tipo’ significa se é um laboratório ou uma sala comum, além de seu número de identificação e a capacidade de alunos que ela comporta.


\begin{table}[h!]
\centering
\begin{tabular}{ | c | c | c | c | c |}\hline
\textbf{Id} & \textbf{Prédio} & \textbf{Tipo} & \textbf{Número} & \textbf{Capacidade}\\\hline
0 & PC & 0 & 9 & 40\\\hline
\end{tabular}
\caption{Estrutura da entidade sala.}
\label{tbl-met-sala}
\end{table}

Para a entidade professor é definida uma estrutura capaz de armazenar o identificador e o nome.

\begin{table}[h!]
\centering
\begin{tabular}{ | c | c | }\hline
\textbf{Id} & \textbf{Nome} \\\hline
0 & Edmar Hell Kampke \\\hline
\end{tabular}
\caption{Estrutura da entidade professor.}
\label{tbl-met-prof}
\end{table}

Para a entidade disciplina é criada uma estrutura com o identificador da disciplina, bem como seu código oficial, seu nome e o nível de dificuldade atribuído à esta disciplina.

\begin{table}[h!]
\centering
\begin{tabular}{ | c | c | c | c | }\hline
\textbf{Id} & \textbf{Código} & \textbf{Nome} & \textbf{Nível} \\\hline
0 & COM06842 & Programação I & 1 \\\hline
\end{tabular}
\caption{Estrutura da entidade disciplina.}
\label{tbl-met-disc}
\end{table}

Para a entidade tipo de sala é criada uma estrutura que armazena o identificador do tipo de sala e sua descrição, que pode ser uma sala comum ou um laboratório.

\begin{table}[h!]
\centering
\begin{tabular}{ | c | c | }\hline
\textbf{Id} & \textbf{Descrição} \\\hline
0 & Sala \\\hline
\end{tabular}
\caption{Estrutura da entidade tipo de sala.}
\label{tbl-met-tp-sala}
\end{table}

Para a entidade turma é definida uma estrutura que possui além do identificador da turma, o nome do curso, representado por uma sigla, o período da turma em questão e seu turno preferencial (0, de manhã, 1, à tarde e 2, à noite).

\begin{table}[h!]
\centering
\begin{tabular}{ | c | c | c | c | }\hline
\textbf{Id} & \textbf{Curso} & \textbf{Período} & \textbf{Turno Preferencial} \\\hline
0 & CC & 1 & 1 \\\hline
\end{tabular}
\caption{Estrutura da entidade turma.}
\label{tbl-met-turm}
\end{table}

Para entidade horário, é criada estrutura para armazenar cada faixa de horário, de modo que corresponda a um período unitário. Desse modo, é definido um vetor com o identificador do horário, o horário inicial e final e a duração em minutos.

\begin{table}[h!]
\centering
\begin{tabular}{ | c | c | c | c | }\hline
\textbf{Id} & \textbf{Horário Inicial} & \textbf{Horário Final} & \textbf{Duração} \\\hline
0 & 07h00 & 08h00 & 60 \\\hline
\end{tabular}
\caption{Estrutura da entidade horário.}
\label{tbl-met-horar}
\end{table}

Para a entidade oferta, é criada uma estrutura capaz de conter vários elementos. Sendo assim, uma estrutura que representa uma oferta deve possuir o seu identificador, o identificador da disciplina que será ofertada, o número de turmas que poderá se matricular nesta oferta, bem como os identificadores destas turmas, o número de vagas que serão disponibilizadas para os alunos, o turno em que esta oferta será destinada, o identificador do professor responsável por esta oferta, além do tipo de sala que será necessária para conduzir as aulas e a carga horária da oferta, que no caso abaixo, representa 2 horas-aula seguidas.

\begin{table}[h!]
\centering
\begin{tabular}{| c | c | M{1,5cm} | c | c | c | c | M{1,5cm} | M{1cm} |}
\hline
\textbf{Id} & \textbf{Disciplina} & \textbf{Nº de turmas} & \textbf{Turmas{[} {]}} & \textbf{Vagas} & \textbf{Turno} & \textbf{Professor} & \textbf{Tipo de Sala} & \textbf{CH} \\\hline
0 & 0 & 2 & 0 18 & 37 & 1 & 12 & 1 & 2 \\ \hline
\end{tabular}
\caption{Estrutura da entidade oferta}
\label{tbl-met-ofert}
\end{table}

Com todos os dados armazenados nas estruturas previamente definidas, o algoritmo inicia o processo de criação de uma solução inicial, que cria as três tabelas-horário essenciais para a solução final do PTHU, que são: a tabela-horário de cada sala, na qual deve constar todas as ofertas que serão lecionadas naquele local, a tabela-horário de cada professor, na qual ficam armazenados as ofertas que o professor leciona, além da tabela-horário de cada turma, que consta as ofertas das disciplinas da turma.

Nas Tabelas \ref{tbl_ofertas}, \ref{tbl_prof} e \ref{tbl_alocadas} são apresentadas como as tabelas-horário, respectivamente, de uma sala, de um professor e uma turma são representadas. As células que possuem número positivos (destaque em cinza) são aquelas com ofertas alocadas. Esses números representam o identificador da oferta ali alocada.

\begin{table}
\centering
\begin{tabular}{|c|c|c|c|c|c|}
\hline
\textbf{} & \multicolumn{5}{c|}{\textbf{Sala X}} \\ \hline
\textbf{Horário} & \textbf{Segunda} & \textbf{Terça} & \textbf{Quarta} & \textbf{Quinta} & \textbf{Sexta} \\ \hline
07:00-08:00 & -1 & -1 & -1 & -1 & -1 \\ \hline
08:00-09:00 & -1 & -1 & -2 & \cellcolor[HTML]{C0C0C0}17 & -2 \\ \hline
09:00-10:00 & -1 & -1 & -2 & \cellcolor[HTML]{C0C0C0}17 & -2 \\ \hline
10:00-11:00 & -1 & \cellcolor[HTML]{C0C0C0}92 & -2 & \cellcolor[HTML]{C0C0C0}27 & -1 \\ \hline
11:00-12:00 & -1 & \cellcolor[HTML]{C0C0C0}92 & -2 & \cellcolor[HTML]{C0C0C0}27 & -1 \\ \hline
13:00-14:00 & -2 & -2 & -2 & -2 & \cellcolor[HTML]{C0C0C0}18 \\ \hline
14:00-15:00 & -2 & -2 & -2 & -2 & \cellcolor[HTML]{C0C0C0}18 \\ \hline
15:00-16:00 & -1 & -2 & \cellcolor[HTML]{C0C0C0}2 & -1 & -2 \\ \hline
16:00-17:00 & -1 & -2 & \cellcolor[HTML]{C0C0C0}2 & -1 & -2 \\ \hline
18:00-19:00 & -2 & -2 & -2 & -2 & -2 \\ \hline
19:00-20:00 & -2 & -2 & -2 & -2 & -2 \\ \hline
20:00-21:00 & -2 & -2 & -2 & -2 & -2 \\ \hline
21:00-22:00 & -2 & -2 & -2 & -2 & -2 \\ \hline
22:00-23:00 & -2 & -2 & -2 & -2 & -2 \\ \hline
\end{tabular}
\caption{Representação de uma tabela-horário de sala com ofertas alocadas}
\label{tbl_ofertas}
\end{table}

\begin{table}
\centering
\begin{tabular}{|c|c|c|c|c|c|}
\hline
\textbf{} & \multicolumn{5}{c|}{\textbf{Professor Y}} \\ \hline
\textbf{Horário} & \textbf{Segunda} & \textbf{Terça} & \textbf{Quarta} & \textbf{Quinta} & \textbf{Sexta} \\ \hline
07:00-08:00 & -1  & -1  & -1  & -1  & -1  \\ \hline
08:00-09:00 & \cellcolor[HTML]{C0C0C0}11  & -1  & -1  & -1  & -1  \\ \hline
09:00-10:00 & \cellcolor[HTML]{C0C0C0}11  & -1  & -1  & -1  & -1  \\ \hline
10:00-11:00 & \cellcolor[HTML]{C0C0C0}11  & -1  & -1  & -1  & -1  \\ \hline
11:00-12:00 & \cellcolor[HTML]{C0C0C0}11  & -1  & -1  & -1  & -1  \\ \hline
13:00-14:00 & -1  & -1  & -1  & -1  & -1  \\ \hline
14:00-15:00 & -1  & -1  & -1  & -1  & -1  \\ \hline
15:00-16:00 & -1  &  \cellcolor[HTML]{C0C0C0}8  & -1  &  \cellcolor[HTML]{C0C0C0}4  & -1  \\ \hline
16:00-17:00 & -1  &  \cellcolor[HTML]{C0C0C0}8  & -1  &  \cellcolor[HTML]{C0C0C0}4  & -1  \\ \hline
18:00-19:00 & -1  &  \cellcolor[HTML]{C0C0C0}8  & -1  & -1  & -1  \\ \hline
19:00-20:00 & -1  &  \cellcolor[HTML]{C0C0C0}8  & -1  & -1  & -1  \\ \hline
20:00-21:00 & -1  & -1  & -1  & -1  & -1  \\ \hline
21:00-22:00 & -1  & -1  & -1  & -1  & -1  \\ \hline
22:00-23:00 & -1  & -1  & -1  & -1  & -1  \\ \hline
\end{tabular}
\caption{Representação de uma tabela-horário de professor com salas alocadas}
\label{tbl_prof}
\end{table}

\begin{table}
\centering
\begin{tabular}{|c|c|c|c|c|c|}
\hline
\textbf{} & \multicolumn{5}{c|}{\textbf{Turma Z}} \\ \hline
\textbf{Horário} & \textbf{Segunda} & \textbf{Terça} & \textbf{Quarta} & \textbf{Quinta} & \textbf{Sexta} \\ \hline
07:00-08:00 & -1  & -1  & -1  & -1  & -1  \\ \hline
08:00-09:00 & -1  & -1  & -1  & -1  & -1  \\ \hline
09:00-10:00 & -1  & -1  & -1  & -1  & -1  \\ \hline
10:00-11:00 &  \cellcolor[HTML]{C0C0C0}0  & -1  & -1  & -1  &  \cellcolor[HTML]{C0C0C0}2  \\ \hline
11:00-12:00 &  \cellcolor[HTML]{C0C0C0}0  & -1  & -1  & -1  &  \cellcolor[HTML]{C0C0C0}2  \\ \hline
13:00-14:00 &  \cellcolor[HTML]{C0C0C0}3  & -1  &  \cellcolor[HTML]{C0C0C0}7  &  \cellcolor[HTML]{C0C0C0}9  & -1  \\ \hline
14:00-15:00 &  \cellcolor[HTML]{C0C0C0}3  & -1  &  \cellcolor[HTML]{C0C0C0}7  &  \cellcolor[HTML]{C0C0C0}9  & -1  \\ \hline
15:00-16:00 & -1  & -1  &  \cellcolor[HTML]{C0C0C0}5  &  \cellcolor[HTML]{C0C0C0}9  & -1  \\ \hline
16:00-17:00 & -1  & -1  &  \cellcolor[HTML]{C0C0C0}5  &  \cellcolor[HTML]{C0C0C0}9  & -1  \\ \hline
18:00-19:00 & -1  & -1  & -1  & -1  & -1  \\ \hline
19:00-20:00 & -1  & -1  & -1  & -1  & -1  \\ \hline
20:00-21:00 & -1  & -1  & -1  & -1  & -1  \\ \hline
21:00-22:00 & -1  & -1  & -1  & -1  & -1  \\ \hline
23:00-22:00 & -1  & -1  & -1  & -1  & -1  \\ \hline
\end{tabular}
\caption{Representação de uma tabela-horário de turma com salas alocadas}
\label{tbl_alocadas}
\end{table}

\section{Função Objetivo}
\label{sec-met-fo}

Cada tabela-horário gerada recebe uma nota que reflete sua qualidade, a função \(f\) que calcula essa nota é chamada de Função Objetivo (FO). Cada restrição violada aumenta o valor da função objetivo de acordo com o peso da restrição. A melhor solução para um problema de tabela-horário é aquela que minimiza o valor da função objetivo.

Seja \textit{S} uma solução do PTHU abordado, e considerando que \(f_{RFt}(S)\) e \(f_{RFc}(S)\) são o número total de violações das restrições fortes e fracas de \textit{S}, respectivamente, então \textit{S} é uma solução viável, se todas as restrições fortes são satisfeitas, ou seja \(f_{RFt}(S) = 0\). De acordo com as restrições propostas por \citeonline{mariano2014alns} e posteriormente expandidas por \citeonline{vital2015grasp}, a função objetivo do PTHU do DCOMP-CCENS-UFES é calculada pela seguinte fórmula:
    \[f(S) = f_{RFt}(S) + f_{RFc}(S)\]
sendo
    \[f_{RFt}(S) = \omega_1\sum_{p=1}^{P}CP_p\ +\ \omega_2\sum_{t=1}^{T}CT_t\ +\ \omega_3\sum_{s=1}^{S}CS_s\ +\ \omega_4OFT\ +\ \omega_5\sum_{s=1}^{S}VS_s\ +\ \omega_6TSI\ +\ \omega_7D3H\]
e
    \[f_{RFc}(S) = \omega_8\sum_{p=1}^{P}IT_p\ +\ \omega_9\sum_{t=1}^{T}JH_t\ +\ \omega_{10}\sum_{t=1}^{T}PP_t\ +\ \omega_{11}\sum_{d=1}^{D}AS_d\ +\ \omega_{12}\sum_{p=1}^{P}ND_p\ +\] 
    \[\omega_{13}ASD\ +\ \omega_{14}ADU\ +\ \omega_{15}DHP\ +\ \omega_{16}AHA\ +\ \omega_{17}AHFP\]

Considerando que:

\begin{enumerate}[leftmargin=1.5\parindent]
    \item \(CP_p\): número de conflitos do professor \textit{p}, ou seja, o número de vezes que o professor \textit{p} ministra aula no mesmo dia e horário;
    \item \(CT_t\): número de conflitos da turma \textit{t}, ou seja, o número de vezes que os alunos da turma \textit{t} assistem mais de uma aula no mesmo dia e mesmo horário;
    \item \(CS_s\): número de conflitos da sala \textit{s}, ou seja, o número de vezes que a sala \textit{s} está atribuída a mais de uma turma no mesmo dia e mesmo horário;
    \item \(OFT\): número de violações em que uma oferta é alocada fora do turno;
    \item \(VS_s\): número de violações na capacidade da sala \textit{s}, ou seja, o número de turmas alocadas na sala \textit{s} cujo número de alunos é maior que a capacidade da sala;
    \item \(TSI\): número de aulas alocadas em salas de tipo “incompatível”, ou seja, se 10 aulas devem ser em laboratório e foram alocadas em salas normais, TSI = 10;
    \item \(D3H\): número de disciplinas de 3 horas aulas semanais alocadas fora dos horários “padrão” (primeiro e último horário, tanto do dia quanto da noite);
	\item \(IT_p\): diferença entre o primeiro e o último dia em que o professor \textit{p} ministra aulas em relação a um intervalo padrão \textit{I}, que deverá ser um parâmetro de entrada. Nesse caso, deve-se contabilizar apenas o que exceder \textit{I}. Ex: Considerando \(I = 3\) e que o professor 1 dá aulas de segunda a quinta, logo \(IT_1 = MAX (0, Quinta - Segunda + 1 - I) \rightarrow IT_1 = 1\); Caso o professor dê aulas de segunda a terça, \(IT_1 = MAX (0, \textit{Terça} - Segunda + 1 - I) \rightarrow IT_1 = 0\);
	\item \(JH_t\): número de janelas de horário da turma \textit{t}, ou seja, o número de horários vagos entre aulas ao longo da semana para a turma \textit{t};
	\item \(PP_t\): número aulas da turma \textit{t} fora do seu período preferencial (M,T,N). Ex: o período preferencial para a turma 1 é a tarde (T), logo, \(PP_1\) = número de aulas para essa turma alocadas no período da manhã;
	\item \(AS_d\): número de aulas seguidas da disciplina \textit{d}, ou seja, o número de vezes que a disciplina \textit{d} é repetida num mesmo dia;
	\item \(ND_p\): número de vezes que o professor \textit{p} ministra aula à noite (qualquer horário) em um dia e pela manhã (qualquer horário) no dia seguinte;
	\item \(ASD\): número de aulas seguidas de disciplinas de nível difícil, ou seja, o número de vezes ao longo da semana em que duas disciplinas “difíceis” são consecutivas;
	\item \(ADU\): número de aulas de disciplinas de nível difícil ministradas no último horário da tarde ou da noite;
	\item \(DHP\): número de aulas de disciplinas, do turno diurno, com carga horária par, alocadas no primeiro horário do dia;
	\item \(AHA\): número de aulas de disciplinas, do turno diurno, alocadas entre o horário de almoço;
	\item \(AHFP\): número de aulas de disciplinas com alocação iniciada fora do horário padrão. Ex: uma aula sendo alocada de 13:00 às 14:00 ou 15:00 às 16:00.
\end{enumerate}

O vetor \(\omega = [\omega_1, \omega_2, \omega_3, ..., \omega_{17}]\) contém os pesos das 17 restrições (7 fortes e 10 fracas) consideradas por \citeonline{vital2015grasp} e apresentadas no Capítulo \ref{sec-intro} deste trabalho.

\section{GRASP}
\label{sec-met-grasp}

A meta-heurística \textit{Greedy Randomized Adaptive Search Procedure} (GRASP), apresentada por \citeonline{feo1989probabilistic} foi inicialmente utilizada para tratar o problema de cobertura de conjuntos.

De acordo com \citeonline{rocha2013algoritmo}, o GRASP já foi aplicado com sucesso em diversos problemas de otimização desde sua proposta inicial, tais como: conjunto independente máximo \cite{feo1994greedy}, problema quadrático de alocação \cite{li1994greedy}, satisfatividade \cite{resende1996grasp}, planarização de grafos \cite{resende1997grasp}, roteamento de circuitos virtuais \cite{resende2003grasp}, entre outros.

No Algoritmo \ref{algo-grasp} é apresentado o pseudocódigo genérico da meta-heurística GRASP.

\begin{algorithm}[H]
\SetKwInput{Entrada}{Entrada}
\SetKwInput{Saida}{Saída}
\SetKwInput{Dados}{Dados}
\SetKwInput{Resultado}{Resultado}
\SetKwBlock{Inicio}{início}{fim}
\SetKwIF{Se}{SenaoSe}{Senao}{se}{então}{senão-se}{senão}{fim\ se}
\SetKwFor{Para}{para}{faça}{fim\ para}
\SetKwFor{ParaPar}{para}{faça em paralelo}{fim-para}
\SetKwFor{ParaCada}{para cada}{faça}{fim\ para\ cada}
\SetKwFor{ParaTodo}{para todo}{faça}{fim\ para\ todo}
\SetKwFor{Enqto}{enquanto}{faça}{fim\ enquanto}
\SetKwRepeat{Repita}{repita}{até}

\Entrada {\(Max_{Iter}\), \(\alpha\)}
\Saida{Solução \textit{\(S^{Melhor}\)}}
\Inicio{
    \(FO^{Melhor}\gets \infty\)\;
    \Para{\(i\gets 1\) \Ate \(Max_{Iter}\)}{
        \(S^{Inicial} \gets GeraSolucaoInicial(\alpha)\)\;
        \(S^{Atual} \gets BuscaLocal(S^{Inicial})\)\;
        \Se{\(f(S^{Atual}) < FO^{Melhor}\)}{
            \(S^{Melhor} \gets S^{Atual}\)\;
            \(FO^{Melhor}\gets f(S^{Atual})\)\;
        }
    }
}
\caption{Algoritmo do GRASP apresentado por
\citeonline{feo1989probabilistic}}
\label{algo-grasp}
\end{algorithm}

O algoritmo GRASP é um procedimento iterativo dividido em  duas fases, que são detalhadas nas próximas seções. Na primeira fase constrói-se uma solução inicial, e na segunda fase aplica-se uma busca local para melhorá-la. A resposta final é a melhor obtida após a execução de todas as iterações \cite{feo1995greedy}. Para essa meta-heurística são utilizados dois parâmetros principais: o número máximo de iterações \(Max_{Iter}\) e o fator de aleatoriedade (\(\alpha\)).

% \begin{table}
% \label{tbl-met-grade-turm}
% \centering
% \begin{tabular}{|c|c|c|c|c|c|}
% \hline
% \textbf{} & \multicolumn{5}{c|}{\textbf{Sala X}} \\ \hline
% \textbf{Horário} & \textbf{Segunda} & \textbf{Terça} & \textbf{Quarta} & \textbf{Quinta} & \textbf{Sexta} \\ \hline
% 07:00-08:00 & -1  & -1  & -1  & -1  & -1  \\ \hline
% 08:00-09:00 &  \cellcolor[HTML]{FFCE93}0  & -1  & -1  & -1  & -1  \\ \hline
% 09:00-10:00 &  \cellcolor[HTML]{FFCE93}0  & -1  & -1  & -1  & -1  \\ \hline
% 10:00-11:00 & -1  & -1  &  \cellcolor[HTML]{68CBD0}0  &  \cellcolor[HTML]{67FD9A}0  & -1  \\ \hline
% 11:00-12:00 & -1  & -1  &  \cellcolor[HTML]{68CBD0}0  &  \cellcolor[HTML]{67FD9A}0  & -1  \\ \hline
% 13:30-14:30 &  \cellcolor[HTML]{FD6864}0  & -1  & -1  & -1  & -1  \\ \hline
% 14:30-15:30 &  \cellcolor[HTML]{FD6864}0  & -1  & -1  & -1  & -1  \\ \hline
% 15:30-16:30 & -1  & -1  & -1  & -1  & -1  \\ \hline
% 16:30-17:30 & -1  & -1  & -1  & -1  & -1  \\ \hline
% 18:20-19:10 & -1  & -1  & -1  & -1  & -1  \\ \hline
% 19:10-20:00 & -1  & -1  & -1  & -1  & -1  \\ \hline
% 20:00-20:50 & -1  & -1  & -1  & -1  & -1  \\ \hline
% 21:00-21:50 & -1  & -1  & -1  & -1  & -1  \\ \hline
% 21:50-22:40 & -1  & -1  & -1  & -1  & -1  \\ \hline
% \end{tabular}
% \caption{Tabela-horário antes de uma iteração do construtor de soluções.}
% \end{table}

\subsection{Construção da solução inicial}
\label{sec-met-si}

Inicialmente, uma solução vazia (sem ofertas alocadas) para o PTHU do DCOMP-CCENS-UFES é criada. Nessa solução inicial, todas as células de cada tabela-horário terá valor negativo, sendo -1 se o horário estiver disponível e -2 se naquele horário já há uma oferta de outro departamento da UFES alocada.

A partir disso, uma lista de ofertas, contendo todas as ofertas, é criada. Essa lista é ordenada de forma decrescente pelo número de vagas. A partir da lista inicial, são feitas simulações de alocação em todas as posições da matriz de salas, com a oferta que possui o maior número de vagas. Cada simulação tem seu custo calculado, e cada uma dessas simulações é inserida em uma lista denominada Lista de Candidatos (LC).

A LC possui a posição em que a simulação de alocação foi efetuada e seu custo de alocação. A LC serve de base para a formação de uma outra lista, denominada Lista Restrita de Candidatos (LRC), que é composta por um intervalo da LC, depois dela ter sido ordenada crescentemente pelo custo de alocação das simulações. Esse intervalo é definido de acordo com o fator de aleatoriedade \(\alpha\) da seguinte forma: \([ c^{min}, c^{min} + \alpha(c^{max} - c^{min})]\) em que \(c^{min}\) representa o menor custo de alocação, \(c^{max}\) representa o maior custo de alocação, e \(\alpha \in [0,1]\).

Quando \(\alpha = 1\), LC tem seu elementos totalmente selecionados para LRC, se \(\alpha = 0\), LRC é preenchida somente com o primeiro elemento de LC. Um horário é escolhido de LRC aleatoriamente e a oferta é acrescentada à solução. É importante destacar que quando \(\alpha = 1\) a escolha do horário é feita de maneira totalmente aleatória, caso \(\alpha = 0\), o horário de LC que possui o menor custo será sempre escolhido, o que representa o uma escolha totalmente gulosa.

Após a alocação da oferta, ela é removida da lista de ofertas e o procedimento é repetido até que todas as ofertas sejam alocadas. Portanto, a construção da solução inicial termina quando todas as ofertas estiverem inseridas na tabela-horário.

Na Figura \ref{tbl_it_constr_sol} é exibido um exemplo de uma iteração da fase de construção de uma solução inicial no método GRASP. Nesse exemplo, a LC da oferta ainda não alocada e que possui maior número de vagas é criada com seis possíveis horários para alocação ordenados crescentemente pelo custo de alocação, calculado durante cada simulação. Em seguida, é formada a LRC, com apenas três horários, supondo o fator de aleatoriedade \(\alpha = 0,5\). Dentre esses três horários é feita uma escolha aleatória para definir em qual horário a oferta é alocada. Essa alocação é representada na Tabela \ref{tbl-met-gerad-soluc} com a alocação da oferta cujo identificador é 0 (destaque em laranja).

\begin{table}[h!]
\centering
\begin{tabular}{ccccccc}
\multicolumn{7}{c}{LC da oferta com maior nº de vagas} \\ \hline
\multicolumn{1}{|c|}{Posição} & \multicolumn{1}{c|}{{[}1{]}{[}2{]}{[}0{]}} & \multicolumn{1}{c|}{{[}3{]}{[}3{]}{[}2{]}} & \multicolumn{1}{c|}{{[}1{]}{[}2{]}{[}2{]}} & \multicolumn{1}{c|}{{[}3{]}{[}2{]}{[}3{]}} & \multicolumn{1}{c|}{{[}7{]}{[}5{]}{[}1{]}} & \multicolumn{1}{c|}{{[}1{]}{[}4{]}{[}4{]}} \\ \hline
\multicolumn{1}{|c|}{Custo} & \multicolumn{1}{c|}{400} & \multicolumn{1}{c|}{500} & \multicolumn{1}{c|}{566} & \multicolumn{1}{c|}{700} & \multicolumn{1}{c|}{885} & \multicolumn{1}{c|}{990} \\ \hline
\multicolumn{7}{c}{\Huge\(\Downarrow\) \vspace{2mm}} \\
\multicolumn{7}{c}{LRC definida de acordo com \(\alpha\)} \\ \cline{2-6}
\multicolumn{1}{c|}{} & \multicolumn{2}{c|}{Posição} & \multicolumn{1}{c|}{{[}1{]}{[}2{]}{[}0{]}} & \multicolumn{1}{c|}{{[}3{]}{[}3{]}{[}2{]}} & \multicolumn{1}{c|}{{[}1{]}{[}2{]}{[}2{]}} &  \\ \cline{2-6}
\multicolumn{1}{c|}{} & \multicolumn{2}{c|}{Custo} & \multicolumn{1}{c|}{400} & \multicolumn{1}{c|}{500} & \multicolumn{1}{c|}{566} &  \\ \cline{2-6}
\multicolumn{7}{c}{\Huge\(\Downarrow\) \vspace{2mm}} \\
\multicolumn{7}{c}{Escolha aleatória: candidato{[}1{]}, posição{[}3{]}{[}3{]}{[}2{]}}
\end{tabular}
\captionof{figure}{Exemplo de uma iteração do construtor de soluções}
\label{tbl_it_constr_sol}
\end{table}

\begin{table}[!htbp]
\centering
\begin{tabular}{|c|c|c|c|c|c|}
\hline
\textbf{} & \multicolumn{5}{c|}{\textbf{Sala X}} \\ \hline
\textbf{Horário} & \textbf{Segunda} & \textbf{Terça} & \textbf{Quarta} & \textbf{Quinta} & \textbf{Sexta} \\ \hline
07:00-08:00 & -1  & -1  & -1  & -1  & -1  \\ \hline
08:00-09:00 & -1  & -1  & -1  & -1  & -1  \\ \hline
09:00-10:00 & -1  & -1  & -1  & -1  & -1  \\ \hline
10:00-11:00 & -1  & -1  &  \cellcolor[HTML]{FFCE93}0  & -1  & -1  \\ \hline
11:00-12:00 & -1  & -1  &  \cellcolor[HTML]{FFCE93}0  & -1  & -1  \\ \hline
13:00-14:00 & -1  & -1  & -1  & -1  & -1  \\ \hline
14:00-15:00 & -1  & -1  & -1  & -1  & -1  \\ \hline
15:00-16:00 & -1  & -1  & -1  & -1  & -1  \\ \hline
16:00-17:00 & -1  & -1  & -1  & -1  & -1  \\ \hline
18:00-19:00 & -1  & -1  & -1  & -1  & -1  \\ \hline
19:00-20:00 & -1  & -1  & -1  & -1  & -1  \\ \hline
20:00-21:00 & -1  & -1  & -1  & -1  & -1  \\ \hline
21:00-22:00 & -1  & -1  & -1  & -1  & -1  \\ \hline
22:00-23:00 & -1  & -1  & -1  & -1  & -1  \\ \hline
\end{tabular}
\caption{Tabela-horário depois de uma iteração do procedimento de criar soluções.}
\label{tbl-met-gerad-soluc}
\end{table}


\subsection{Busca Local}
\label{sec-met-busca}

O objetivo da busca local é melhorar a solução inicialmente construída. A estratégia de busca local utilizada foi a meta-heurística \textit{Simulated Annealing}, pois ela se mostrou satisfatória em outros trabalhos, como de \citeonline{rocha2013algoritmo}, \citeonline{mariano2014alns} e \citeonline{carvalho2016etal}. 

\subsubsection{\textit{Simulated Annealing}}
\label{sec-met-sa}

A meta-heurística \textit{Simulated Annealing} tem como inspiração o processo da metalurgia conhecido como recozimento, em que derrete-se um metal e ele é esfriado lentamente até a solidificação, de forma controlada, para que sua estrutura se mantenha organizada e equilibrada \cite{kirkpatrick1983optimization}.

Os cinco principais parâmetros que o \textit{Simulated Annealing} possui são: a solução inicial (\emph{S}), a temperatura inicial (\(T_i\)), a temperatura final (\(T_f\)), a taxa de resfriamento (\(\beta\)) e o número de soluções vizinhas (\(N_v\)) que deverão ser geradas a cada iteração.

O algoritmo começa com uma temperatura inicial e vai sendo resfriada até chegar na temperatura final. Em cada temperatura (iteração), começando com a temperatura inicial \(T_i\), são gerados \(N_v\) vizinhos. Se o vizinho gerado tem uma FO menor que a da solução atual, esta é atualizada. Se o vizinho tiver uma FO maior, ele pode ser aceito com uma probabilidade \(e^{-\Delta f/T}\), em que \(\Delta f\) é a diferença entre os valores das FO's da solução vizinha e da solução atual, e \(T\) é a temperatura atual. Quanto maior \(\Delta f\), e menor a temperatura atual \(T\), menor será a chance de aceitar uma solução vizinha. Tipicamente, o algoritmo aceita grande diversificação de soluções no início, quando a temperatura está alta. À medida que a temperatura decresce, poucas soluções piores são aceitas e assim é intensificada a busca em uma determinada vizinhança. Ao final de cada iteração, a temperatura \(T\) é multiplicada pelo valor \(\beta\), sendo \(\beta \in [0,1]\) a taxa de resfriamento. O Algoritmo termina quando \(T < T_f\).

O Algoritmo \ref{algosa} apresenta o pseudocódigo da meta-heurística \textit{Simulated Annealing} aplicada na fase da busca local.

\begin{algorithm}
\caption{Algoritmo do \emph{SA} adaptado de \citeonline{rocha2013algoritmo}}
\label{algosa}
\SetKwInput{Entrada}{Entrada}
\SetKwInput{Saida}{Saída}
\SetKwInput{Dados}{Dados}
\SetKwInput{Resultado}{Resultado}
\SetKwBlock{Inicio}{início}{fim}
\SetKwIF{Se}{SenaoSe}{Senao}{se}{então}{senão-se}{senão}{fim\ se}
\SetKwFor{Para}{para}{faça}{fim\ para}
\SetKwFor{ParaPar}{para}{faça em paralelo}{fim-para}
\SetKwFor{ParaCada}{para cada}{faça}{fim\ para\ cada}
\SetKwFor{ParaTodo}{para todo}{faça}{fim\ para\ todo}
\SetKwFor{Enqto}{enquanto}{faça}{fim\ enquanto}
\SetKwRepeat{Repita}{repita}{até}

\Entrada {Solução S, \(T_i\), \(T_f\), \(\beta\), \(N_v\)}
\Saida{Solução \(S^{Melhor}\)}
\Inicio{
    \(T \gets T_i\)\;
    \(S^{Melhor} \gets S\)\;
    \Enqto{\(T > T_f\)}{
        \Para{\(i\gets 1\) \Ate \(Max_{Iter}\)}{    
            \(S^{Vizinho} \gets GeraVizinho(S^{Atual})\)\;
            \(\Delta f \gets f(S^{Vizinho}) - f(S^{Atual})\)\;
            \eSe{\(\Delta f < 0\)}{
                \(S^{Atual} \gets S^{Vizinho}\)\;
            }{
                \(\text{Gera um número aleatório}\ \rho \in [0, 1]\)\;
                \Se{\(\rho < e^{-\Delta f/T}\)}{
                    \(S^{Atual} \gets S^{Vizinho}\)\;
                }
            }
            \Se{\(f(S^{Vizinho}) < f(S^{Melhor})\)}{
                    \(S^{Melhor} \gets S^{Vizinho}\)\;
            }
        }
        \(T \gets T * \beta\)\;
    }
}
\end{algorithm}

Para gerar novas soluções vizinhas da solução atual, foram utilizados três movimentos. Os movimentos clássicos \textit{Move} e \textit{Swap} implementados por \citeonline{vital2015grasp}, e o movimento \textit{Lecture Move}, baseado neles e proposto por \citeonline{muller2009itc2007}.

\subsubsection{\textit{Lecture Move}}
\label{sec-met-lm}

Uma vizinhança é definida como um movimento ou um conjunto de movimentos realizados que geram um subconjunto de soluções no espaço de soluções. De acordo com \citeonline{schaerf1999survey}, os clássicos \textit{Move} e \textit{Swap} são os movimentos de vizinhança que estão entre os mais utilizados para problemas de tabela-horário.

O movimento \textit{Move} consiste em mover uma oferta escolhida aleatoriamente para uma posição não ocupada na tabela-horário, também escolhida aleatoriamente. Na Figura \ref{move} ocorre a realização de um movimento do tipo \textit{Move}. Na Figura \ref{move:a} é apresentada uma tabela-horário antes da execução do movimento, e na Figura \ref{move:b}, a tabela-horário após a execução do movimento. A oferta de código \textbf{92} previamente alocada na Terça-feira, no horário de 10:00-12:00 (Figura \ref{move:a}) é movida para Segunda-feira no horário de 08:00-10:00 (Figura \ref{move:b}). Nota-se que no \textit{Move}, a oferta selecionada pode ser movida para qualquer umas das células (ofertas) vazias, desde que essa mudança não cause uma inviabilidade na solução, ou seja, desde que não infrinja nenhuma restrição forte.

\begin{figure}[h!]
    \tiny
    \begin{subfigure}[b]{0.5\textwidth}
    \centering
    \begin{tabular}{|c|c|c|c|c|}
        \hline
        & \multicolumn{3}{c|}{Sala X} &  \\ \hline
        Horário & Segunda & Terça & Quarta & $\cdots$ \\ \hline
        07:00-08:00 & -1 & -1 & -2 & $\cdots$ \\ \hline
        08:00-09:00 & -1 & -1 & 27 & $\cdots$ \\ \hline
        09:00-10:00 & -1 & -1 & 27 & $\cdots$ \\ \hline
        10:00-11:00 & -1 & \cellcolor[HTML]{FFCE93}92 & -2 & $\cdots$ \\ \hline
        11:00-12:00 & -1 & \cellcolor[HTML]{FFCE93}92 & -2 & $\cdots$ \\ \hline
        $\vdots$ & $\vdots$ & $\vdots$ & $\vdots$ &  \\ \hline
    \end{tabular}
    \caption{Antes do \textit{Move}}
    \label{move:a}
    \end{subfigure}
    \hfill
    \begin{subfigure}[b]{0.5\textwidth}
    \centering
    \begin{tabular}{|c|c|c|c|c|}
        \hline
        & \multicolumn{3}{c|}{Sala X} &  \\ \hline
        Horário & Segunda & Terça & Quarta & $\cdots$ \\ \hline
        07:00-08:00 & -1 & -1 & -2 & $\cdots$ \\ \hline
        08:00-09:00 & \cellcolor[HTML]{FFCE93}92 & -1 & 27 & $\cdots$ \\ \hline
        09:00-10:00 & \cellcolor[HTML]{FFCE93}92 & -1 & 27 & $\cdots$ \\ \hline
        10:00-11:00 & -1 & -1 & -2 & $\cdots$ \\ \hline
        11:00-12:00 & -1 & -1 & -2 & $\cdots$ \\ \hline
        $\vdots$ & $\vdots$ & $\vdots$ & $\vdots$ &  \\ \hline
    \end{tabular}
    \caption{Depois do \textit{Move}}
    \label{move:b}
    \end{subfigure}
    \caption{Exemplo de um movimento do tipo \textit{Move}}
    \label{move}
\end{figure}

O movimento \textit{Swap} consiste na troca entre duas ofertas escolhidas aleatoriamente na tabela-horário. Na Figura \ref{swap} é apresentado um exemplo de aplicação do \textit{Swap}. Na Figura \ref{swap:a} está apresentada uma tabela-horário antes da execução do movimento, e na Figura \ref{swap:b}, a tabela-horário após a execução do movimento. A oferta de código \textbf{92} previamente alocada na Terça-feira, no horário de 10:00-12:00 (Figura \ref{swap:a}) é movida para Quarta-feira no horário de 08:00-10:00, enquanto a oferta de código \textbf{27}, previamente alocada nessa posição passa a ocupar o lugar da oferta \textbf{92} (Figura \ref{swap:b}). Nota-se que no \textit{Swap}, as ofertas selecionadas podem ser movidas para qualquer umas das células (ofertas) compatíveis, desde que essa mudança não cause uma inviabilidade na solução, ou seja, desde que não infrinja nenhuma restrição forte. 


\begin{figure}[h!]
    \tiny
    \begin{subfigure}[b]{0.5\textwidth}
    \centering
    \begin{tabular}{|c|c|c|c|c|}
        \hline
        & \multicolumn{3}{c|}{Sala X} &  \\ \hline
        Horário & Segunda & Terça & Quarta & $\cdots$ \\ \hline
        07:00-08:00 & -1 & -1 & -2 & $\cdots$ \\ \hline
        08:00-09:00 & -1 & -1 & \cellcolor[HTML]{CBCEFB}27 & $\cdots$ \\ \hline
        09:00-10:00 & -1 & -1 & \cellcolor[HTML]{CBCEFB}27 & $\cdots$ \\ \hline
        10:00-11:00 & -1 & \cellcolor[HTML]{FFCE93}92 & -2 & $\cdots$ \\ \hline
        11:00-12:00 & -1 & \cellcolor[HTML]{FFCE93}92 & -2 & $\cdots$ \\ \hline
        $\vdots$ & $\vdots$ & $\vdots$ & $\vdots$ &  \\ \hline
    \end{tabular}
    \caption{Antes do \textit{Swap}}
    \label{swap:a}
    \end{subfigure}
    \hfill
    \begin{subfigure}[b]{0.5\textwidth}
    \centering
    \begin{tabular}{|c|c|c|c|c|}
        \hline
        & \multicolumn{3}{c|}{Sala X} &  \\ \hline
        Horário & Segunda & Terça & Quarta & $\cdots$ \\ \hline
        07:00-08:00 & -1 & -1 & -2 & $\cdots$ \\ \hline
        08:00-09:00 & -1 & -1 & \cellcolor[HTML]{FFCE93}92 & $\cdots$ \\ \hline
        09:00-10:00 & -1 & -1 & \cellcolor[HTML]{FFCE93}92 & $\cdots$ \\ \hline
        10:00-11:00 & -1 & \cellcolor[HTML]{CBCEFB}27 & -2 & $\cdots$ \\ \hline
        11:00-12:00 & -1 & \cellcolor[HTML]{CBCEFB}27 & -2 & $\cdots$ \\ \hline
        $\vdots$ & $\vdots$ & $\vdots$ & $\vdots$ &  \\ \hline
    \end{tabular}
    \caption{Depois do \textit{Swap}}
    \label{swap:b}
    \end{subfigure}
    \caption{Exemplo de um movimento do tipo \textit{Swap}}
    \label{swap}
\end{figure}


Os movimentos \textit{Move} e \textit{Swap} foram implementados no trabalho de \citeonline{vital2015grasp}. Neste trabalho, utilizando como base os movimentos \textit{Move} e \textit{Swap} implementados por \citeonline{vital2015grasp}, foi implementado um novo movimento, denominado \textit{Lecture Move}. O movimento \textit{Lecture Move} consiste na troca entre duas ofertas ou entre uma oferta e um horário vazio. Em outras palavras, o \textit{Lecture Move} corresponde a uma combinação de \textit{Move} e \textit{Swap}. 

No Algoritmo \ref{algo_lm} é apresentado o pseudocódigo do movimento \textit{Lecture Move}.


\begin{algorithm}[H]
\SetKwInput{Entrada}{Entrada}
\SetKwInput{Saida}{Saída}
\SetKwInput{Dados}{Dados}
\SetKwInput{Resultado}{Resultado}
\SetKwBlock{Inicio}{início}{fim}
\SetKwIF{Se}{SenaoSe}{Senao}{se}{então}{senão\ se}{senão}{fim\ se}
\SetKwFor{Para}{para}{faça}{fim\ para}
\SetKwFor{ParaPar}{para}{faça em paralelo}{fim-para}
\SetKwFor{ParaCada}{para cada}{faça}{fim\ para\ cada}
\SetKwFor{ParaTodo}{para todo}{faça}{fim\ para\ todo}
\SetKwFor{Enqto}{enquanto}{faça}{fim\ enquanto}
\SetKwRepeat{Repita}{repita}{até}

\Entrada {Solução \(S\)}
\Saida{Solução \(S^{Vizinha}\)}
\Inicio{
    \(Trocou \gets false\)\;
    \Enqto{\textit{não Trocou}}{
        Escolhe aleatoriamente posição 1 (p1)\;
        Escolhe aleatoriamente posição 2 (p2)\;
        \uSe{p1 possui oferta alocada \textbf{E} p2 possui oferta alocada}{
            \(S^{Vizinha} \gets Swap(S, p1, p2)\)\;
            \(Trocou \gets true\)\;
        }
        \uSenaoSe{p1 possui oferta alocada \textbf{E} p2 NÃO possui oferta alocada}{
            \(S^{Vizinha} \gets Move(S, p1, p2)\)\;
            \(Trocou \gets true\)\;
        }
        \uSenaoSe{p1 NÃO possui oferta alocada \textbf{E} p2 possui oferta alocada}{
            \(S^{Vizinha} \gets Move(S, p2, p1)\)\;
            \(Trocou \gets true\)\;
        }
    }
}
\caption{Algoritmo do \textit{Lecture Move}}
\label{algo_lm}
\end{algorithm}

% \section{Critérios de parada e Aceitação}
% \label{sec-met-parada}

% Para o critério de aceitação do \textit{Simulated Annealing}, dada uma solução \(S\), uma solução vizinha \(S'\) é aceita se \(f(S') < f(S)\). A \(S'\) também pode ser aceita caso \(f(S') \geq f(S)\), mas apenas com uma probabilidade de aceitação \(e^{-\Delta f/t}\). A temperatura \(T\) inicia com um valor é multiplicada por \(\beta\) a cada iteração, sendo \(\beta \in [0, 1]\) a taxa de resfriamento.

% Os valores de \(T_i,\ T_f,\ \beta,\ N_v\) foram calibrados durante os experimentos. É estipulado um intervalo de tempo que é iniciado sempre que uma iteração começa, e a cada solução vizinha gerada que melhore a função objetivo, este contador de tempo é zerado, para que possíveis novos vizinhos possam ser gerados na mesma iteração, ocasionando novas melhoras. Caso o tempo limite seja alcançado sem nenhuma melhora, a busca local avança para uma nova iteração.
% ==============================================================================
% TCC2 - TULIO MULLER
% Capítulo 4 - Resultados Computacionais
% ==============================================================================
\chapter{Resultados Computacionais}
\label{sec-resultados}

Este capítulo apresenta os experimentos computacionais e os resultados alcançados neste trabalho, constando as análises comparativas com outro trabalho na literatura, bem como as soluções manuais elaboradas pelos coordenadores de curso do DCOMP-CCENS-UFES. Mostra também as escolhas dos parâmetros do GRASP, os parâmetros específicos da busca local \textit{Simulated Annealing} e os movimentos utilizados para a geração de soluções vizinhas.

\section{Escolha de Parâmetros}
\label{sec-res-param}

O algoritmo GRASP possui dois parâmetros: o número máximo de iterações \(Max_{Iter}\) e o valor \(\alpha\), que define a forma como a construção da solução inicial será conduzida. Se o parâmetro \(\alpha\) tem seu valor muito próximo a 0 (zero), o algoritmo de construção inicial tem comportamento predominantemente guloso, e produz soluções de boa qualidade, porém pouco diversificadas. Caso \(\alpha\) é mais próximo de 1 (um), as soluções possuem características mais aleatórias, porém com a desvantagem de possuir um valor de função objetivo mais alto. Para o fator de aleatoriedade \(\alpha\) foram escolhidos 5 valores distintos: 0, 0.25, 0.5, 0.75 e 1, para serem usados nos experimentos. Além desses valores, em uma das execuções do GRASP o valor usado não foi fixo, ou seja, em cada iteração um valor aleatório de \(\alpha\) no intervalo [0,1] é escolhido e usado. Para efeitos de comparação com o trabalho de \citeonline{vital2015grasp}, o valor de  \(\alpha = 0.15\) também foi testado. E da mesma forma também que \citeonline{vital2015grasp}, o parâmetro \(Max_{Iter}\) foi definido pelo tempo de execução: 500 segundos. 

Para o \textit{Simulated Annealing}, os parâmetros foram calibrados de forma a permitir um certo grau de diversificação no início da busca e maior intensificação no final do processo. Para cada valor de \(\alpha\) e para cada instância (2013/2 e 2016/1) o \textit{SA} foi testado com dois conjuntos de movimentos: \textit{Move} e \textit{Swap} com 50\% de probabilidade cada e \textit{Lecture Move}.

Na tabela \ref{tbl_param} são apresentados os parâmetros definidos durante a calibração para o algoritmo SA.


\begin{table}[!htbp]
\centering
\begin{tabular}{c|c|c}
\hline \hline
\textbf{Parâmetro} & \textbf{Descrição} & \textbf{Valor} \\ \hline
\(T_{i}\) & Temperatura inicial para o SA & 1000 \\
\(T_{f}\) & Temperatura de congelamento para o SA & 0.01 \\
\(\beta\) & Taxa de resfriamento para o SA & 0.975 \\
\(N_{v}\) & Número máximo de iterações para o SA & 500 \\ \hline \hline
\end{tabular}
\caption{Parâmetros e valores dos algoritmos}
\label{tbl_param}
\end{table}

Para comparações dos resultados, os valores das penalizações (vetor \(\omega\)) das restrições fortes e fracas da função objetivo, usadas neste trabalho, foram as mesmas adotadas por \citeonline{vital2015grasp}.

\begin{table}[!htbp]
\centering
\begin{tabular}{c|c}
\hline \hline
\textbf{Penalização} & \textbf{Valor} \\ \hline
\(\omega_{1}\) & 5000 \\
\(\omega_{2}\) & 5000 \\
\(\omega_{3}\) & 5000 \\
\(\omega_{4}\) & 5000 \\
\(\omega_{5}\) & 5000 \\
\(\omega_{6}\) & 300 \\
\(\omega_{7}\) & 5000 \\
\(\omega_{8}\) & 10 \\
\(\omega_{9}\) & 20 \\
\(\omega_{10}\) & 4 \\
\(\omega_{11}\) & 600 \\
\(\omega_{12}\) & 10 \\
\(\omega_{13}\) & 10 \\
\(\omega_{14}\) & 10 \\
\(\omega_{15}\) & 500 \\
\(\omega_{16}\) & 200 \\
\(\omega_{17}\) & 150 \\ \hline \hline
\end{tabular}
\caption{Penalizações e valores do PTHU}
\label{tbl_penal}
\end{table}


\section{Detalhes de Implementação}
\label{sec-res-impl}

Os algoritmos descritos neste trabalho foram implementados na linguagem C++ e os testes computacionais foram realizados em uma máquina com CPU Intel Core i5-7200U CPU @ 2.50GHz com 12,0 GB de mémória RAM e sistema operacional Microsoft Windows 10.

As ofertas das disciplinas \textit{Trabalho de Conclusão de Curso I}, \textit{Trabalho de Conclusão de Curso II} e \textit{Estágio em Informática} não foram consideradas para o problema, visto que são disciplinas que não necessitam serem consideradas no horário e não requerem espaço físico.

\section{Análise dos Resultados}
\label{sec-res-anal}

O algoritmo foi executado 10 vezes para cada instância (2013/2 e 2016/1), para cada valor de \(\alpha\) (0, 0.15, 0.25. 0.5, 0.75, 1 e Aleatório) e para cada conjunto de movimentos da busca local (\textit{Move}/\textit{Swap} e \textit{Lecture Move}). Ao final das execuções, todas as soluções obtidas apresentaram-se viáveis, ou seja, não violaram nenhuma restrição forte para o problema.

Nas Tabelas \ref{tbl_resultados_alfa152013} e \ref{tbl_resultados_alfa152016}, pode-se verificar que, mantendo \(\alpha = 0.15\), mesmo valor usado por \citeonline{vital2015grasp}, para ambos os semestres, a solução com o movimento \textit{Lecture Move} apresentou melhores resultados. Para 2013/2, a média dos resultados foi \(18,37\%\) melhor do que a solução com os movimentos \textit{Move/Swap}, enquanto para o semestre 2016/1, a melhora foi de \(20,08\%\).

A melhoria da solução com o \textit{Lecture Move} em relação ao  \textit{Move/Swap} pode ser explicada pela eliminação da escolha aleatória de qual movimento seria realizado. Com a introdução do \textit{Lecture Move}, primeiro são escolhidos os horários, se os 2 horários escolhidos já possuem aulas alocadas, tenta-se realizar o \textit{Swap}, mas se somente um horário já está alocado, tenta-se fazer o \textit{Move}.

\begin{table}[!htbp]
\centering
\begin{tabular}{|c|c|c|}
\hline
 & Move/Swap & Lecture Move \\ \hline
Melhor Solução & 406 & 306 \\ \hline
Média & 463,6 & 331,4 \\ \hline
Tempo & 267,012 & 255,710 \\ \hline
\end{tabular}
\caption{Melhores Soluções, médias das soluções e tempo para 2013/2 com \(\alpha\) = 0.15}
\label{tbl_resultados_alfa152013}
\end{table}

\begin{table}[!htbp]
\centering
\begin{tabular}{|c|c|c|}
\hline
 & Move/Swap & Lecture Move \\ \hline
Melhor Solução & 278 & 208 \\ \hline
Média & 298,8 & 238,0 \\ \hline
Tempo & 231,525 & 301,577 \\ \hline
\end{tabular}
\caption{Melhores Soluções, médias das soluções e tempo para 2016/1 com \(\alpha\) = 0.15}
\label{tbl_resultados_alfa152016}
\end{table}


Na Tabela \ref{tbl_resultados_tulio} são mostrados as melhores resultados, a média dos valores e a média de tempo utilizando o movimento de busca local \textit{Lecture Move} e variando o valor de \(\alpha\). Pode-se observar que a melhor média de soluções para o semestre 2013/2 foi obtida com \(\alpha\) = 0, com a média dos resultados de 320,8, enquanto que para 2016/1 foi com \(\alpha\) = 0.75, com a média dos resultados de 233,4.

Para o semestre 2013/2, a melhor solução encontrada possui  \(\alpha\) = 0, isso significa que uma estratégia totalmente gulosa foi melhor aproveitada para essa instância.
Já para o semestre 2016/1, com a melhor solução possuindo  \(\alpha\) = 0.75, percebe-se que uma maior aleatoriedade foi mais benéfica para essa instância.

\begin{table}[!htbp]
\begin{adjustbox}{width=1.2\textwidth,center=\textwidth}
\centering
\begin{tabular}{|cc|c|c|c|c|c|c|c|}
\hline
 &  & 0 & 0.15 & 0.25 & 0.5 & 0.75 & 1.0 & Aleatório \\ \hline
\multicolumn{1}{|c|}{} & Melhor Solução & 268 & 306 & 324 & 306 & 296 & 302 & 282 \\ \cline{2-9} 
\multicolumn{1}{|c|}{\begin{tabular}[c]{@{}c@{}}\(2013/\)2\\ \textit{Lecture Move}\end{tabular}} & Média & 320,8 & 331,4 & 351,8 & 347,8 & 347,0 & 350,0 & 337,8 \\ \cline{2-9} 
\multicolumn{1}{|c|}{} & Tempo & 183,860 & 255,710 & 261,245 & 231,892 & 363,247 & 197,625 & 336,832 \\ \hline
\multicolumn{1}{|c|}{} & Melhor Solução & 214 & 208 & 200 & 220 & 198 & 214 & 222 \\ \cline{2-9} 
\multicolumn{1}{|c|}{\begin{tabular}[c]{@{}c@{}}\(2016/1\)\\ \textit{Lecture Move}\end{tabular}} & Média & 247,4 & 238,0 & 234,4 & 242,8 & 233,4 & 243,8 & 234,8 \\ \cline{2-9} 
\multicolumn{1}{|c|}{} & Tempo & 342,737 & 301,577 & 215,537 & 271,353 & 185,639 & 261,776 & 285,848 \\ \hline
\end{tabular}
\end{adjustbox}
\caption{Melhores soluções, média das soluções e tempo para o Movimento Lecture Move com variação de alfa}
\label{tbl_resultados_tulio}
\end{table}

Na Tabela \ref{tbl_resultados_comp} pode-se verificar a melhor solução, os tempos médios de execução e a média da FO obtidas neste trabalho, bem como da solução proposta por \citeonline{vital2015grasp} e o valor da FO da solução construída manualmente pelos coordenadores de curso.

Para o semestre de 2013/2 a solução manual obtida pelo DCOMP resultou em uma FO = 760, já para o trabalho de \citeonline{vital2015grasp} a melhor solução encontrada foi FO = 466. Assim é possível verificar que a solução apresentada por este trabalho alcançou uma melhora percentual de \(57,79\%\) em relação à solução manual 2013/2 e uma melhora de \(20.98\%\) quando comparada à solução de \citeonline{vital2015grasp}.

Já para o semestre de 2016/1, a solução manual obtida pelo DCOMP teve sua FO calculada em 2374. Assim, pode-se verificar que a solução obtida por este trabalho apresenta uma melhora de \(90.17\%\) comparada com a solução manual e de \(21.63\%\) quando comparada à solução de \citeonline{vital2015grasp}.

\begin{table}[!htbp]
% \begin{adjustbox}{width=1.2\textwidth,center=\textwidth}
\centering
\begin{tabular}{|cc|c|c|c|}
\hline
 &  & Manual & \citeonline{vital2015grasp} & GRASP+SA+LM \\ \hline
\multicolumn{1}{|c|}{} & Melhor Solução & 760 & 406 & 268 \\ \cline{2-5} 
\multicolumn{1}{|c|}{2013/2} & Média & - & 463,6 & 320,8 \\ \cline{2-5} 
\multicolumn{1}{|c|}{} & Tempo & - & 267,012 & 206,482 \\ \hline
\multicolumn{1}{|c|}{} & Melhor Solução & 2374 & 278 & 198 \\ \cline{2-5} 
\multicolumn{1}{|c|}{2016/1} & Média & - & 297,8 & 233,4 \\ \cline{2-5} 
\multicolumn{1}{|c|}{} & Tempo & - & 231,525 & 185,639 \\ \hline
\end{tabular}
% \end{adjustbox}
\caption{Melhores soluções, média das soluções e tempo para o Movimento Lecture Move}
\label{tbl_resultados_comp}
\end{table}


% Na Tabela \ref{tbl_medias_2016} pode-se verificar os tempos médios de execução e a média da FO dos outros fatores de aleatoriedade, bem como da solução proposta por \citeonline{vital2015grasp} e valor da FO da solução construída manualmente pelos coordenadores de curso.

% \begin{table}[!htbp]
% \begin{adjustbox}{width=1.2\textwidth,center=\textwidth}
% \centering
% \begin{tabular}{c|c|c|c|c|c|c|c|c|c}
% \hline \hline
%  & Manual & Vital & 0 & 0,15 & 0,25 & 0,50 & 0,75 & 1 & Aleatório \\ \hline
% Função Objetivo & 760 & 406 & 320,8 & 331,4 & 351,8 & 347,8 & 347 & 350 & 337,8 \\
% Tempo (segundos) & - & 116,85 & 206,482 & 255,71 & 261,245 & 231,892 & 363,247 & 197,625 & 336,832\\
% \hline \hline
% \end{tabular}
% \end{adjustbox}
% \caption{Média das Funções objetivo e tempo para 2013/2 com fatores de aleatoriedade diferentes}
% \label{tbl_medias_2013}
% \end{table}

% \begin{table}[!htbp]
% \begin{adjustbox}{width=1.2\textwidth,center=\textwidth}
% \centering
% \begin{tabular}{c|c|c|c|c|c|c|c|c|c}
% \hline \hline
%  & Manual & Vital & 0 & 0,15 & 0,25 & 0,50 & 0,75 & 1 & Aleatório \\ \hline
% Função Objetivo & 2374 & 297,8 & 247,4 & 238 & 234,4 & 242,8 & 233,4 & 243,8 & 234,8 \\
% Tempo (segundos) & - & 231,525 & 342,737 & 301,577 & 215,537 & 271,353 & 185,639 & 261,776 & 285,848\\
% \hline \hline
% \end{tabular}
% \end{adjustbox}
% \caption{Média das Funções objetivo e tempo para 2016/1 com fatores de aleatoriedade diferentes}
% \label{tbl_medias_2016}
% \end{table}



% \begin{table}[!htbp]
% \begin{adjustbox}{width=1.2\textwidth,center=\textwidth}
% \begin{tabular}{c|c|c|c|c|c|c|c|c|c}
% \hline \hline
% Penalização & \begin{tabular}[c]{@{}c@{}}Melhor Solução \\ Manual 2016/1\end{tabular} & \begin{tabular}[c]{@{}c@{}}Melhor Solução \\ Vital\end{tabular} & \begin{tabular}[c]{@{}c@{}}Melhor solução\\ alfa = 0\end{tabular} & \begin{tabular}[c]{@{}c@{}}Melhor solução\\ alfa = 0.15\end{tabular} & \begin{tabular}[c]{@{}c@{}}Melhor solução\\ alfa = 0.25\end{tabular} & \begin{tabular}[c]{@{}c@{}}Melhor solução\\ alfa = 0.50\end{tabular} & \begin{tabular}[c]{@{}c@{}}Melhor solução\\ alfa = 0.75\end{tabular} & \begin{tabular}[c]{@{}c@{}}Melhor solução\\ alfa = 1.00\end{tabular} & \begin{tabular}[c]{@{}c@{}}Melhor solução\\ alfa aleatório\end{tabular} \\
% \hline
% \(CP_p\) &  0 & 0 & 0 & 0 & 0 & 0 & 0 & 0 & 0 \\
% \(CT_t\) &  0 & 0 & 0 & 0 & 0 & 0 & 0 & 0 & 0 \\
% \(CS_s\) &  0 & 0 & 0 & 0 & 0 & 0 & 0 & 0 & 0 \\
% \(OFT\) &  0 & 0 & 0 & 0 & 0 & 0 & 0 & 0 & 0 \\
% \(VS_s\) &  0 & 0 & 0 & 0 & 0 & 0 & 0 & 0 & 0 \\
% \(TSI\) &  0 & 0 & 0 & 0 & 0 & 0 & 0 & 0 & 0 \\
% \(DH3\) &  0 & 0 & 0 & 0 & 0 & 0 & 0 & 0 & 0 \\
% \(IT_p\) &  10 & 10 & 8 & 7 & 7 & 5 & 7 & 7 & 7 \\
% \(JH_t\) &  6 & 0 & 0 & 0 & 0 & 0 & 0 & 0 & 0 \\
% \(PP_t\) &  0 & 24 & 17 & 14 & 16 & 14 & 19 & 13 & 18 \\
% \(AS_d\) &  0 & 0 & 0 & 0 & 0 & 0 & 0 & 0 & 0 \\
% \(ND_p\) &  1 & 0 & 0 & 0 & 0 & 0 & 0 & 0 & 0 \\
% \(ASD\) &  34 & 10 & 5 & 8 & 9 & 8 & 7 & 9 & 6 \\
% \(ADU\) &  19 & 11 & 7 & 10 & 10 & 12 & 8 & 9 & 8 \\
% \(DHP\) &  0 & 0 & 0 & 0 & 0 & 0 & 0 & 0 & 0 \\
% \(AHA\) &  0 & 0 & 0 & 0 & 0 & 0 & 0 & 0 & 0 \\
% \(AHFP\) &  0 & 0 & 0 & 0 & 0 & 0 & 0 & 0 & 0 \\\hline
% Função Objetivo &  760 & 406 & 268 & 306 & 324 & 306 & 296 & 302 & 282 \\
% \(Tempo\ (segundos)\) &  - & 116,85 & 183,86 & 400,17 & 304,2 & 24,94 & 78,82 & 81,53 & 397,98 \\
% \(Alfa\) & - & 0,15 & 0,00 & 0,15 & 0,25 & 0,50 & 0,75 & 1,00 & 0,972076 \\
% \hline \hline
% \end{tabular}
% \end{adjustbox}
% \caption{Comparação das melhores soluções para 2013/2 com fatores de aleatoriedade diferentes}
% \label{tbl_melhores_2013}
% \end{table}

% \begin{table}[!htbp]
% \begin{adjustbox}{width=1.2\textwidth,center=\textwidth}
% \begin{tabular}{c|c|c|c|c|c|c|c|c|c}
% \hline \hline
% Penalização & \begin{tabular}[c]{@{}c@{}}Melhor Solução \\ Manual 2013/2\end{tabular} & \begin{tabular}[c]{@{}c@{}}Melhor Solução \\ Vital\end{tabular} & \begin{tabular}[c]{@{}c@{}}Melhor solução\\ alfa = 0\end{tabular} & \begin{tabular}[c]{@{}c@{}}Melhor solução\\ alfa = 0.15\end{tabular} & \begin{tabular}[c]{@{}c@{}}Melhor solução\\ alfa = 0.25\end{tabular} & \begin{tabular}[c]{@{}c@{}}Melhor solução\\ alfa = 0.50\end{tabular} & \begin{tabular}[c]{@{}c@{}}Melhor solução\\ alfa = 0.75\end{tabular} & \begin{tabular}[c]{@{}c@{}}Melhor solução\\ alfa = 1.00\end{tabular} & \begin{tabular}[c]{@{}c@{}}Melhor solução\\ alfa aleatório\end{tabular} \\ \hline
% \(CP_p\) & 0 & 0 & 0 & 0 & 0 & 0 & 0 & 0 & 0 \\
% \(CT_t\) & 0 & 0 & 0 & 0 & 0 & 0 & 0 & 0 & 0 \\
% \(CS_s\) & 0 & 0 & 0 & 0 & 0 & 0 & 0 & 0 & 0 \\
% \(OFT\) & 0 & 0 & 0 & 0 & 0 & 0 & 0 & 0 & 0 \\
% \(VS_s\) & 0 & 0 & 0 & 0 & 0 & 0 & 0 & 0 & 0 \\
% \(TSI\) & 0 & 0 & 0 & 0 & 0 & 0 & 0 & 0 & 0 \\
% \(DH3\) & 0 & 0 & 0 & 0 & 0 & 0 & 0 & 0 & 0 \\
% \(IT_p\) & 3 & 5 & 4 & 3 & 3 & 4 & 5 & 3 & 4 \\
% \(JH_t\) & 8 & 0 & 0 & 0 & 0 & 0 & 0 & 0 & 0 \\
% \(PP_t\) & 16 & 17 & 11 & 12 & 10 & 15 & 12 & 11 & 13 \\
% \(AS_d\) & 3 & 0 & 0 & 0 & 0 & 0 & 0 & 0 & 0 \\
% \(ND_p\) & 1 & 0 & 0 & 0 & 0 & 0 & 0 & 0 & 0 \\
% \(ASD\) & 14 & 10 & 8 & 7 & 8 & 9 & 6 & 9 & 9 \\
% \(ADU\) & 18 & 6 & 5 & 6 & 5 & 3 & 4 & 5 & 4 \\
% \(DHP\) & 0 & 0 & 0 & 0 & 0 & 0 & 0 & 0 & 0 \\
% \(AHA\) & 0 & 0 & 0 & 0 & 0 & 0 & 0 & 0 & 0 \\
% \(AHFP\) & 0 & 0 & 0 & 0 & 0 & 0 & 0 & 0 & 0 \\ \hline
% Função Objetivo & 2374 & 278 & 214 & 208 & 200 & 220 & 198 & 214 & 222 \\
% \(Tempo\ (segundos)\) & - & 82,28 & 15,12 & 88,98 & 182,18 & 71,92 & 341,34 & 57,12 & 454,48 \\
% \(Alfa\) & - & 0,15 & 0,00 & 0,15 & 0,25 & 0,50 & 0,75 & 1,00 & 0.440718 \\
% \hline \hline
% \end{tabular}
% \end{adjustbox}
% \caption{Comparação das melhores soluções para 2016/1 com fatores de aleatoriedade diferentes}
% \label{tbl_melhores_2016}
% \end{table}
% ==============================================================================
% TCC2 - TULIO MULLER
% Capítulo 5 - Conclusões
% ==============================================================================
\chapter{Conclusões}
\label{sec-conclusoes}

Este trabalho teve como objetivo o estudo do impacto da aleatoriedade e de movimentos na meta-heurística GRASP com \textit{Simulated Annealing} para a resolução do Problema de Tabela-Horários de Universidades, considerando o Departamento de Computação do Centro de Ciências Exatas, Naturais e da Saúde da Universidade Federal do Espírito Santo como caso de estudo.

Neste trabalho foi implementado o movimento \textit{Lecture Move}, que é a combinação dos clássicos movimentos \textit{Move} e \textit{Swap}, sendo executado para diferentes fatores de aleatoriedade (\(\alpha\)).

Os resultados obtidos foram comparados com outro trabalho na literatura \cite{vital2015grasp}, bem como as soluções manuais elaboradas pelos coordenadores de curso do DCOMP-CCENS-UFES, apresentando soluções melhores para ambos os casos.

Portanto, os resultados obtidos por este trabalho mostraram-se ser mais eficazes que os existentes na literatura, podendo ser aplicado para a geração de Tabela-Horários em semestres futuros, e assim obter tabelas de horários aproximadamente 78,2\% melhores, ou seja, com menos violações de restrições, do que as soluções atuais que são construídas manualmente pelos coordenadores de curso.

Como trabalhos futuros, sugere-se a implementação de novos movimentos para problemas de tabela-horário, como exemplo, o clássico \textit{Cadeia de Kempe}, além de movimentos específicos para restrições fracas, assim como \citeonline{muller2009itc2007} fez para o ITC-2007.
Além disso, outros trabalhos futuros poderiam realizar a implementação de outras meta-heurísticas como Busca Tabu e \textit{Iterated Local Search}, que também podem usar o \textit{SA} como busca local.

Todo o código fonte deste trabalho, bem como as instâncias e saídas, se encontram disponíveis através do link \href{https://github.com/mmtulio/TCC2}{https://github.com/mmtulio/TCC2}.



%%% Páginas finais do documento: bibliografia e anexos. %%%

% Finaliza a parte no bookmark do PDF para que se inicie o bookmark na raiz e adiciona espaço de parte no sumário.
\phantompart

% Marca o início dos elementos pós-textuais.
\postextual

% Referências bibliográficas
\bibliography{bibliografia}


% Apêndices.
\begin{apendicesenv}

% Imprime uma página indicando o início dos apêndices.
%\partapendices

\begin{apendices}

\chapter{Lista de dados do CCENS-UFES}
\label{ap-dados}
A seguir, são apresentadas as listas de todos os dados para os períodos de 2013/2 e 2016/1 do DCOMP.

\section{Dados 2013/2}
\label{dados_2013}

\begin{table}[H]
\footnotesize
\centering
\begin{tabular}{|c|c|c|c|c|}
\hline
\textbf{Id} & \textbf{Prédio} & \textbf{Tipo} & \textbf{Número} & \textbf{Capacidade} \\ \hline
0 & Prédio Central & Sala & 9 & 40 \\ \hline
1 & Prédio Central & Sala & 4 & 55 \\ \hline
2 & Prédio Novo & Sala & 3 & 90 \\ \hline
3 & Prédio Novo & Sala & 1 & 55 \\ \hline
4 & Prédio Novo & Sala & 12 & 65 \\ \hline
5 & Prédio Novo & Sala & 9 & 90 \\ \hline
6 & Prédio Antigo & Sala & 3 & 70 \\ \hline
7 & Prédio Central & Sala & 0 & 80 \\ \hline
8 & ChiChiu & Laboratório & 3 & 40 \\ \hline
9 & ChiChiu & Laboratório & 2 & 40 \\ \hline
10 & ChiChiu & Laboratório & 1 & 40 \\ \hline
11 & Reuni & Laboratório & 7 & 20 \\ \hline
12 & Reuni & Laboratório & 5 & 20 \\ \hline
13 & Reuni & Laboratório & 6 & 20 \\ \hline
\end{tabular}
\caption{Lista de locais disponíveis}
\label{ap-lista-locais-2013}
\end{table}

\begin{table}[H]
\footnotesize
\centering
\begin{tabular}{|c|c|}
\hline
\textbf{Id} & \textbf{Nome} \\ \hline
0 & Antonio Almeida De Barros Junior \\ \hline
1 & Bruno Vilela Oliveira \\ \hline
2 & Clayton Vieira Fraga Filho \\ \hline
3 & Edmar Hell Kampke \\ \hline
4 & Geraldo Regis Mauri \\ \hline
5 & Helder De Amorim Mendes \\ \hline
6 & Jacson Rodrigues Correia Da Silva \\ \hline
7 & Juliana Pinheiro Campos \\ \hline
8 & Larice Nogueira De Andrade \\ \hline
9 & Paulo Roberto Nunes De Souza \\ \hline
10 & Rodrigo Freitas Silva \\ \hline
11 & Simone Dornelas Costa \\ \hline
12 & Thiago Meireles Paixao \\ \hline
13 & Valeria Alves Da Silva \\ \hline
14 & Alexandre Rosa \\ \hline
15 & Tharso \\ \hline
16 & Bernado \\ \hline
17 & Atila \\ \hline
18 & Ronald \\ \hline
19 & Maristela \\ \hline
20 & Clovis \\ \hline
21 & Paulo Henrique Souza \\ \hline
22 & Eleonesio Strey \\ \hline
23 & Aline De Menezes \\ \hline
\end{tabular}
\caption{Lista de professores}
\label{ap-lista-prof-2013}
\end{table}

{\footnotesize
\begin{longtable}{|c|c|c|c|}
\hline
\textbf{Id} & \textbf{Código} & \textbf{Nome} & \textbf{Nível} \\ \hline
0 & COM06842 & Programação I & 1 \\ \hline
1 & COM06850 & Introducao a Ciencia da Computacao & 0 \\ \hline
2 & COM06851 & Matemática Discreta & 1 \\ \hline
3 & COM10076 & Arquitetura de Computadores & 0 \\ \hline
4 & COM06992 & Estrutura de Dados I & 1 \\ \hline
5 & COM10078 & Estrutura de Dados II & 1 \\ \hline
6 & COM06853 & Lógica Computacional & 1 \\ \hline
7 & COM10080 & Lógica Computacional II & 1 \\ \hline
8 & COM10081 & Metodologia de Pesquisa em Informática & 1 \\ \hline
9 & COM10082 & Programação II & 1 \\ \hline
10 & COM10275 & Engenharia de Requisitos de Software & 1 \\ \hline
11 & COM10392 & Linguagens de Programação & 1 \\ \hline
12 & COM10393 & Métodos de Otimização & 1 \\ \hline
13 & COM10394 & Redes de Computadores & 0 \\ \hline
14 & COM10395 & Teoria da Computação & 0 \\ \hline
15 & COM10733 & Gerência de Projeto de Software & 1 \\ \hline
16 & ENG10791 & Compiladores & 1 \\ \hline
17 & ENG10792 & Inteligência Artificial & 0 \\ \hline
18 & ENG10793 & Trabalho de Conclusão de Curso em Ciencia da Computacao I & 0 \\ \hline
19 & COM11063 & Trabalho de Conclusão de Curso em Ciencia da Computacao II & 0 \\ \hline
20 & COM10396 & Desenvolvimento de Sistemas Para Web & 0 \\ \hline
21 & COM11014 & Gerenciamento de Banco de Dados & 0 \\ \hline
22 & COM05207 & Informática & 0 \\ \hline
23 & COM06039 & Lógica e Técnica de Programação & 0 \\ \hline
24 & COM06847 & Introducao à Informatica & 0 \\ \hline
25 & COM06852 & Introdução aos Sistemas de Informação & 0 \\ \hline
26 & COM06996 & Informática e Sociedade & 0 \\ \hline
27 & COM10014 & Computabilidade e Complexidade & 1 \\ \hline
28 & COM10015 & Engenharia de Software & 1 \\ \hline
29 & COM10016 & Sistema de Apoio à Decisão & 0 \\ \hline
30 & COM10128 & Algoritmos Numéricos & 1 \\ \hline
31 & COM10129 & Banco de Dados & 1 \\ \hline
32 & COM10131 & Otimização Linear & 1 \\ \hline
33 & COM10132 & Sistemas Operacionais & 1 \\ \hline
34 & COM10399 & Processamento Digital de Imagens & 0 \\ \hline
35 & COM10507 & Interface Humano-computador & 0 \\ \hline
36 & COM10508 & Projeto de Sistemas de Software & 1 \\ \hline
37 & COM10607 & Computação Forense & 0 \\ \hline
38 & COM10612 & Tópicos Especiais em Informatica I & 0 \\ \hline
39 & COM10616 & Sistemas Distribuídos & 1 \\ \hline
40 & COM11259 & Sistemas de Software Livre & 0 \\ \hline
41 & COM11260 & Estágio em Informática & 0 \\ \hline
42 & COM11261 & Trabalho de Conclusão de Curso em Sistemas de Informação II & 0 \\ \hline
43 & ENG11006 & Comércio Eletrônico & 0 \\ \hline
44 & ENG11007 & Segurança e Auditoria de Sistemas & 0 \\ \hline
45 & MPA06839 & Cálculo A & 1 \\ \hline
46 & ENG06849 & Inglês Instrumental & 0 \\ \hline
47 & MPA06840 & Vetores e Geometria Analitica & 1 \\ \hline
48 & MPA & Cálculo C & 1 \\ \hline
49 & DQF & Fund Física II & 1 \\ \hline
50 & ENG & Estatística & 0 \\ \hline
51 & CFM10426 & Admnistração e Economia & 0 \\ \hline
52 & ENG06854 & Português Instrumental & 0 \\ \hline
53 & MPA06855 & Algebra Linear & 1 \\ \hline
54 & VET10127 & Libras & 0 \\ \hline
\caption{Lista de disciplinas.\label{ap-lista-discp-2013}}
\end{longtable}
}

\begin{table}[H]
\footnotesize
\centering
\begin{tabular}{|c|c|}
\hline
\textbf{Id} & \textbf{Descrição} \\ \hline
0 & Sala \\ \hline
1 & Laboratório \\ \hline
\end{tabular}
\caption{Lista de tipos de salas}
\label{ap-lista-tipo-sala-2013}
\end{table}

\begin{table}[H]
\footnotesize
\centering
\begin{tabular}{|c|c|c|c|}
\hline
\textbf{Id} & \textbf{Curso} & \textbf{Período} & \textbf{Turno Preferencial} \\ \hline
0 & Ciência da Computação & 1 & 1 \\ \hline
1 & Ciência da Computação & 3 & 1 \\ \hline
2 & Ciência da Computação & 5 & 1 \\ \hline
3 & Ciência da Computação & 7 & 1 \\ \hline
4 & Sistemas de Informação & 1 & 2 \\ \hline
5 & Sistemas de Informação & 3 & 2 \\ \hline
6 & Sistemas de Informação & 5 & 2 \\ \hline
7 & Sistemas de Informação & 7 & 2 \\ \hline
8 & Sistemas de Informação & 9 & 2 \\ \hline
9 & Matemática & 5 & 2 \\ \hline
10 & Agronomia & 1 & 1 \\ \hline
11 & Engenharia Industrial Madeireira & 3 & 1 \\ \hline
12 & Engenharia de Alimentos & 3 & 1 \\ \hline
13 & Geologia & 3 & 1 \\ \hline
14 & Ciências Biológicas & 2 & 1 \\ \hline
15 & Nutrição & 2 & 1 \\ \hline
16 & Engenharia Florestal & 2 & 1 \\ \hline
17 & Matemática & 1 & 2 \\ \hline
18 & Engenharia Quimica & 1 & 1 \\ \hline
19 & Sistemas de Informação & 1 & 2 \\ \hline
20 & Matemática & 1 & 2 \\ \hline
21 & Ciência da Computação & 1 & 1 \\ \hline
\end{tabular}
\caption{Lista de turmas}
\label{ap-lista-turmas-2013}
\end{table}

\begin{table}[H]
\footnotesize
\centering
\begin{tabular}{|c|c|c|c|}
\hline
\textbf{Id} & \textbf{Horário Inicial} & \textbf{Horário Final} & \textbf{Duração} \\ \hline
0 & 07h00 & 08h00 & 60 \\ \hline
1 & 08h00 & 09h00 & 60 \\ \hline
2 & 09h00 & 10h00 & 60 \\ \hline
3 & 10h00 & 11h00 & 60 \\ \hline
4 & 11h00 & 12h00 & 60 \\ \hline
5 & 13h00 & 14h00 & 60 \\ \hline
6 & 14h00 & 15h00 & 60 \\ \hline
7 & 15h00 & 16h00 & 60 \\ \hline
8 & 16h00 & 17h00 & 60 \\ \hline
9 & 18h00 & 19h00 & 60 \\ \hline
10 & 19h00 & 20h00 & 60 \\ \hline
11 & 20h00 & 21h00 & 60 \\ \hline
12 & 21h00 & 22h00 & 60 \\ \hline
13 & 22h00 & 23h00 & 60 \\ \hline
\end{tabular}
\caption{Lista de horários}
\label{ap-lista-horarios-2013}
\end{table}

{\footnotesize
\begin{longtable}{|c|c|c|c|c|c|c|c|c|}
\hline
\textbf{Id} & \textbf{Disciplina} & \textbf{\begin{tabular}[c]{@{}c@{}}Nº de\\ Turmas\end{tabular}} & \textbf{Turmas{[}{]}} & \textbf{Vagas} & \textbf{Turno} & \textbf{Professor} & \textbf{\begin{tabular}[c]{@{}c@{}}Tipo de\\ Sala\end{tabular}} & \textbf{Ch} \\ \hline
0 & 0 & 2 & 0 18 & 37 & 1 & 12 & 1 & 2 \\ \hline
1 & 0 & 2 & 0 18 & 37 & 1 & 12 & 1 & 2 \\ \hline
2 & 1 & 2 & 0 21 & 40 & 1 & 1 & 0 & 2 \\ \hline
3 & 1 & 2 & 0 21 & 40 & 1 & 1 & 0 & 2 \\ \hline
4 & 2 & 2 & 0 21 & 40 & 1 & 3 & 0 & 2 \\ \hline
5 & 2 & 2 & 0 21 & 40 & 1 & 3 & 0 & 2 \\ \hline
6 & 3 & 1 & 1 & 30 & 1 & 13 & 0 & 2 \\ \hline
7 & 3 & 1 & 1 & 30 & 1 & 13 & 0 & 2 \\ \hline
8 & 5 & 1 & 1 & 30 & 1 & 5 & 0 & 2 \\ \hline
9 & 5 & 1 & 1 & 30 & 1 & 5 & 1 & 2 \\ \hline
10 & 7 & 1 & 1 & 40 & 1 & 7 & 0 & 2 \\ \hline
11 & 7 & 1 & 1 & 40 & 1 & 7 & 0 & 2 \\ \hline
12 & 8 & 1 & 1 & 30 & 1 & 8 & 1 & 2 \\ \hline
13 & 9 & 1 & 1 & 30 & 1 & 2 & 1 & 2 \\ \hline
14 & 9 & 1 & 1 & 30 & 1 & 2 & 1 & 2 \\ \hline
15 & 10 & 1 & 2 & 30 & 1 & 2 & 1 & 2 \\ \hline
16 & 10 & 1 & 2 & 30 & 1 & 2 & 0 & 2 \\ \hline
17 & 11 & 1 & 2 & 20 & 1 & 10 & 0 & 2 \\ \hline
18 & 11 & 1 & 2 & 20 & 1 & 10 & 0 & 2 \\ \hline
19 & 12 & 1 & 2 & 20 & 1 & 4 & 1 & 2 \\ \hline
20 & 12 & 1 & 2 & 20 & 1 & 4 & 1 & 2 \\ \hline
21 & 13 & 1 & 2 & 20 & 1 & 10 & 1 & 2 \\ \hline
22 & 13 & 1 & 2 & 20 & 1 & 10 & 1 & 2 \\ \hline
23 & 14 & 1 & 2 & 30 & 1 & 7 & 0 & 2 \\ \hline
24 & 14 & 1 & 2 & 30 & 1 & 7 & 0 & 2 \\ \hline
25 & 15 & 1 & 3 & 20 & 1 & 2 & 1 & 2 \\ \hline
26 & 15 & 1 & 3 & 20 & 1 & 2 & 1 & 2 \\ \hline
27 & 16 & 1 & 3 & 20 & 1 & 10 & 0 & 2 \\ \hline
28 & 16 & 1 & 3 & 20 & 1 & 10 & 0 & 2 \\ \hline
29 & 17 & 1 & 3 & 20 & 1 & 6 & 1 & 2 \\ \hline
30 & 17 & 1 & 3 & 20 & 1 & 6 & 1 & 2 \\ \hline
31 & 20 & 2 & 2 3 & 20 & 1 & 1 & 1 & 2 \\ \hline
32 & 20 & 2 & 2 3 & 20 & 1 & 1 & 1 & 2 \\ \hline
33 & 21 & 1 & 3 & 20 & 1 & 0 & 1 & 2 \\ \hline
34 & 21 & 1 & 3 & 20 & 1 & 0 & 1 & 2 \\ \hline
35 & 24 & 1 & 4 & 30 & 2 & 13 & 1 & 2 \\ \hline
36 & 24 & 1 & 4 & 30 & 2 & 13 & 1 & 2 \\ \hline
37 & 25 & 2 & 4 19 & 70 & 2 & 3 & 0 & 2 \\ \hline
38 & 6 & 2 & 4 19 & 70 & 2 & 6 & 0 & 2 \\ \hline
39 & 6 & 2 & 4 19 & 70 & 2 & 6 & 0 & 2 \\ \hline
40 & 0 & 1 & 4 & 30 & 2 & 3 & 1 & 2 \\ \hline
41 & 0 & 1 & 4 & 30 & 2 & 3 & 1 & 2 \\ \hline
42 & 4 & 1 & 5 & 30 & 2 & 7 & 1 & 2 \\ \hline
43 & 4 & 1 & 5 & 30 & 2 & 7 & 1 & 2 \\ \hline
44 & 27 & 1 & 5 & 40 & 2 & 7 & 0 & 2 \\ \hline
45 & 27 & 1 & 5 & 40 & 2 & 7 & 0 & 2 \\ \hline
46 & 28 & 1 & 5 & 30 & 2 & 1 & 0 & 2 \\ \hline
47 & 28 & 1 & 5 & 30 & 2 & 1 & 0 & 2 \\ \hline
48 & 29 & 1 & 5 & 40 & 2 & 11 & 0 & 2 \\ \hline
49 & 29 & 1 & 5 & 40 & 2 & 11 & 0 & 2 \\ \hline
50 & 31 & 1 & 6 & 30 & 2 & 0 & 1 & 2 \\ \hline
51 & 31 & 1 & 6 & 30 & 2 & 0 & 1 & 2 \\ \hline
52 & 32 & 1 & 6 & 20 & 2 & 4 & 0 & 2 \\ \hline
53 & 32 & 1 & 6 & 20 & 2 & 4 & 1 & 2 \\ \hline
54 & 33 & 1 & 6 & 20 & 2 & 6 & 1 & 2 \\ \hline
55 & 33 & 1 & 6 & 20 & 2 & 6 & 0 & 2 \\ \hline
56 & 36 & 1 & 6 & 30 & 2 & 2 & 0 & 2 \\ \hline
57 & 36 & 1 & 6 & 30 & 2 & 2 & 1 & 2 \\ \hline
58 & 35 & 1 & 6 & 20 & 2 & 11 & 0 & 2 \\ \hline
59 & 35 & 1 & 6 & 20 & 2 & 11 & 1 & 2 \\ \hline
60 & 26 & 1 & 7 & 30 & 2 & 8 & 0 & 2 \\ \hline
61 & 39 & 1 & 7 & 20 & 2 & 5 & 0 & 2 \\ \hline
62 & 39 & 1 & 7 & 20 & 2 & 5 & 0 & 2 \\ \hline
63 & 43 & 1 & 7 & 20 & 2 & 11 & 0 & 2 \\ \hline
64 & 43 & 1 & 7 & 20 & 2 & 11 & 1 & 2 \\ \hline
65 & 44 & 1 & 7 & 20 & 2 & 11 & 0 & 2 \\ \hline
66 & 44 & 1 & 7 & 20 & 2 & 11 & 1 & 2 \\ \hline
67 & 34 & 1 & 7 & 20 & 2 & 12 & 1 & 1 \\ \hline
68 & 34 & 1 & 7 & 20 & 2 & 12 & 1 & 3 \\ \hline
69 & 40 & 1 & 8 & 20 & 2 & 6 & 0 & 2 \\ \hline
70 & 37 & 1 & 8 & 20 & 2 & 6 & 1 & 2 \\ \hline
71 & 37 & 1 & 8 & 20 & 2 & 6 & 1 & 2 \\ \hline
72 & 38 & 1 & 8 & 20 & 2 & 13 & 0 & 2 \\ \hline
73 & 38 & 1 & 8 & 20 & 2 & 13 & 0 & 2 \\ \hline
74 & 41 & 1 & 8 & 10 & 2 & 8 & 0 & 2 \\ \hline
75 & 42 & 1 & 8 & 10 & 2 & 3 & 0 & 2 \\ \hline
76 & 22 & 3 & 14 15 16 & 30 & 1 & 8 & 1 & 2 \\ \hline
77 & 22 & 1 & 10 & 30 & 1 & 8 & 1 & 2 \\ \hline
78 & 30 & 1 & 9 & 15 & 2 & 12 & 0 & 2 \\ \hline
79 & 30 & 1 & 9 & 15 & 2 & 12 & 0 & 2 \\ \hline
80 & 23 & 3 & 11 12 13 & 37 & 1 & 1 & 1 & 2 \\ \hline
81 & 23 & 3 & 11 12 13 & 37 & 1 & 1 & 1 & 2 \\ \hline
82 & 0 & 1 & 21 & 30 & 1 & 3 & 1 & 2 \\ \hline
83 & 0 & 1 & 21 & 30 & 1 & 3 & 1 & 2 \\ \hline
84 & 0 & 1 & 19 & 30 & 2 & 12 & 1 & 2 \\ \hline
85 & 0 & 1 & 19 & 30 & 2 & 12 & 1 & 2 \\ \hline
86 & 24 & 1 & 19 & 30 & 2 & 8 & 1 & 2 \\ \hline
87 & 24 & 1 & 19 & 30 & 2 & 8 & 1 & 2 \\ \hline
88 & 24 & 1 & 17 & 35 & 2 & 13 & 1 & 2 \\ \hline
89 & 24 & 1 & 17 & 35 & 2 & 13 & 1 & 2 \\ \hline
90 & 24 & 1 & 20 & 35 & 2 & 1 & 1 & 2 \\ \hline
91 & 24 & 1 & 20 & 35 & 2 & 1 & 1 & 2 \\ \hline
92 & 18 & 1 & 3 & 10 & 1 & 3 & 0 & 2 \\ \hline
93 & 45 & 2 & 0 21 & 30 & 1 & 15 & 0 & 2 \\ \hline
94 & 45 & 2 & 0 21 & 30 & 1 & 15 & 0 & 2 \\ \hline
95 & 45 & 2 & 0 21 & 30 & 1 & 15 & 0 & 2 \\ \hline
96 & 46 & 2 & 0 21 & 30 & 1 & 14 & 0 & 2 \\ \hline
97 & 47 & 2 & 0 21 & 30 & 1 & 16 & 0 & 2 \\ \hline
98 & 47 & 2 & 0 21 & 30 & 1 & 16 & 0 & 2 \\ \hline
99 & 48 & 1 & 1 & 30 & 1 & 17 & 0 & 2 \\ \hline
100 & 48 & 1 & 1 & 30 & 1 & 17 & 0 & 2 \\ \hline
101 & 49 & 1 & 1 & 30 & 1 & 18 & 0 & 2 \\ \hline
102 & 49 & 1 & 1 & 30 & 1 & 18 & 0 & 2 \\ \hline
103 & 50 & 1 & 2 & 30 & 1 & 19 & 0 & 2 \\ \hline
104 & 50 & 1 & 2 & 30 & 1 & 19 & 0 & 2 \\ \hline
105 & 51 & 1 & 3 & 30 & 1 & 20 & 0 & 4 \\ \hline
106 & 52 & 2 & 4 19 & 70 & 2 & 14 & 0 & 2 \\ \hline
107 & 47 & 2 & 4 19 & 70 & 2 & 21 & 0 & 2 \\ \hline
108 & 47 & 2 & 4 19 & 70 & 2 & 21 & 0 & 2 \\ \hline
109 & 53 & 1 & 5 & 45 & 2 & 22 & 0 & 2 \\ \hline
110 & 53 & 1 & 5 & 45 & 2 & 22 & 0 & 2 \\ \hline
111 & 51 & 1 & 7 & 40 & 2 & 20 & 0 & 4 \\ \hline
112 & 54 & 1 & 8 & 20 & 2 & 23 & 0 & 2 \\ \hline
113 & 54 & 1 & 8 & 20 & 2 & 23 & 0 & 2 \\ \hline
\caption{Lista de ofertas.\label{ap-lista-ofertas-2013}}
\end{longtable}
}
\pagebreak

\section{Dados 2016/1}
\label{dados_2016}

\begin{table}[H]
\footnotesize
\centering
\begin{tabular}{|c|c|c|c|c|}
\hline
\textbf{Id} & \textbf{Prédio} & \textbf{Tipo} & \textbf{Número} & \textbf{Capacidade} \\ \hline
0 & Prédio Central & 0 & 9 & 40 \\ \hline
1 & Prédio Central & 0 & 4 & 55 \\ \hline
2 & Prédio Novo & 0 & 1 & 55 \\ \hline
3 & Prédio Novo & 0 & 12 & 65 \\ \hline
4 & Maracanã & 0 & 0 & 80 \\ \hline
5 & ChiChiu & 1 & 1 & 40 \\ \hline
6 & ChiChiu & 1 & 2 & 40 \\ \hline
7 & ChiChiu & 1 & 3 & 40 \\ \hline
8 & Reuni & 1 & 6 & 20 \\ \hline
9 & Reuni & 1 & 7 & 20 \\ \hline
\end{tabular}
\caption{Lista de locais disponíveis}
\label{ap-lista-locais}
\end{table}

\begin{table}[H]
\footnotesize
\centering
\begin{tabular}{|c|c|}
\hline
\textbf{Id} & \textbf{Nome} \\ \hline
0 & Leandro \\ \hline
1 & Bruno Vilela \\ \hline
2 & André \\ \hline
3 & Dayan \\ \hline
4 & Edmar Hell \\ \hline
5 & Geraldo Mauri \\ \hline
6 & Helder Mendes \\ \hline
7 & Jacson Rodrigues \\ \hline
8 & Juliana Campos \\ \hline
9 & Larice Nogueira \\ \hline
10 & Marcelo \\ \hline
11 & Rômulo Louzada \\ \hline
12 & Rodrigo Freitas \\ \hline
13 & Simone Dornelas \\ \hline
14 & Valeria Alves \\ \hline
\end{tabular}
\caption{Lista de professores}
\label{ap-lista-prof}
\end{table}

{\footnotesize
\begin{longtable}{|c|c|c|c|}
\hline
\textbf{Id} & \textbf{Código} & \textbf{Nome} & \textbf{Nível} \\ \hline
0 & COM06842 & Programação I & 1 \\ \hline
1 & COM06850 & Introdução a Ciência da Computação & 0 \\ \hline
2 & COM06851 & Matemática Discreta & 1 \\ \hline
3 & COM10076 & Arquitetura de Computadores & 0 \\ \hline
4 & COM06992 & Estrutura de Dados I & 1 \\ \hline
5 & COM10078 & Estrutura de Dados II & 1 \\ \hline
6 & COM06853 & Lógica Computacional & 1 \\ \hline
7 & COM10080 & Lógica Computacional II & 1 \\ \hline
8 & COM10081 & Metodologia de Pesquisa em Informática & 1 \\ \hline
9 & COM10082 & Programação II & 1 \\ \hline
10 & COM10275 & Engenharia de Requisitos de Software & 1 \\ \hline
11 & COM10392 & Linguagens de Programação & 1 \\ \hline
12 & COM10393 & Métodos de Otimização & 1 \\ \hline
13 & COM10394 & Redes de Computadores & 0 \\ \hline
14 & COM10395 & Teoria da Computação & 0 \\ \hline
15 & COM10733 & Gerência de Projeto de Software & 1 \\ \hline
16 & ENG10791 & Compiladores & 1 \\ \hline
17 & ENG10792 & Inteligência Artificial & 0 \\ \hline
18 & ENG10793 & Trabalho de Conclusão de Curso em Ciência da Computação I & 0 \\ \hline
19 & COM11063 & Trabalho de Conclusão de Curso em Ciência da Computação II & 0 \\ \hline
20 & COM10396 & Desenvolvimento de Sistemas para WEB & 0 \\ \hline
21 & COM11014 & Gerenciamento de Banco de Dados & 0 \\ \hline
22 & COM05207 & Informática & 0 \\ \hline
23 & COM06039 & Lógica e Técnica de Programação & 0 \\ \hline
24 & COM06847 & Introdução a informática & 0 \\ \hline
25 & COM06852 & Introdução aos Sistemas de Informação & 0 \\ \hline
26 & COM06996 & Informática e Sociedade & 0 \\ \hline
27 & COM10014 & Computabilidade e Complexidade & 0 \\ \hline
28 & COM10015 & Engenharia de Software & 1 \\ \hline
29 & COM10016 & Sistema de Apoio a Decisão & 0 \\ \hline
30 & COM10128 & Algoritmos Numéricos & 1 \\ \hline
31 & COM10129 & Banco de Dados & 1 \\ \hline
32 & COM10131 & Otimização Linear & 1 \\ \hline
33 & COM10132 & Sistemas Operacionais & 1 \\ \hline
34 & COM10399 & Processamento Digital de Imagens & 0 \\ \hline
35 & COM10507 & Interface Humano-Computador & 0 \\ \hline
36 & COM10508 & Projeto de Sistemas de Software & 1 \\ \hline
37 & COM10607 & Computação Forense & 0 \\ \hline
38 & COM10612 & Tópicos Especiais em Informática I & 0 \\ \hline
39 & COM10616 & Sistemas Distribuídos & 1 \\ \hline
40 & COM11259 & Sistemas de Software Livre & 0 \\ \hline
41 & COM11260 & Estágio em Informática & 0 \\ \hline
42 & COM11261 & Trabalho de Conclusão de Curso em Sistemas de Informação II & 0 \\ \hline
43 & ENG11006 & Comércio Eletrônico & 0 \\ \hline
44 & ENG11007 & Segurança e Auditoria de Sistemas & 0 \\ \hline
55 & COM06999 & Circuitos Digitais & 0 \\ \hline
57 & COM10130 & Linguagens Formais e Autômatos & 1 \\ \hline
58 & COM10133 & Teoria dos Grafos & 1 \\ \hline
59 & COM10602 & Análise e Projeto de Algoritmos & 1 \\ \hline
60 & COM10604 & Computação Gráfica & 0 \\ \hline
61 & COM10603 & Direito e Legislação & 0 \\ \hline
62 & COM11013 & Tópicos Especiais em Programação & 0 \\ \hline
64 & COM11608 & Tópicos Especiais em Inteligência Artificial & 0 \\ \hline
65 & COM11013 & Tópicos Especiais em Otimização & 0 \\ \hline
66 & COM06985 & Teoria Geral de Sistemas & 0 \\ \hline
67 & COM06984 & Fundamentos de Programação Web & 0 \\ \hline
68 & COM10608 & Gerenciamento e Administração de Redes & 0 \\ \hline
69 & COM11211 & Gestão da Qualidade de Software & 1 \\ \hline
70 & ENG10792 & Segurança em Redes & 0 \\ \hline
71 & COM11212 & Trabalho de Conclusão de Curso I (SI) & 0 \\ \hline
\caption{Lista de disciplinas.\label{ap-lista-discp}}
\end{longtable}
}


\begin{table}[H]
\footnotesize
\centering
\begin{tabular}{|c|c|}
\hline
\textbf{Id} & \textbf{Descrição} \\ \hline
0 & Sala \\ \hline
1 & Lab \\ \hline
\end{tabular}
\caption{Lista de tipos de salas}
\label{ap-lista-tipo-sala}
\end{table}

\begin{table}[H]
\footnotesize
\centering
\begin{tabular}{|c|c|c|c|}
\hline
\textbf{Id} & \textbf{Curso} & \textbf{Período} & \textbf{Turno Preferencial} \\ \hline
0 & Ciência da Computação & 2 & 1 \\ \hline
1 & Ciência da Computação & 3 & 1 \\ \hline
2 & Ciência da Computação & 6 & 1 \\ \hline
3 & Ciência da Computação & 8 & 1 \\ \hline
4 & Sistemas de Informação & 2 & 2 \\ \hline
5 & Sistemas de Informação & 4 & 2 \\ \hline
6 & Sistemas de Informação & 6 & 2 \\ \hline
7 & Sistemas de Informação & 8 & 2 \\ \hline
8 & Matemática & 1 & 2 \\ \hline
9 & Matemática & 5 & 2 \\ \hline
10 & Agronomia & 1 & 1 \\ \hline
11 & Engenharia de Alimentos & 3 & 1 \\ \hline
12 & Geologia & 7 & 1 \\ \hline
13 & Engenharia Florestal & 1 & 1 \\ \hline
14 & Sistemas de Informação & 1 & 2 \\ \hline
\end{tabular}
\caption{Lista de turmas}
\label{ap-lista-turmas}
\end{table}

\begin{table}[H]
\footnotesize
\centering
\begin{tabular}{|c|c|c|c|}
\hline
\textbf{Id} & \textbf{Horário Inicial} & \textbf{Horário Final} & \textbf{Duração} \\ \hline
0 & 07h00 & 08h00 & 60 \\ \hline
1 & 08h00 & 09h00 & 60 \\ \hline
2 & 09h00 & 10h00 & 60 \\ \hline
3 & 10h00 & 11h00 & 60 \\ \hline
4 & 11h00 & 12h00 & 60 \\ \hline
5 & 13h00 & 14h00 & 60 \\ \hline
6 & 14h00 & 15h00 & 60 \\ \hline
7 & 15h00 & 16h00 & 60 \\ \hline
8 & 16h00 & 17h00 & 60 \\ \hline
9 & 18h00 & 19h00 & 60 \\ \hline
10 & 19h00 & 20h00 & 60 \\ \hline
11 & 20h00 & 21h00 & 60 \\ \hline
12 & 21h00 & 22h00 & 60 \\ \hline
13 & 22h00 & 23h00 & 60 \\ \hline
\end{tabular}
\caption{Lista de horários}
\label{ap-lista-horarios}
\end{table}

{\footnotesize
\begin{longtable}{|c|c|c|c|c|c|c|c|c|}
\hline
\textbf{Id} & \textbf{Disciplina} & \textbf{\begin{tabular}[c]{@{}c@{}}Nº de\\ Turmas\end{tabular}} & \textbf{Turmas{[}{]}} & \textbf{Vagas} & \textbf{Turno} & \textbf{Professor} & \textbf{\begin{tabular}[c]{@{}c@{}}Tipo de\\ Sala\end{tabular}} & \textbf{Ch} \\ \hline
0 & 55 & 1 & 0 & 30 & 1 & 15 & 0 & 2 \\ \hline
1 & 55 & 1 & 0 & 30 & 1 & 12 & 0 & 2 \\ \hline
2 & 4 & 1 & 0 & 40 & 1 & 11 & 1 & 2 \\ \hline
3 & 4 & 1 & 0 & 40 & 1 & 11 & 1 & 2 \\ \hline
4 & 26 & 1 & 0 & 30 & 1 & 10 & 0 & 2 \\ \hline
5 & 6 & 1 & 0 & 30 & 1 & 9 & 0 & 2 \\ \hline
6 & 6 & 1 & 0 & 30 & 1 & 9 & 0 & 2 \\ \hline
7 & 30 & 1 & 1 & 30 & 1 & 23 & 0 & 2 \\ \hline
8 & 30 & 1 & 1 & 30 & 1 & 23 & 0 & 2 \\ \hline
9 & 31 & 1 & 1 & 30 & 1 & 0 & 0 & 2 \\ \hline
10 & 31 & 1 & 1 & 30 & 1 & 0 & 0 & 2 \\ \hline
11 & 28 & 1 & 1 & 30 & 1 & 2 & 0 & 2 \\ \hline
12 & 28 & 1 & 1 & 30 & 1 & 2 & 0 & 2 \\ \hline
13 & 57 & 1 & 1 & 30 & 1 & 9 & 1 & 2 \\ \hline
14 & 57 & 1 & 1 & 30 & 1 & 9 & 1 & 2 \\ \hline
15 & 32 & 1 & 1 & 30 & 1 & 6 & 0 & 2 \\ \hline
16 & 32 & 1 & 1 & 30 & 1 & 6 & 0 & 2 \\ \hline
17 & 33 & 1 & 1 & 20 & 1 & 5 & 0 & 2 \\ \hline
18 & 33 & 1 & 1 & 20 & 1 & 5 & 0 & 2 \\ \hline
19 & 58 & 1 & 1 & 30 & 1 & 4 & 0 & 2 \\ \hline
20 & 58 & 1 & 1 & 30 & 1 & 4 & 0 & 2 \\ \hline
21 & 59 & 1 & 2 & 30 & 1 & 13 & 0 & 2 \\ \hline
22 & 59 & 1 & 2 & 30 & 1 & 13 & 0 & 2 \\ \hline
23 & 60 & 1 & 2 & 30 & 1 & 5 & 1 & 2 \\ \hline
24 & 60 & 1 & 2 & 30 & 1 & 5 & 1 & 2 \\ \hline
25 & 61 & 1 & 2 & 30 & 1 & 10 & 0 & 2 \\ \hline
26 & 35 & 1 & 2 & 20 & 1 & 14 & 0 & 2 \\ \hline
27 & 35 & 1 & 2 & 20 & 1 & 14 & 1 & 2 \\ \hline
28 & 36 & 1 & 2 & 30 & 1 & 2 & 1 & 2 \\ \hline
29 & 36 & 1 & 2 & 30 & 1 & 2 & 1 & 2 \\ \hline
30 & 39 & 1 & 2 & 20 & 1 & 5 & 1 & 2 \\ \hline
31 & 39 & 1 & 2 & 20 & 1 & 5 & 1 & 2 \\ \hline
32 & 62 & 1 & 2 & 30 & 1 & 8 & 1 & 2 \\ \hline
33 & 62 & 1 & 2 & 30 & 1 & 8 & 1 & 2 \\ \hline
34 & 64 & 1 & 3 & 30 & 1 & 8 & 1 & 2 \\ \hline
35 & 64 & 1 & 3 & 30 & 1 & 8 & 1 & 2 \\ \hline
36 & 65 & 1 & 3 & 30 & 1 & 3 & 1 & 2 \\ \hline
37 & 65 & 1 & 3 & 30 & 1 & 3 & 1 & 2 \\ \hline
38 & 67 & 1 & 4 & 40 & 2 & 1 & 0 & 2 \\ \hline
39 & 67 & 1 & 4 & 40 & 2 & 1 & 0 & 2 \\ \hline
40 & 2 & 1 & 4 & 40 & 2 & 4 & 0 & 2 \\ \hline
41 & 2 & 1 & 4 & 40 & 2 & 4 & 0 & 2 \\ \hline
42 & 66 & 1 & 4 & 40 & 2 & 14 & 0 & 2 \\ \hline
43 & 66 & 1 & 4 & 40 & 2 & 14 & 0 & 2 \\ \hline
44 & 3 & 1 & 5 & 50 & 2 & 15 & 0 & 2 \\ \hline
45 & 3 & 1 & 5 & 50 & 2 & 15 & 0 & 2 \\ \hline
46 & 10 & 1 & 5 & 40 & 2 & 2 & 1 & 2 \\ \hline
47 & 10 & 1 & 5 & 40 & 2 & 2 & 1 & 2 \\ \hline
48 & 5 & 1 & 5 & 30 & 2 & 11 & 1 & 2 \\ \hline
49 & 5 & 1 & 5 & 30 & 2 & 11 & 1 & 2 \\ \hline
50 & 9 & 1 & 5 & 30 & 2 & 1 & 1 & 2 \\ \hline
51 & 9 & 1 & 5 & 30 & 2 & 1 & 1 & 2 \\ \hline
52 & 61 & 1 & 6 & 40 & 2 & 10 & 0 & 2 \\ \hline
53 & 15 & 1 & 6 & 20 & 2 & 2 & 1 & 2 \\ \hline
54 & 15 & 1 & 6 & 20 & 2 & 2 & 1 & 2 \\ \hline
55 & 21 & 1 & 6 & 20 & 2 & 0 & 1 & 2 \\ \hline
56 & 21 & 1 & 6 & 20 & 2 & 0 & 1 & 2 \\ \hline
57 & 8 & 1 & 6 & 40 & 2 & 10 & 0 & 2 \\ \hline
58 & 12 & 1 & 6 & 20 & 2 & 6 & 1 & 2 \\ \hline
59 & 12 & 1 & 6 & 20 & 2 & 6 & 1 & 2 \\ \hline
60 & 13 & 1 & 6 & 20 & 2 & 13 & 1 & 2 \\ \hline
61 & 13 & 1 & 6 & 20 & 2 & 13 & 1 & 2 \\ \hline
62 & 20 & 1 & 7 & 20 & 2 & 8 & 1 & 2 \\ \hline
63 & 20 & 1 & 7 & 20 & 2 & 8 & 1 & 2 \\ \hline
64 & 68 & 1 & 7 & 20 & 2 & 7 & 1 & 2 \\ \hline
65 & 68 & 1 & 7 & 20 & 2 & 7 & 1 & 2 \\ \hline
66 & 69 & 1 & 7 & 20 & 2 & 2 & 1 & 2 \\ \hline
67 & 69 & 1 & 7 & 20 & 2 & 2 & 1 & 2 \\ \hline
68 & 70 & 1 & 7 & 20 & 2 & 7 & 1 & 2 \\ \hline
69 & 70 & 1 & 7 & 20 & 2 & 7 & 1 & 2 \\ \hline
70 & 22 & 1 & 9 & 40 & 1 & 12 & 1 & 3 \\ \hline
71 & 22 & 1 & 12 & 40 & 1 & 12 & 1 & 3 \\ \hline
72 & 0 & 1 & 8 & 28 & 2 & 12 & 1 & 2 \\ \hline
73 & 0 & 1 & 8 & 28 & 2 & 12 & 1 & 2 \\ \hline
74 & 0 & 1 & 13 & 30 & 2 & 12 & 1 & 2 \\ \hline
75 & 0 & 1 & 13 & 30 & 2 & 12 & 1 & 2 \\ \hline
76 & 23 & 1 & 10 & 40 & 1 & 0 & 1 & 2 \\ \hline
77 & 23 & 1 & 10 & 40 & 1 & 0 & 1 & 2 \\ \hline
78 & 23 & 1 & 11 & 40 & 1 & 0 & 1 & 2 \\ \hline
79 & 23 & 1 & 11 & 40 & 1 & 0 & 1 & 2 \\ \hline
80 & 56 & 1 & 0 & 30 & 1 & 19 & 0 & 2 \\ \hline
81 & 56 & 1 & 0 & 30 & 1 & 19 & 0 & 2 \\ \hline
82 & 53 & 1 & 0 & 21 & 1 & 18 & 0 & 2 \\ \hline
83 & 53 & 1 & 0 & 21 & 1 & 18 & 0 & 2 \\ \hline
84 & 45 & 1 & 4 & 40 & 2 & 21 & 0 & 2 \\ \hline
85 & 45 & 1 & 4 & 40 & 2 & 21 & 0 & 2 \\ \hline
86 & 45 & 1 & 4 & 40 & 2 & 21 & 0 & 2 \\ \hline
87 & 50 & 1 & 5 & 40 & 2 & 20 & 0 & 2 \\ \hline
88 & 50 & 1 & 5 & 40 & 2 & 20 & 0 & 2 \\ \hline
89 & 63 & 1 & 7 & 40 & 2 & 17 & 0 & 2 \\ \hline
90 & 63 & 1 & 3 & 30 & 1 & 17 & 0 & 2 \\ \hline
91 & 46 & 1 & 4 & 40 & 2 & 22 & 0 & 2 \\ \hline
92 & 52 & 1 & 0 & 30 & 1 & 16 & 0 & 2 \\ \hline
\caption{Lista de ofertas.\label{ap-lista-ofertas}}
\end{longtable}
}

% \chapter{Tabelas de resultados para diferentes fatores de aleatoriedade}
% \label{ap-results}


% \begin{table}[!h]
% \begin{adjustbox}{width=1\textwidth,center=\textwidth}
% \begin{tabular}{c|c|c|c|c|c|c|c|c|c|c|c}
% \hline \hline
% Penalização & \multicolumn{1}{c|}{S01} & \multicolumn{1}{c|}{S02} & \multicolumn{1}{c|}{S03} & \multicolumn{1}{c|}{S04} & \multicolumn{1}{c|}{S05} & \multicolumn{1}{c|}{S06} & \multicolumn{1}{c|}{S07} & \multicolumn{1}{c|}{S08} & \multicolumn{1}{c|}{S09} & S10 & Média \\ \hline
% \(CP_p\) & 0 & 0 & 0 & 0 & 0 & 0 & 0 & 0 & 0 & 0 & 0 \\
% \(CT_t\) & 0 & 0 & 0 & 0 & 0 & 0 & 0 & 0 & 0 & 0 & 0 \\
% \(CS_s\) & 0 & 0 & 0 & 0 & 0 & 0 & 0 & 0 & 0 & 0 & 0 \\
% \(OFT\) & 0 & 0 & 0 & 0 & 0 & 0 & 0 & 0 & 0 & 0 & 0 \\
% \(VS_s\) & 0 & 0 & 0 & 0 & 0 & 0 & 0 & 0 & 0 & 0 & 0 \\
% \(TSI\) & 0 & 0 & 0 & 0 & 0 & 0 & 0 & 0 & 0 & 0 & 0 \\
% \(DH3\) & 0 & 0 & 0 & 0 & 0 & 0 & 0 & 0 & 0 & 0 & 0 \\
% \(IT_p\) & 6 & 8 & 7 & 8 & 8 & 8 & 7 & 7 & 6 & 6 & 7,1 \\
% \(JH_t\) & 0 & 0 & 0 & 0 & 0 & 0 & 0 & 0 & 0 & 0 & 0 \\
% \(PP_t\) & 11 & 14 & 18 & 13 & 16 & 17 & 17 & 21 & 13 & 12 & 15,2 \\
% \(AS_d\) & 0 & 0 & 0 & 0 & 0 & 0 & 0 & 0 & 0 & 0 & 0 \\
% \(ND_p\) & 0 & 0 & 0 & 0 & 0 & 0 & 0 & 0 & 0 & 0 & 0 \\
% \(ASD\) & 9 & 9 & 11 & 12 & 10 & 5 & 10 & 13 & 11 & 11 & 10,1 \\
% \(ADU\) & 9 & 9 & 8 & 10 & 10 & 7 & 9 & 9 & 9 & 8 & 8,8 \\
% \(DHP\) & 0 & 0 & 0 & 0 & 0 & 0 & 0 & 0 & 0 & 0 & 0 \\
% \(AHA\) & 0 & 0 & 0 & 0 & 0 & 0 & 0 & 0 & 0 & 0 & 0 \\
% \(AHFP\) & 0 & 0 & 0 & 0 & 0 & 0 & 0 & 0 & 0 & 0 & 0 \\ \hline
% Função Objetivo & 284 & 316 & 332 & 352 & 344 & 268 & 328 & 374 & 312 & 298 & 320,8 \\
% \(Tempo\ (segundos)\) & 181,78 & 154,66 & 219,13 & 132,59 & 377,29 & 183,86 & 417,77 & 65,69 & 133,82 & 198,23 & 206,482 \\
% \(Alfa\) & 0,00 & 0,00 & 0,00 & 0,00 & 0,00 & 0,00 & 0,00 & 0,00 & 0,00 & 0,00 & 0,00 \\
% \hline \hline
% \end{tabular}
% \end{adjustbox}
% \caption{Resultados obtidos para 2013/2 com \(\alpha = 0\)}
% \label{tbl_result_2013_alfa0}
% \end{table}

% \begin{table}[!h]
% \begin{adjustbox}{width=1\textwidth,center=\textwidth}
% \begin{tabular}{c|c|c|c|c|c|c|c|c|c|c|c}
% \hline \hline
% Penalização & \multicolumn{1}{c|}{S01} & \multicolumn{1}{c|}{S02} & \multicolumn{1}{c|}{S03} & \multicolumn{1}{c|}{S04} & \multicolumn{1}{c|}{S05} & \multicolumn{1}{c|}{S06} & \multicolumn{1}{c|}{S07} & \multicolumn{1}{c|}{S08} & \multicolumn{1}{c|}{S09} & S10 & Média \\ \hline
% \(CP_p\) & 0 & 0 & 0 & 0 & 0 & 0 & 0 & 0 & 0 & 0 & 0 \\ 
% \(CT_t\) & 0 & 0 & 0 & 0 & 0 & 0 & 0 & 0 & 0 & 0 & 0 \\ 
% \(CS_s\) & 0 & 0 & 0 & 0 & 0 & 0 & 0 & 0 & 0 & 0 & 0 \\ 
% \(OFT\) & 0 & 0 & 0 & 0 & 0 & 0 & 0 & 0 & 0 & 0 & 0 \\ 
% \(VS_s\) & 0 & 0 & 0 & 0 & 0 & 0 & 0 & 0 & 0 & 0 & 0 \\ 
% \(TSI\) & 0 & 0 & 0 & 0 & 0 & 0 & 0 & 0 & 0 & 0 & 0 \\ 
% \(DH3\) & 0 & 0 & 0 & 0 & 0 & 0 & 0 & 0 & 0 & 0 & 0 \\ 
% \(IT_p\) & 9 & 6 & 9 & 7 & 6 & 7 & 7 & 7 & 6 & 7 & 7,1 \\ 
% \(JH_t\) & 0 & 0 & 0 & 0 & 0 & 0 & 0 & 0 & 0 & 0 & 0 \\ 
% \(PP_t\) & 25 & 21 & 17 & 9 & 12 & 11 & 11 & 14 & 16 & 15 & 15,1 \\ 
% \(AS_d\) & 0 & 0 & 0 & 0 & 0 & 0 & 0 & 0 & 0 & 0 & 0 \\ 
% \(ND_p\) & 0 & 0 & 0 & 0 & 0 & 0 & 0 & 0 & 0 & 0 & 0 \\ 
% \(ASD\) & 8 & 9 & 9 & 12 & 11 & 9 & 8 & 8 & 9 & 10 & 9,3 \\ 
% \(ADU\) & 8 & 9 & 12 & 10 & 12 & 12 & 12 & 10 & 10 & 12 & 10,7 \\ 
% \(DHP\) & 0 & 0 & 0 & 0 & 0 & 0 & 0 & 0 & 0 & 0 & 0 \\ 
% \(AHA\) & 0 & 0 & 0 & 0 & 0 & 0 & 0 & 0 & 0 & 0 & 0 \\ 
% \(AHFP\) & 0 & 0 & 0 & 0 & 0 & 0 & 0 & 0 & 0 & 0 & 0 \\ \hline 
% Função Objetivo & 350 & 324 & 368 & 326 & 338 & 324 & 314 & 306 & 314 & 350 & 331,4 \\
% \(Tempo\ (segundos)\) & 73,82 & 77,14 & 279,03 &  266.59  & 441,83 & 51,33 & 344,09 & 400,17 & 190,61 & 443,37 & 255,71 \\ 
% \(Alfa\) & 0,15 & 0,15 & 0,15 & 0,15 & 0,15 & 0,15 & 0,15 & 0,15 & 0,15 & 0,15 & 0,15 \\ 
% \hline \hline
% \end{tabular}
% \end{adjustbox}
% \caption{Resultados obtidos para 2013/2 com \(\alpha = 0.15\)}
% \label{tbl_result_2013_alfa015}
% \end{table}

% \begin{table}[!h]
% \begin{adjustbox}{width=1\textwidth,center=\textwidth}
% \begin{tabular}{c|c|c|c|c|c|c|c|c|c|c|c}
% \hline \hline
% Penalização & \multicolumn{1}{c|}{S01} & \multicolumn{1}{c|}{S02} & \multicolumn{1}{c|}{S03} & \multicolumn{1}{c|}{S04} & \multicolumn{1}{c|}{S05} & \multicolumn{1}{c|}{S06} & \multicolumn{1}{c|}{S07} & \multicolumn{1}{c|}{S08} & \multicolumn{1}{c|}{S09} & S10 & Média \\ \hline
% \(CP_p\) & 0 & 0 & 0 & 0 & 0 & 0 & 0 & 0 & 0 & 0 & 0 \\ 
% \(CT_t\) & 0 & 0 & 0 & 0 & 0 & 0 & 0 & 0 & 0 & 0 & 0 \\ 
% \(CS_s\) & 0 & 0 & 0 & 0 & 0 & 0 & 0 & 0 & 0 & 0 & 0 \\ 
% \(OFT\) & 0 & 0 & 0 & 0 & 0 & 0 & 0 & 0 & 0 & 0 & 0 \\ 
% \(VS_s\) & 0 & 0 & 0 & 0 & 0 & 0 & 0 & 0 & 0 & 0 & 0 \\ 
% \(TSI\) & 0 & 0 & 0 & 0 & 0 & 0 & 0 & 0 & 0 & 0 & 0 \\ 
% \(DH3\) & 0 & 0 & 0 & 0 & 0 & 0 & 0 & 0 & 0 & 0 & 0 \\ 
% \(IT_p\) & 8 & 7 & 9 & 8 & 9 & 9 & 8 & 6 & 8 & 7 & 7,9 \\ 
% \(JH_t\) & 0 & 0 & 0 & 0 & 0 & 0 & 0 & 0 & 0 & 0 & 0 \\ 
% \(PP_t\) & 17 & 16 & 18 & 18 & 17 & 20 & 16 & 17 & 20 & 13 & 17,2 \\ 
% \(AS_d\) & 0 & 0 & 0 & 0 & 0 & 0 & 0 & 0 & 0 & 0 & 0 \\ 
% \(ND_p\) & 0 & 0 & 0 & 0 & 0 & 0 & 0 & 0 & 0 & 0 & 0 \\ 
% \(ASD\) & 10 & 9 & 10 & 10 & 11 & 12 & 12 & 11 & 12 & 11 & 10,8 \\ 
% \(ADU\) & 11 & 10 & 9 & 8 & 9 & 8 & 10 & 12 & 9 & 10 & 9,6 \\ 
% \(DHP\) & 0 & 0 & 0 & 0 & 0 & 0 & 0 & 0 & 0 & 0 & 0 \\ 
% \(AHA\) & 0 & 0 & 0 & 0 & 0 & 0 & 0 & 0 & 0 & 0 & 0 \\ 
% \(AHFP\) & 0 & 0 & 0 & 0 & 0 & 0 & 0 & 0 & 0 & 0 & 0 \\ \hline 
% Função Objetivo & 358 & 324 & 352 & 332 & 358 & 370 & 364 & 358 & 370 & 332 & 351,8 \\
% \(Tempo\ (segundos)\) & 135,87 & 304,2 & 109,46 & 356,26 & 295,68 & 389,86 & 131,53 & 240,19 & 243,85 & 405,55 & 261,245 \\ 
% \(Alfa\) & 0,25 & 0,25 & 0,25 & 0,25 & 0,25 & 0,25 & 0,25 & 0,25 & 0,25 & 0,25 & 0,25 \\ 
% \hline \hline
% \end{tabular}
% \end{adjustbox}
% \caption{Resultados obtidos para 2013/2 com \(\alpha = 0.25\)}
% \label{tbl_result_2013_alfa025}
% \end{table}

% \begin{table}[!h]
% \begin{adjustbox}{width=1\textwidth,center=\textwidth}
% \begin{tabular}{c|c|c|c|c|c|c|c|c|c|c|c}
% \hline \hline
% Penalização & \multicolumn{1}{c|}{S01} & \multicolumn{1}{c|}{S02} & \multicolumn{1}{c|}{S03} & \multicolumn{1}{c|}{S04} & \multicolumn{1}{c|}{S05} & \multicolumn{1}{c|}{S06} & \multicolumn{1}{c|}{S07} & \multicolumn{1}{c|}{S08} & \multicolumn{1}{c|}{S09} & S10 & Média \\ \hline
% \(CP_p\) & 0 & 0 & 0 & 0 & 0 & 0 & 0 & 0 & 0 & 0 & 0 \\ 
% \(CT_t\) & 0 & 0 & 0 & 0 & 0 & 0 & 0 & 0 & 0 & 0 & 0 \\ 
% \(CS_s\) & 0 & 0 & 0 & 0 & 0 & 0 & 0 & 0 & 0 & 0 & 0 \\ 
% \(OFT\) & 0 & 0 & 0 & 0 & 0 & 0 & 0 & 0 & 0 & 0 & 0 \\ 
% \(VS_s\) & 0 & 0 & 0 & 0 & 0 & 0 & 0 & 0 & 0 & 0 & 0 \\ 
% \(TSI\) & 0 & 0 & 0 & 0 & 0 & 0 & 0 & 0 & 0 & 0 & 0 \\ 
% \(DH3\) & 0 & 0 & 0 & 0 & 0 & 0 & 0 & 0 & 0 & 0 & 0 \\ 
% \(IT_p\) & 8 & 7 & 5 & 8 & 8 & 6 & 9 & 8 & 7 & 7 & 7,3 \\ 
% \(JH_t\) & 0 & 0 & 0 & 0 & 0 & 0 & 0 & 0 & 0 & 0 & 0 \\ 
% \(PP_t\) & 17 & 12 & 14 & 20 & 19 & 19 & 16 & 16 & 22 & 17 & 17,2 \\ 
% \(AS_d\) & 0 & 0 & 0 & 0 & 0 & 0 & 0 & 0 & 0 & 0 & 0 \\ 
% \(ND_p\) & 0 & 0 & 0 & 0 & 0 & 0 & 0 & 0 & 0 & 0 & 0 \\ 
% \(ASD\) & 7 & 9 & 8 & 10 & 15 & 9 & 11 & 9 & 13 & 10 & 10,1 \\ 
% \(ADU\) & 10 & 11 & 12 & 10 & 10 & 11 & 11 & 9 & 9 & 12 & 10,5 \\ 
% \(DHP\) & 0 & 0 & 0 & 0 & 0 & 0 & 0 & 0 & 0 & 0 & 0 \\ 
% \(AHA\) & 0 & 0 & 0 & 0 & 0 & 0 & 0 & 0 & 0 & 0 & 0 \\ 
% \(AHFP\) & 0 & 0 & 0 & 0 & 0 & 0 & 0 & 0 & 0 & 0 & 0 \\ \hline 
% Função Objetivo & 318 & 318 & 306 & 360 & 406 & 336 & 374 & 324 & 378 & 358 & 347,8 \\
% \(Tempo\ (segundos)\) & 93,3 & 390 & 24,94 & 280,4 & 81,27 & 54,25 & 175,97 & 339,81 & 430,21 & 448,77 & 231,892 \\ 
% \(Alfa\) & 0,50 & 0,50 & 0,50 & 0,50 & 0,50 & 0,50 & 0,50 & 0,50 & 0,50 & 0,50 & 0,50 \\ \hline \hline
% \end{tabular}
% \end{adjustbox}
% \caption{Resultados obtidos para 2013/2 com \(\alpha = 0.50\)}
% \label{tbl_result_2013_alfa050}
% \end{table}

% \begin{table}[!h]
% \begin{adjustbox}{width=1\textwidth,center=\textwidth}
% \begin{tabular}{c|c|c|c|c|c|c|c|c|c|c|c}
% \hline \hline
% Penalização & \multicolumn{1}{c|}{S01} & \multicolumn{1}{c|}{S02} & \multicolumn{1}{c|}{S03} & \multicolumn{1}{c|}{S04} & \multicolumn{1}{c|}{S05} & \multicolumn{1}{c|}{S06} & \multicolumn{1}{c|}{S07} & \multicolumn{1}{c|}{S08} & \multicolumn{1}{c|}{S09} & S10 & Média \\ \hline
% \(CP_p\) & 0 & 0 & 0 & 0 & 0 & 0 & 0 & 0 & 0 & 0 & 0 \\  
% \(CT_t\) & 0 & 0 & 0 & 0 & 0 & 0 & 0 & 0 & 0 & 0 & 0 \\  
% \(CS_s\) & 0 & 0 & 0 & 0 & 0 & 0 & 0 & 0 & 0 & 0 & 0 \\  
% \(OFT\) & 0 & 0 & 0 & 0 & 0 & 0 & 0 & 0 & 0 & 0 & 0 \\  
% \(VS_s\) & 0 & 0 & 0 & 0 & 0 & 0 & 0 & 0 & 0 & 0 & 0 \\  
% \(TSI\) & 0 & 0 & 0 & 0 & 0 & 0 & 0 & 0 & 0 & 0 & 0 \\  
% \(DH3\) & 0 & 0 & 0 & 0 & 0 & 0 & 0 & 0 & 0 & 0 & 0 \\  
% \(IT_p\) & 7 & 8 & 10 & 7 & 7 & 7 & 9 & 7 & 8 & 7 & 7,7 \\  
% \(JH_t\) & 0 & 0 & 0 & 0 & 0 & 0 & 0 & 0 & 0 & 0 & 0 \\  
% \(PP_t\) & 21 & 18 & 10 & 25 & 19 & 19 & 20 & 16 & 18 & 19 & 18,5 \\  
% \(AS_d\) & 0 & 0 & 0 & 0 & 0 & 0 & 0 & 0 & 0 & 0 & 0 \\  
% \(ND_p\) & 0 & 0 & 0 & 0 & 0 & 0 & 0 & 0 & 0 & 0 & 0 \\  
% \(ASD\) & 11 & 13 & 10 & 11 & 9 & 7 & 11 & 9 & 9 & 8 & 9,8 \\  
% \(ADU\) & 8 & 8 & 12 & 9 & 10 & 8 & 9 & 12 & 10 & 12 & 9,8 \\  
% \(DHP\) & 0 & 0 & 0 & 0 & 0 & 0 & 0 & 0 & 0 & 0 & 0 \\  
% \(AHA\) & 0 & 0 & 0 & 0 & 0 & 0 & 0 & 0 & 0 & 0 & 0 \\  
% \(AHFP\) & 0 & 0 & 0 & 0 & 0 & 0 & 0 & 0 & 0 & 0 & 0 \\ \hline 
% Função Objetivo & 344 & 362 & 360 & 370 & 336 & 296 & 370 & 344 & 342 & 346 & 347 \\  
% \(Tempo\ (segundos)\) & 344,75 & 495,98 & 469,61 & 455,12 &  295.56  & 78,82 & 394,32 & 146,64 & 404,99 & 478,99 & 363,2466667 \\  
% \(Alfa\) & 0,75 & 0,75 & 0,75 & 0,75 & 0,75 & 0,75 & 0,75 & 0,75 & 0,75 & 0,75 & 0,75 \\  
% \hline \hline
% \end{tabular}
% \end{adjustbox}
% \caption{Resultados obtidos para 2013/2 com \(\alpha = 0.75\)}
% \label{tbl_result_2013_alfa075}
% \end{table}

% \begin{table}[!h]
% \begin{adjustbox}{width=1\textwidth,center=\textwidth}
% \begin{tabular}{c|c|c|c|c|c|c|c|c|c|c|c}
% \hline \hline
% Penalização & \multicolumn{1}{c|}{S01} & \multicolumn{1}{c|}{S02} & \multicolumn{1}{c|}{S03} & \multicolumn{1}{c|}{S04} & \multicolumn{1}{c|}{S05} & \multicolumn{1}{c|}{S06} & \multicolumn{1}{c|}{S07} & \multicolumn{1}{c|}{S08} & \multicolumn{1}{c|}{S09} & S10 & Média \\ \hline
% \(CP_p\) & 0 & 0 & 0 & 0 & 0 & 0 & 0 & 0 & 0 & 0 & 0 \\ 
% \(CT_t\) & 0 & 0 & 0 & 0 & 0 & 0 & 0 & 0 & 0 & 0 & 0 \\ 
% \(CS_s\) & 0 & 0 & 0 & 0 & 0 & 0 & 0 & 0 & 0 & 0 & 0 \\ 
% \(OFT\) & 0 & 0 & 0 & 0 & 0 & 0 & 0 & 0 & 0 & 0 & 0 \\ 
% \(VS_s\) & 0 & 0 & 0 & 0 & 0 & 0 & 0 & 0 & 0 & 0 & 0 \\ 
% \(TSI\) & 0 & 0 & 0 & 0 & 0 & 0 & 0 & 0 & 0 & 0 & 0 \\ 
% \(DH3\) & 0 & 0 & 0 & 0 & 0 & 0 & 0 & 0 & 0 & 0 & 0 \\ 
% \(IT_p\) & 8 & 6 & 7 & 7 & 9 & 8 & 7 & 7 & 5 & 7 & 7,1 \\ 
% \(JH_t\) & 0 & 0 & 0 & 0 & 0 & 0 & 0 & 0 & 0 & 0 & 0 \\ 
% \(PP_t\) & 20 & 18 & 16 & 20 & 22 & 16 & 21 & 19 & 20 & 13 & 18,5 \\ 
% \(AS_d\) & 0 & 0 & 0 & 0 & 0 & 0 & 0 & 0 & 0 & 0 & 0 \\ 
% \(ND_p\) & 0 & 0 & 0 & 0 & 0 & 0 & 0 & 0 & 0 & 0 & 0 \\ 
% \(ASD\) & 8 & 10 & 13 & 8 & 9 & 11 & 12 & 13 & 16 & 9 & 10,9 \\ 
% \(ADU\) & 9 & 10 & 10 & 9 & 11 & 8 & 9 & 11 & 10 & 9 & 9,6 \\ 
% \(DHP\) & 0 & 0 & 0 & 0 & 0 & 0 & 0 & 0 & 0 & 0 & 0 \\ 
% \(AHA\) & 0 & 0 & 0 & 0 & 0 & 0 & 0 & 0 & 0 & 0 & 0 \\ 
% \(AHFP\) & 0 & 0 & 0 & 0 & 0 & 0 & 0 & 0 & 0 & 0 & 0 \\ \hline 
% Função Objetivo & 330 & 332 & 364 & 320 & 378 & 334 & 364 & 386 & 390 & 302 & 350 \\
% \(Tempo\ (segundos)\) & 244,71 & 48,23 & 76,69 & 306,68 & 160,64 & 184,95 & 236,53 & 149 & 487,29 & 81,53 & 197,625 \\ 
% \(Alfa\) & 1,00 & 1,00 & 1,00 & 1,00 & 1,00 & 1,00 & 1,00 & 1,00 & 1,00 & 1,00 & 1,00 \\ 
% \hline \hline
% \end{tabular}
% \end{adjustbox}
% \caption{Resultados obtidos para 2013/2 com \(\alpha = 1.00\)}
% \label{tbl_result_2013_alfa100}
% \end{table}

% \begin{table}[!h]
% \begin{adjustbox}{width=1\textwidth,center=\textwidth}
% \begin{tabular}{c|c|c|c|c|c|c|c|c|c|c|c}
% \hline \hline
% Penalização & \multicolumn{1}{c|}{S01} & \multicolumn{1}{c|}{S02} & \multicolumn{1}{c|}{S03} & \multicolumn{1}{c|}{S04} & \multicolumn{1}{c|}{S05} & \multicolumn{1}{c|}{S06} & \multicolumn{1}{c|}{S07} & \multicolumn{1}{c|}{S08} & \multicolumn{1}{c|}{S09} & S10 & Média \\ \hline
%  \(CP_p\) & 0 & 0 & 0 & 0 & 0 & 0 & 0 & 0 & 0 & 0 & 0 \\ 
% \(CT_t\) & 0 & 0 & 0 & 0 & 0 & 0 & 0 & 0 & 0 & 0 & 0 \\ 
% \(CS_s\) & 0 & 0 & 0 & 0 & 0 & 0 & 0 & 0 & 0 & 0 & 0 \\ 
% \(OFT\) & 0 & 0 & 0 & 0 & 0 & 0 & 0 & 0 & 0 & 0 & 0 \\ 
% \(VS_s\) & 0 & 0 & 0 & 0 & 0 & 0 & 0 & 0 & 0 & 0 & 0 \\ 
% \(TSI\) & 0 & 0 & 0 & 0 & 0 & 0 & 0 & 0 & 0 & 0 & 0 \\ 
% \(DH3\) & 0 & 0 & 0 & 0 & 0 & 0 & 0 & 0 & 0 & 0 & 0 \\ 
% \(IT_p\) & 7 & 7 & 7 & 5 & 7 & 7 & 7 & 6 & 7 & 5 & 6,5 \\ 
% \(JH_t\) & 0 & 0 & 0 & 0 & 0 & 0 & 0 & 0 & 0 & 0 & 0 \\ 
% \(PP_t\) & 18 & 9 & 21 & 19 & 18 & 15 & 12 & 23 & 18 & 19 & 17,2 \\ 
% \(AS_d\) & 0 & 0 & 0 & 0 & 0 & 0 & 0 & 0 & 0 & 0 & 0 \\ 
% \(ND_p\) & 0 & 0 & 0 & 0 & 0 & 0 & 0 & 0 & 0 & 0 & 0 \\ 
% \(ASD\) & 10 & 10 & 10 & 12 & 11 & 11 & 10 & 13 & 6 & 12 & 10,5 \\ 
% \(ADU\) & 10 & 11 & 11 & 10 & 9 & 9 & 13 & 9 & 8 & 9 & 9,9 \\ 
% \(DHP\) & 0 & 0 & 0 & 0 & 0 & 0 & 0 & 0 & 0 & 0 & 0 \\ 
% \(AHA\) & 0 & 0 & 0 & 0 & 0 & 0 & 0 & 0 & 0 & 0 & 0 \\ 
% \(AHFP\) & 0 & 0 & 0 & 0 & 0 & 0 & 0 & 0 & 0 & 0 & 0 \\ \hline 
% Função Objetivo & 342 & 316 & 364 & 346 & 342 & 330 & 348 & 372 & 282 & 336 & 337,8 \\
% \(Tempo\ (segundos)\) & 140,84 & 453,52 & 406,71 & 101,39 & 237,34 & 363,07 & 435,82 & 412,63 & 397,98 & 419,02 & 336,832 \\ 
% \(Alfa\) & 0,373424 & 0,835078 & 0,416883 & 0,230140 & 0,106235 & 0,538652 & 0,316141 & 0,460829 & 0,972076 & 0,324442 & 0,457390 \\ 
% \hline \hline
% \end{tabular}
% \end{adjustbox}
% \caption{Resultados obtidos para 2013/2 com \(\alpha = aleatorio\)}
% \label{tbl_result_alfa_2013_aleat}
% \end{table}

% % -----------------------------------------------------------------------------------------

% \begin{table}[!h]
% \begin{adjustbox}{width=1\textwidth,center=\textwidth}
% \begin{tabular}{c|c|c|c|c|c|c|c|c|c|c|c}
% \hline \hline
% Penalização & \multicolumn{1}{c|}{S01} & \multicolumn{1}{c|}{S02} & \multicolumn{1}{c|}{S03} & \multicolumn{1}{c|}{S04} & \multicolumn{1}{c|}{S05} & \multicolumn{1}{c|}{S06} & \multicolumn{1}{c|}{S07} & \multicolumn{1}{c|}{S08} & \multicolumn{1}{c|}{S09} & S10 & Média \\ \hline
% \(CP_p\) & 0 & 0 & 0 & 0 & 0 & 0 & 0 & 0 & 0 & 0 & 0 \\ 
% \(CT_t\) & 0 & 0 & 0 & 0 & 0 & 0 & 0 & 0 & 0 & 0 & 0 \\ 
% \(CS_s\) & 0 & 0 & 0 & 0 & 0 & 0 & 0 & 0 & 0 & 0 & 0 \\ 
% \(OFT\) & 0 & 0 & 0 & 0 & 0 & 0 & 0 & 0 & 0 & 0 & 0 \\ 
% \(VS_s\) & 0 & 0 & 0 & 0 & 0 & 0 & 0 & 0 & 0 & 0 & 0 \\ 
% \(TSI\) & 0 & 0 & 0 & 0 & 0 & 0 & 0 & 0 & 0 & 0 & 0 \\ 
% \(DH3\) & 0 & 0 & 0 & 0 & 0 & 0 & 0 & 0 & 0 & 0 & 0 \\ 
% \(IT_p\) & 5 & 3 & 4 & 5 & 3 & 4 & 4 & 4 & 4 & 6 & 4,2 \\ 
% \(JH_t\) & 0 & 0 & 0 & 0 & 0 & 0 & 0 & 0 & 0 & 0 & 0 \\ 
% \(PP_t\) & 9 & 16 & 12 & 14 & 12 & 12 & 16 & 11 & 12 & 12 & 12,6 \\ 
% \(AS_d\) & 0 & 0 & 0 & 0 & 0 & 0 & 0 & 0 & 0 & 0 & 0 \\ 
% \(ND_p\) & 0 & 0 & 0 & 0 & 0 & 0 & 0 & 0 & 0 & 0 & 0 \\ 
% \(ASD\) & 8 & 9 & 10 & 12 & 10 & 10 & 9 & 8 & 10 & 9 & 9,5 \\ 
% \(ADU\) & 8 & 8 & 7 & 4 & 6 & 4 & 6 & 5 & 6 & 6 & 6 \\ 
% \(DHP\) & 0 & 0 & 0 & 0 & 0 & 0 & 0 & 0 & 0 & 0 & 0 \\ 
% \(AHA\) & 0 & 0 & 0 & 0 & 0 & 0 & 0 & 0 & 0 & 0 & 0 \\ 
% \(AHFP\) & 0 & 0 & 0 & 0 & 0 & 0 & 0 & 0 & 0 & 0 & 0 \\ \hline 
% Função Objetivo & 246 & 264 & 258 & 266 & 238 & 228 & 254 & 214 & 248 & 258 & 247,4 \\
% \(Tempo\ (segundos)\) & 342,21 & 492,11 & 212,96 & 152,46 & 410,48 & 500,02 & 440,53 & 15,12 & 470,04 & 391,44 & 342,737 \\ 
% \(Alfa\) & 0,00 & 0,00 & 0,00 & 0,00 & 0,00 & 0,00 & 0,00 & 0,00 & 0,00 & 0,00 & 0,00 \\ 
% \hline \hline
% \end{tabular}
% \end{adjustbox}
% \caption{Resultados obtidos para 2016/1 com \(\alpha = 0.00\)}
% \label{tbl_result_2016_alfa0}
% \end{table}

% \begin{table}[!h]
% \begin{adjustbox}{width=1\textwidth,center=\textwidth}
% \begin{tabular}{c|c|c|c|c|c|c|c|c|c|c|c}
% \hline \hline
% Penalização & \multicolumn{1}{c|}{S01} & \multicolumn{1}{c|}{S02} & \multicolumn{1}{c|}{S03} & \multicolumn{1}{c|}{S04} & \multicolumn{1}{c|}{S05} & \multicolumn{1}{c|}{S06} & \multicolumn{1}{c|}{S07} & \multicolumn{1}{c|}{S08} & \multicolumn{1}{c|}{S09} & S10 & Média \\ \hline
% \(CP_p\) & 0 & 0 & 0 & 0 & 0 & 0 & 0 & 0 & 0 & 0 & 0 \\ 
% \(CT_t\) & 0 & 0 & 0 & 0 & 0 & 0 & 0 & 0 & 0 & 0 & 0 \\ 
% \(CS_s\) & 0 & 0 & 0 & 0 & 0 & 0 & 0 & 0 & 0 & 0 & 0 \\ 
% \(OFT\) & 0 & 0 & 0 & 0 & 0 & 0 & 0 & 0 & 0 & 0 & 0 \\ 
% \(VS_s\) & 0 & 0 & 0 & 0 & 0 & 0 & 0 & 0 & 0 & 0 & 0 \\ 
% \(TSI\) & 0 & 0 & 0 & 0 & 0 & 0 & 0 & 0 & 0 & 0 & 0 \\ 
% \(DH3\) & 0 & 0 & 0 & 0 & 0 & 0 & 0 & 0 & 0 & 0 & 0 \\ 
% \(IT_p\) & 3 & 4 & 4 & 5 & 4 & 4 & 4 & 5 & 4 & 4 & 4,1 \\ 
% \(JH_t\) & 0 & 0 & 0 & 0 & 0 & 0 & 0 & 0 & 0 & 0 & 0 \\ 
% \(PP_t\) & 12 & 11 & 12 & 13 & 13 & 12 & 12 & 15 & 11 & 14 & 12,5 \\ 
% \(AS_d\) & 0 & 0 & 0 & 0 & 0 & 0 & 0 & 0 & 0 & 0 & 0 \\ 
% \(ND_p\) & 0 & 0 & 0 & 0 & 0 & 0 & 0 & 0 & 0 & 0 & 0 \\ 
% \(ASD\) & 7 & 10 & 9 & 9 & 9 & 10 & 9 & 8 & 8 & 10 & 8,9 \\ 
% \(ADU\) & 6 & 4 & 5 & 5 & 6 & 6 & 8 & 6 & 6 & 6 & 5,8 \\ 
% \(DHP\) & 0 & 0 & 0 & 0 & 0 & 0 & 0 & 0 & 0 & 0 & 0 \\ 
% \(AHA\) & 0 & 0 & 0 & 0 & 0 & 0 & 0 & 0 & 0 & 0 & 0 \\ 
% \(AHFP\) & 0 & 0 & 0 & 0 & 0 & 0 & 0 & 0 & 0 & 0 & 0 \\ \hline 
% Função Objetivo & 208 & 224 & 228 & 242 & 242 & 248 & 258 & 250 & 224 & 256 & 238 \\
% \(Tempo\ (segundos)\) & 88,98 & 345,55 & 408,59 & 136,56 & 272,43 & 360,43 & 425,34 & 354,38 & 477,55 & 145,96 & 301,577 \\ 
% \(Alfa\) & 0,15 & 0,15 & 0,15 & 0,15 & 0,15 & 0,15 & 0,15 & 0,15 & 0,15 & 0,15 & 0,15 \\ 
% \hline \hline
% \end{tabular}
% \end{adjustbox}
% \caption{Resultados obtidos para 2016/1 com \(\alpha = 0.15\)}
% \label{tbl_result_2016_alfa015}
% \end{table}

% \begin{table}[!h]
% \begin{adjustbox}{width=1\textwidth,center=\textwidth}
% \begin{tabular}{c|c|c|c|c|c|c|c|c|c|c|c}
% \hline \hline
% Penalização & \multicolumn{1}{c|}{S01} & \multicolumn{1}{c|}{S02} & \multicolumn{1}{c|}{S03} & \multicolumn{1}{c|}{S04} & \multicolumn{1}{c|}{S05} & \multicolumn{1}{c|}{S06} & \multicolumn{1}{c|}{S07} & \multicolumn{1}{c|}{S08} & \multicolumn{1}{c|}{S09} & S10 & Média \\ \hline
% \(CP_p\) & 0 & 0 & 0 & 0 & 0 & 0 & 0 & 0 & 0 & 0 & 0 \\ 
% \(CT_t\) & 0 & 0 & 0 & 0 & 0 & 0 & 0 & 0 & 0 & 0 & 0 \\ 
% \(CS_s\) & 0 & 0 & 0 & 0 & 0 & 0 & 0 & 0 & 0 & 0 & 0 \\ 
% \(OFT\) & 0 & 0 & 0 & 0 & 0 & 0 & 0 & 0 & 0 & 0 & 0 \\ 
% \(VS_s\) & 0 & 0 & 0 & 0 & 0 & 0 & 0 & 0 & 0 & 0 & 0 \\ 
% \(TSI\) & 0 & 0 & 0 & 0 & 0 & 0 & 0 & 0 & 0 & 0 & 0 \\ 
% \(DH3\) & 0 & 0 & 0 & 0 & 0 & 0 & 0 & 0 & 0 & 0 & 0 \\ 
% \(IT_p\) & 5 & 3 & 5 & 4 & 4 & 4 & 4 & 4 & 3 & 3 & 3,9 \\ 
% \(JH_t\) & 0 & 0 & 0 & 0 & 0 & 0 & 0 & 0 & 0 & 0 & 0 \\ 
% \(PP_t\) & 12 & 18 & 11 & 10 & 14 & 11 & 10 & 15 & 15 & 10 & 12,6 \\ 
% \(AS_d\) & 0 & 0 & 0 & 0 & 0 & 0 & 0 & 0 & 0 & 0 & 0 \\ 
% \(ND_p\) & 0 & 0 & 0 & 0 & 0 & 0 & 0 & 0 & 0 & 0 & 0 \\ 
% \(ASD\) & 7 & 11 & 9 & 10 & 10 & 9 & 9 & 11 & 10 & 8 & 9,4 \\ 
% \(ADU\) & 7 & 4 & 4 & 5 & 5 & 5 & 6 & 4 & 6 & 5 & 5,1 \\ 
% \(DHP\) & 0 & 0 & 0 & 0 & 0 & 0 & 0 & 0 & 0 & 0 & 0 \\ 
% \(AHA\) & 0 & 0 & 0 & 0 & 0 & 0 & 0 & 0 & 0 & 0 & 0 \\ 
% \(AHFP\) & 0 & 0 & 0 & 0 & 0 & 0 & 0 & 0 & 0 & 0 & 0 \\ \hline 
% Função Objetivo & 238 & 252 & 224 & 230 & 246 & 224 & 230 & 250 & 250 & 200 & 234,4 \\
% \(Tempo\ (segundos)\) & 359,7 & 36,63 & 432,12 & 173,56 & 295,46 & 235,74 & 175,05 & 39,8 & 225,13 & 182,18 & 215,537 \\ 
% \(Alfa\) & 0,25 & 0,25 & 0,25 & 0,25 & 0,25 & 0,25 & 0,25 & 0,25 & 0,25 & 0,25 & 0,25 \\ 
% \hline \hline
% \end{tabular}
% \end{adjustbox}
% \caption{Resultados obtidos para 2016/1 com \(\alpha = 0.25\)}
% \label{tbl_result_2016_alfa025}
% \end{table}

% \begin{table}[!h]
% \begin{adjustbox}{width=1\textwidth,center=\textwidth}
% \begin{tabular}{c|c|c|c|c|c|c|c|c|c|c|c}
% \hline \hline
% Penalização & \multicolumn{1}{c|}{S01} & \multicolumn{1}{c|}{S02} & \multicolumn{1}{c|}{S03} & \multicolumn{1}{c|}{S04} & \multicolumn{1}{c|}{S05} & \multicolumn{1}{c|}{S06} & \multicolumn{1}{c|}{S07} & \multicolumn{1}{c|}{S08} & \multicolumn{1}{c|}{S09} & S10 & Média \\ \hline
% \(CP_p\) & 0 & 0 & 0 & 0 & 0 & 0 & 0 & 0 & 0 & 0 & 0 \\ 
% \(CT_t\) & 0 & 0 & 0 & 0 & 0 & 0 & 0 & 0 & 0 & 0 & 0 \\ 
% \(CS_s\) & 0 & 0 & 0 & 0 & 0 & 0 & 0 & 0 & 0 & 0 & 0 \\ 
% \(OFT\) & 0 & 0 & 0 & 0 & 0 & 0 & 0 & 0 & 0 & 0 & 0 \\ 
% \(VS_s\) & 0 & 0 & 0 & 0 & 0 & 0 & 0 & 0 & 0 & 0 & 0 \\ 
% \(TSI\) & 0 & 0 & 0 & 0 & 0 & 0 & 0 & 0 & 0 & 0 & 0 \\ 
% \(DH3\) & 0 & 0 & 0 & 0 & 0 & 0 & 0 & 0 & 0 & 0 & 0 \\ 
% \(IT_p\) & 4 & 6 & 5 & 3 & 3 & 4 & 4 & 5 & 3 & 4 & 4,1 \\ 
% \(JH_t\) & 0 & 0 & 0 & 0 & 0 & 0 & 0 & 0 & 0 & 0 & 0 \\ 
% \(PP_t\) & 15 & 13 & 15 & 13 & 15 & 17 & 13 & 11 & 16 & 14 & 14,2 \\ 
% \(AS_d\) & 0 & 0 & 0 & 0 & 0 & 0 & 0 & 0 & 0 & 0 & 0 \\ 
% \(ND_p\) & 0 & 0 & 0 & 0 & 0 & 0 & 0 & 0 & 0 & 0 & 0 \\ 
% \(ASD\) & 9 & 9 & 9 & 11 & 10 & 11 & 9 & 10 & 8 & 10 & 9,6 \\ 
% \(ADU\) & 3 & 4 & 4 & 5 & 5 & 4 & 6 & 7 & 7 & 4 & 4,9 \\ 
% \(DHP\) & 0 & 0 & 0 & 0 & 0 & 0 & 0 & 0 & 0 & 0 & 0 \\ 
% \(AHA\) & 0 & 0 & 0 & 0 & 0 & 0 & 0 & 0 & 0 & 0 & 0 \\ 
% \(AHFP\) & 0 & 0 & 0 & 0 & 0 & 0 & 0 & 0 & 0 & 0 & 0 \\ \hline 
% Função Objetivo & 220 & 242 & 240 & 242 & 240 & 258 & 242 & 264 & 244 & 236 & 242,8 \\
% \(Tempo\ (segundos)\) & 71,92 & 18,77 & 458,16 & 367,29 & 311,31 & 218,08 & 349,99 & 437,7 & 235,9 & 244,41 & 271,353 \\ 
% \(Alfa\) & 0,50 & 0,50 & 0,50 & 0,50 & 0,50 & 0,50 & 0,50 & 0,50 & 0,50 & 0,50 & 0,50 \\ 
% \hline \hline
% \end{tabular}
% \end{adjustbox}
% \caption{Resultados obtidos para 2016/1 com \(\alpha = 0.50\)}
% \label{tbl_result_2016_alfa050}
% \end{table}

% \begin{table}[!h]
% \begin{adjustbox}{width=1\textwidth,center=\textwidth}
% \begin{tabular}{c|c|c|c|c|c|c|c|c|c|c|c}
% \hline \hline
% Penalização & \multicolumn{1}{c|}{S01} & \multicolumn{1}{c|}{S02} & \multicolumn{1}{c|}{S03} & \multicolumn{1}{c|}{S04} & \multicolumn{1}{c|}{S05} & \multicolumn{1}{c|}{S06} & \multicolumn{1}{c|}{S07} & \multicolumn{1}{c|}{S08} & \multicolumn{1}{c|}{S09} & S10 & Média \\ \hline
% \(CP_p\) & 0 & 0 & 0 & 0 & 0 & 0 & 0 & 0 & 0 & 0 & 0 \\ 
% \(CT_t\) & 0 & 0 & 0 & 0 & 0 & 0 & 0 & 0 & 0 & 0 & 0 \\ 
% \(CS_s\) & 0 & 0 & 0 & 0 & 0 & 0 & 0 & 0 & 0 & 0 & 0 \\ 
% \(OFT\) & 0 & 0 & 0 & 0 & 0 & 0 & 0 & 0 & 0 & 0 & 0 \\ 
% \(VS_s\) & 0 & 0 & 0 & 0 & 0 & 0 & 0 & 0 & 0 & 0 & 0 \\ 
% \(TSI\) & 0 & 0 & 0 & 0 & 0 & 0 & 0 & 0 & 0 & 0 & 0 \\ 
% \(DH3\) & 0 & 0 & 0 & 0 & 0 & 0 & 0 & 0 & 0 & 0 & 0 \\ 
% \(IT_p\) & 5 & 5 & 4 & 5 & 4 & 5 & 4 & 5 & 4 & 4 & 4,5 \\ 
% \(JH_t\) & 0 & 0 & 0 & 0 & 0 & 0 & 0 & 0 & 0 & 0 & 0 \\ 
% \(PP_t\) & 12 & 11 & 15 & 14 & 13 & 14 & 11 & 12 & 15 & 14 & 13,1 \\ 
% \(AS_d\) & 0 & 0 & 0 & 0 & 0 & 0 & 0 & 0 & 0 & 0 & 0 \\ 
% \(ND_p\) & 0 & 0 & 0 & 0 & 0 & 0 & 0 & 0 & 0 & 0 & 0 \\ 
% \(ASD\) & 6 & 8 & 9 & 10 & 7 & 9 & 10 & 8 & 10 & 9 & 8,6 \\ 
% \(ADU\) & 4 & 5 & 6 & 4 & 5 & 6 & 4 & 5 & 6 & 5 & 5 \\ 
% \(DHP\) & 0 & 0 & 0 & 0 & 0 & 0 & 0 & 0 & 0 & 0 & 0 \\ 
% \(AHA\) & 0 & 0 & 0 & 0 & 0 & 0 & 0 & 0 & 0 & 0 & 0 \\ 
% \(AHFP\) & 0 & 0 & 0 & 0 & 0 & 0 & 0 & 0 & 0 & 0 & 0 \\ \hline 
% Função Objetivo & 198 & 224 & 250 & 246 & 212 & 256 & 224 & 228 & 260 & 236 & 233,4 \\
% \(Tempo\ (segundos)\) & 341,34 & 317,31 & 128,6 & 56,32 & 18,08 & 209,11 & 439,95 & 148,15 & 18,91 & 178,62 & 185,639 \\ 
% \(Alfa\) & 0,75 & 0,75 & 0,75 & 0,75 & 0,75 & 0,75 & 0,75 & 0,75 & 0,75 & 0,75 & 0,75 \\ 
% \hline \hline
% \end{tabular}
% \end{adjustbox}
% \caption{Resultados obtidos para 2016/1 com \(\alpha = 0.75\)}
% \label{tbl_result_2016_alfa075}
% \end{table}

% \begin{table}[!h]
% \begin{adjustbox}{width=1\textwidth,center=\textwidth}
% \begin{tabular}{c|c|c|c|c|c|c|c|c|c|c|c}
% \hline \hline
% Penalização & \multicolumn{1}{c|}{S01} & \multicolumn{1}{c|}{S02} & \multicolumn{1}{c|}{S03} & \multicolumn{1}{c|}{S04} & \multicolumn{1}{c|}{S05} & \multicolumn{1}{c|}{S06} & \multicolumn{1}{c|}{S07} & \multicolumn{1}{c|}{S08} & \multicolumn{1}{c|}{S09} & S10 & Média \\ \hline
% \(CP_p\) & 0 & 0 & 0 & 0 & 0 & 0 & 0 & 0 & 0 & 0 & 0 \\ 
% \(CT_t\) & 0 & 0 & 0 & 0 & 0 & 0 & 0 & 0 & 0 & 0 & 0 \\ 
% \(CS_s\) & 0 & 0 & 0 & 0 & 0 & 0 & 0 & 0 & 0 & 0 & 0 \\ 
% \(OFT\) & 0 & 0 & 0 & 0 & 0 & 0 & 0 & 0 & 0 & 0 & 0 \\ 
% \(VS_s\) & 0 & 0 & 0 & 0 & 0 & 0 & 0 & 0 & 0 & 0 & 0 \\ 
% \(TSI\) & 0 & 0 & 0 & 0 & 0 & 0 & 0 & 0 & 0 & 0 & 0 \\ 
% \(DH3\) & 0 & 0 & 0 & 0 & 0 & 0 & 0 & 0 & 0 & 0 & 0 \\ 
% \(IT_p\) & 3 & 3 & 5 & 5 & 4 & 4 & 5 & 4 & 6 & 5 & 4,4 \\ 
% \(JH_t\) & 0 & 0 & 0 & 0 & 0 & 0 & 0 & 0 & 0 & 0 & 0 \\ 
% \(PP_t\) & 14 & 11 & 15 & 13 & 11 & 16 & 12 & 14 & 10 & 11 & 12,7 \\ 
% \(AS_d\) & 0 & 0 & 0 & 0 & 0 & 0 & 0 & 0 & 0 & 0 & 0 \\ 
% \(ND_p\) & 0 & 0 & 0 & 0 & 0 & 0 & 0 & 0 & 0 & 0 & 0 \\ 
% \(ASD\) & 10 & 9 & 9 & 11 & 10 & 12 & 10 & 8 & 7 & 8 & 9,4 \\ 
% \(ADU\) & 5 & 5 & 6 & 6 & 6 & 5 & 5 & 4 & 7 & 6 & 5,5 \\ 
% \(DHP\) & 0 & 0 & 0 & 0 & 0 & 0 & 0 & 0 & 0 & 0 & 0 \\ 
% \(AHA\) & 0 & 0 & 0 & 0 & 0 & 0 & 0 & 0 & 0 & 0 & 0 \\ 
% \(AHFP\) & 0 & 0 & 0 & 0 & 0 & 0 & 0 & 0 & 0 & 0 & 0 \\ \hline 
% Função Objetivo & 236 & 214 & 260 & 272 & 244 & 274 & 248 & 216 & 240 & 234 & 243,8 \\
% \(Tempo\ (segundos)\) & 411,31 & 57,12 & 165,04 & 38,51 & 352,91 & 238,38 & 279,98 & 131,89 & 493,82 & 448,8 & 261,776 \\ 
% \(Alfa\) & 1,00 & 1,00 & 1,00 & 1,00 & 1,00 & 1,00 & 1,00 & 1,00 & 1,00 & 1,00 & 1,00 \\ 
% \hline \hline
% \end{tabular}
% \end{adjustbox}
% \caption{Resultados obtidos para 2016/1 com \(\alpha = 1.00\)}
% \label{tbl_result_2016_alfa100}
% \end{table}

% \begin{table}[!h]
% \begin{adjustbox}{width=1\textwidth,center=\textwidth}
% \begin{tabular}{c|c|c|c|c|c|c|c|c|c|c|c}
% \hline \hline
% Penalização & \multicolumn{1}{c|}{S01} & \multicolumn{1}{c|}{S02} & \multicolumn{1}{c|}{S03} & \multicolumn{1}{c|}{S04} & \multicolumn{1}{c|}{S05} & \multicolumn{1}{c|}{S06} & \multicolumn{1}{c|}{S07} & \multicolumn{1}{c|}{S08} & \multicolumn{1}{c|}{S09} & S10 & Média \\ \hline
% \(CP_p\) & 0 & 0 & 0 & 0 & 0 & 0 & 0 & 0 & 0 & 0 & 0 \\ 
% \(CT_t\) & 0 & 0 & 0 & 0 & 0 & 0 & 0 & 0 & 0 & 0 & 0 \\ 
% \(CS_s\) & 0 & 0 & 0 & 0 & 0 & 0 & 0 & 0 & 0 & 0 & 0 \\ 
% \(OFT\) & 0 & 0 & 0 & 0 & 0 & 0 & 0 & 0 & 0 & 0 & 0 \\ 
% \(VS_s\) & 0 & 0 & 0 & 0 & 0 & 0 & 0 & 0 & 0 & 0 & 0 \\ 
% \(TSI\) & 0 & 0 & 0 & 0 & 0 & 0 & 0 & 0 & 0 & 0 & 0 \\ 
% \(DH3\) & 0 & 0 & 0 & 0 & 0 & 0 & 0 & 0 & 0 & 0 & 0 \\ 
% \(IT_p\) & 3 & 3 & 5 & 4 & 2 & 4 & 4 & 5 & 5 & 3 & 3,8 \\ 
% \(JH_t\) & 0 & 0 & 0 & 0 & 0 & 0 & 0 & 0 & 0 & 0 & 0 \\ 
% \(PP_t\) & 14 & 13 & 15 & 13 & 11 & 14 & 11 & 13 & 12 & 16 & 13,2 \\ 
% \(AS_d\) & 0 & 0 & 0 & 0 & 0 & 0 & 0 & 0 & 0 & 0 & 0 \\ 
% \(ND_p\) & 0 & 0 & 0 & 0 & 0 & 0 & 0 & 0 & 0 & 0 & 0 \\ 
% \(ASD\) & 9 & 9 & 10 & 9 & 11 & 9 & 8 & 10 & 9 & 9 & 9,3 \\ 
% \(ADU\) & 6 & 5 & 4 & 4 & 5 & 5 & 7 & 4 & 5 & 6 & 5,1 \\ 
% \(DHP\) & 0 & 0 & 0 & 0 & 0 & 0 & 0 & 0 & 0 & 0 & 0 \\ 
% \(AHA\) & 0 & 0 & 0 & 0 & 0 & 0 & 0 & 0 & 0 & 0 & 0 \\ 
% \(AHFP\) & 0 & 0 & 0 & 0 & 0 & 0 & 0 & 0 & 0 & 0 & 0 \\ \hline 
% Função Objetivo & 236 & 222 & 250 & 222 & 224 & 236 & 234 & 242 & 238 & 244 & 234,8 \\
% \(Tempo\ (segundos)\) & 129,6 & 434,59 & 346,08 & 454,48 & 173,25 & 160,57 & 467,52 & 187,36 & 143,63 & 361,4 & 285,848 \\ 
% \(Alfa\) & 0,510941 & 0,632710 & 0,212043 & 0,440718 & 0,806024 & 0,862484 & 0,445692 & 0,333445 & 0,515641 & 0,801355 & 0,556105 \\ 
% \hline \hline
% \end{tabular}
% \end{adjustbox}
% \caption{Resultados obtidos para 2016/1 com \(\alpha = aleatorio\)}
% \label{tbl_result_2016_aleat}
% \end{table}


% \chapter{Tabela de horários do CCENS-UFES}
% \label{ap-horarios}


\chapter{Tabela de horários obtidas pelo GRASP para Salas do período 2013/2}


\begin{table}[!h]
\begin{adjustbox}{width=1\textwidth,center=\textwidth}
\centering
\begin{tabular}{|c|c|c|c|c|c|}
\hline
 & Segunda-Feira & Terça-Feira & Quarta-Feira & Quinta-Feira & Sexta-Feira \\ \hline
07h00 &  &  &  &  &  \\ \hline
08h00 & \begin{tabular}[c]{@{}c@{}}Lógica \\ Computacional II\end{tabular} &  &  &  &  \\ \hline
09h00 & \begin{tabular}[c]{@{}c@{}}Lógica \\ Computacional II\end{tabular} &  &  &  &  \\ \hline
10h00 & \begin{tabular}[c]{@{}c@{}}Arquitetura de \\ Computadores\end{tabular} &  &  &  & \begin{tabular}[c]{@{}c@{}}Introdução a \\ Ciência da Computação\end{tabular} \\ \hline
11h00 & \begin{tabular}[c]{@{}c@{}}Arquitetura de \\ Computadores\end{tabular} &  &  &  & \begin{tabular}[c]{@{}c@{}}Introdução a \\ Ciência da Computação\end{tabular} \\ \hline
13h00 &  &  &  &  & \begin{tabular}[c]{@{}c@{}}Teoria \\ da Computação\end{tabular} \\ \hline
14h00 &  &  &  &  & \begin{tabular}[c]{@{}c@{}}Teoria \\ da Computação\end{tabular} \\ \hline
15h00 & Compiladores &  &  &  &  \\ \hline
16h00 & Compiladores &  &  &  &  \\ \hline
18h00 &  &  &  &  &  \\ \hline
19h00 &  &  &  &  &  \\ \hline
20h00 &  &  &  &  &  \\ \hline
21h00 &  &  &  &  &  \\ \hline
22h00 &  &  &  &  &  \\ \hline
\end{tabular}
\end{adjustbox}
\begin{adjustbox}{width=1\textwidth,center=\textwidth}
\centering
\begin{tabular}{|c|c|c|c|c|c|}
\hline
\textbf{Disciplina} & \textbf{Curso} & \textbf{Período} & \textbf{Professor} & \textbf{Capacidade} & \textbf{Número de Vagas} \\ \hline
Lógica Computacional II & CC & 3 & Juliana Pinheiro Campos & 40 & 40 \\ \hline
Arquitetura de Computadores & CC & 3 & Valeria Alves Da Silva & 40 & 30 \\ \hline
Compiladores & CC & 7 & Rodrigo Freitas Silva & 40 & 20 \\ \hline
Introdução a Ciência da Computação & CC & 1 & Bruno Vilela Oliveira & 40 & 40 \\ \hline
Teoria da Computação & CC & 5 & Juliana Pinheiro Campos & 40 & 30 \\ \hline
\end{tabular}
\end{adjustbox}
\caption{Prédio Central - Sala 09}
\end{table}

\begin{table}[!h]
\begin{adjustbox}{width=1\textwidth,center=\textwidth}
\centering
\begin{tabular}{|c|c|c|c|c|c|}
\hline
 & Segunda-Feira & Terça-Feira & Quarta-Feira & Quinta-Feira & Sexta-Feira \\ \hline
07h00 &  &  &  &  &  \\ \hline
08h00 &  &  &  &  &  \\ \hline
09h00 &  &  &  &  &  \\ \hline
10h00 &  &  &  &  &  \\ \hline
11h00 &  &  &  &  &  \\ \hline
13h00 & Compiladores &  &  & \begin{tabular}[c]{@{}c@{}}Administração\\ e Economia\end{tabular} &  \\ \hline
14h00 & Compiladores &  &  & \begin{tabular}[c]{@{}c@{}}Administração\\ e Economia\end{tabular} &  \\ \hline
15h00 & Cálculo C &  & Cálculo C & \begin{tabular}[c]{@{}c@{}}Administração\\ e Economia\end{tabular} &  \\ \hline
16h00 & Cálculo C &  & Cálculo C & \begin{tabular}[c]{@{}c@{}}Administração\\ e Economia\end{tabular} &  \\ \hline
18h00 &  & Álgebra linear &  &  &  \\ \hline
19h00 &  & Álgebra linear &  &  &  \\ \hline
20h00 &  &  &  & Álgebra linear &  \\ \hline
21h00 &  &  &  & Álgebra linear &  \\ \hline
22h00 &  &  &  &  &  \\ \hline
\end{tabular}
\end{adjustbox}
\begin{adjustbox}{width=1\textwidth,center=\textwidth}
\centering
\begin{tabular}{|c|c|c|c|c|c|}
\hline
\textbf{Disciplina} & \textbf{Curso} & \textbf{Período} & \textbf{Professor} & \textbf{Capacidade} & \textbf{Número de Vagas} \\ \hline
Compiladores & CC & 7 & Rodrigo Freitas Silva & 55 & 20 \\ \hline
Cálculo C & CC & 3 & Átila & 55 & 30 \\ \hline
Álgebra linear & SI & 3 & Eleonésio Strey & 55 & 45 \\ \hline
Administração e Economia & CC & 7 & Clóvis & 55 & 30 \\ \hline
\end{tabular}
\end{adjustbox}
\caption{Prédio Central - Sala 04}
\end{table}

\begin{table}[!h]
\begin{adjustbox}{width=1\textwidth,center=\textwidth}
\centering
\begin{tabular}{|c|c|c|c|c|c|}
\hline
 & Segunda-Feira & Terça-Feira & Quarta-Feira & Quinta-Feira & Sexta-Feira \\ \hline
07h00 &  &  &  &  &  \\ \hline
08h00 &  &  &  &  &  \\ \hline
09h00 &  &  &  &  &  \\ \hline
10h00 &  &  &  &  &  \\ \hline
11h00 &  &  &  &  &  \\ \hline
13h00 &  & \multicolumn{1}{c|}{\begin{tabular}[c]{@{}c@{}}Linguagens \\ de Programação\end{tabular}} &  &  &  \\ \hline
14h00 &  & \multicolumn{1}{c|}{\begin{tabular}[c]{@{}c@{}}Linguagens \\ de Programação\end{tabular}} &  &  &  \\ \hline
15h00 &  &  &  &  &  \\ \hline
16h00 &  &  &  &  &  \\ \hline
18h00 &  &  &  &  &  \\ \hline
19h00 &  &  &  &  &  \\ \hline
20h00 &  &  &  &  &  \\ \hline
21h00 &  &  &  &  &  \\ \hline
22h00 &  &  &  &  &  \\ \hline
\end{tabular}
\end{adjustbox}
\begin{adjustbox}{width=1\textwidth,center=\textwidth}
\centering
\begin{tabular}{|c|c|c|c|c|c|}
\hline
\textbf{Disciplina} & \textbf{Curso} & \textbf{Período} & \textbf{Professor} & \textbf{Capacidade} & \textbf{Número de Vagas} \\ \hline
Linguagens de Programação & CC & 5 & Rodrigo Freitas Silva & 90 & 20 \\ \hline
\end{tabular}
\end{adjustbox}
\caption{Prédio Novo - Sala 03}
\end{table}

\begin{table}[!h]
\begin{adjustbox}{width=1\textwidth,center=\textwidth}
\centering
\begin{tabular}{|c|c|c|c|c|c|}
\hline
 & Segunda-Feira & Terça-Feira & Quarta-Feira & Quinta-Feira & Sexta-Feira \\ \hline
07h00 &  &  &  &  &  \\ \hline
08h00 & Estatística &  & Estatística &  &  \\ \hline
09h00 & Estatística &  & Estatística &  &  \\ \hline
10h00 & \begin{tabular}[c]{@{}c@{}}Teoria da \\ Computação\end{tabular} &  & \begin{tabular}[c]{@{}c@{}}Fundamentos \\ de Física III\end{tabular} &  &  \\ \hline
11h00 & \begin{tabular}[c]{@{}c@{}}Teoria da \\ Computação\end{tabular} &  & \begin{tabular}[c]{@{}c@{}}Fundamentos \\ de Física III\end{tabular} &  &  \\ \hline
13h00 &  & \begin{tabular}[c]{@{}c@{}}Lógica \\ Computacional II\end{tabular} & \begin{tabular}[c]{@{}c@{}}Arquitetura de \\ Computadores\end{tabular} & \begin{tabular}[c]{@{}c@{}}Engenharia de \\ Requisitos de Software\end{tabular} &  \\ \hline
14h00 &  & \begin{tabular}[c]{@{}c@{}}Lógica \\ Computacional II\end{tabular} & \begin{tabular}[c]{@{}c@{}}Arquitetura de \\ Computadores\end{tabular} & \begin{tabular}[c]{@{}c@{}}Engenharia de \\ Requisitos de Software\end{tabular} &  \\ \hline
15h00 & \begin{tabular}[c]{@{}c@{}}Inglês \\ Instrumental\end{tabular} &  &  &  & \begin{tabular}[c]{@{}c@{}}Fundamentos \\ de Física III\end{tabular} \\ \hline
16h00 & \begin{tabular}[c]{@{}c@{}}Inglês \\ Instrumental\end{tabular} &  &  &  & \begin{tabular}[c]{@{}c@{}}Fundamentos \\ de Física III\end{tabular} \\ \hline
18h00 & \begin{tabular}[c]{@{}c@{}}Sistemas\\ Distribuídos\end{tabular} &  & \begin{tabular}[c]{@{}c@{}}Engenharia \\ de Software\end{tabular} & \begin{tabular}[c]{@{}c@{}}Administração \\ e Economia\end{tabular} &  \\ \hline
19h00 & \begin{tabular}[c]{@{}c@{}}Sistemas\\ Distribuídos\end{tabular} &  & \begin{tabular}[c]{@{}c@{}}Engenharia \\ de Software\end{tabular} & \begin{tabular}[c]{@{}c@{}}Administração \\ e Economia\end{tabular} &  \\ \hline
20h00 & \begin{tabular}[c]{@{}c@{}}Sistema de \\ Apoio a Decisão\end{tabular} &  & \begin{tabular}[c]{@{}c@{}}Sistema de \\ Apoio a Decisão\end{tabular} & \begin{tabular}[c]{@{}c@{}}Administração \\ e Economia\end{tabular} & \begin{tabular}[c]{@{}c@{}}Sistemas \\ Operacionais\end{tabular} \\ \hline
21h00 & \begin{tabular}[c]{@{}c@{}}Sistema de \\ Apoio a Decisão\end{tabular} & & \begin{tabular}[c]{@{}c@{}}Sistema de \\ Apoio a Decisão\end{tabular} & \begin{tabular}[c]{@{}c@{}}Administração \\ e Economia\end{tabular} & \begin{tabular}[c]{@{}c@{}}Sistemas \\ Operacionais\end{tabular} \\ \hline
22h00 &  &  &  &  &  \\ \hline
\end{tabular}
\end{adjustbox}
\begin{adjustbox}{width=1\textwidth,center=\textwidth}
\centering
\begin{tabular}{|c|c|c|c|c|c|}
\hline
\textbf{Disciplina} & \textbf{Curso} & \textbf{Período} & \textbf{Professor} & \textbf{Capacidade} & \textbf{Número de Vagas} \\ \hline
Arquitetura de Computadores & CC & 3 & Valeria Alves Da Silva & 55 & 30 \\ \hline
Lógica Computacional II & CC & 3 & Juliana Pinheiro Campos & 55 & 40 \\ \hline
Engenharia de Requisitos de Software & CC & 5 & Clayton Vieira Fraga Filho & 55 & 30 \\ \hline
Teoria da Computação & CC & 5 & Juliana Pinheiro Campos & 55 & 30 \\ \hline
Engenharia de Software & SI & 3 & Bruno Vilela Oliveira & 55 & 30 \\ \hline
Sistema de Apoio a Decisão & SI & 3 & Simone Dornelas Costa & 55 & 40 \\ \hline
Sistemas Operacionais & SI & 5 & Jacson Rodrigues Correia Da Silva & 55 & 20 \\ \hline
Sistemas Distribuídos & SI & 7 & Helder De Amorim Mendes & 55 & 20 \\ \hline
Inglês Instrumental & CC & 1 & Alexandre Rosa & 55 & 30 \\ \hline
Fundamentos de Física III & CC & 3 & Ronald & 55 & 30 \\ \hline
Estatística & CC & 5 & Maristela & 55 & 30 \\ \hline
Administração e Economia & SI & 7 & Clóvis & 55 & 40 \\ \hline
\end{tabular}
\end{adjustbox}
\caption{Prédio Novo - Sala 01}
\end{table}



\begin{table}[!h]
\begin{adjustbox}{width=1\textwidth,center=\textwidth}
\centering
\begin{tabular}{|c|c|c|c|c|c|}
\hline
 & Segunda-Feira & Terça-Feira & Quarta-Feira & Quinta-Feira & Sexta-Feira \\ \hline
07h00 &  &  &  &  &  \\ \hline
08h00 &  &  &  &  &  \\ \hline
09h00 &  &  &  &  &  \\ \hline
10h00 &  &  &  &  &  \\ \hline
11h00 &  &  &  &  &  \\ \hline
13h00 &  &  &  &  &  \\ \hline
14h00 &  &  &  &  &  \\ \hline
15h00 &  &  &  &  &  \\ \hline
16h00 &  &  &  &  &  \\ \hline
18h00 & \begin{tabular}[c]{@{}c@{}}Computabilidade \\ e Complexidade\end{tabular} & \begin{tabular}[c]{@{}c@{}}Tópicos Especiais \\ em Informática I\end{tabular} &  & \begin{tabular}[c]{@{}c@{}}Tópicos Especiais \\ em Informática I\end{tabular} & \begin{tabular}[c]{@{}c@{}}Segurança e \\ Auditoria de Sistemas\end{tabular} \\ \hline
19h00 & \begin{tabular}[c]{@{}c@{}}Computabilidade \\ e Complexidade\end{tabular} & \begin{tabular}[c]{@{}c@{}}Tópicos Especiais \\ em Informática I\end{tabular} &  & \begin{tabular}[c]{@{}c@{}}Tópicos Especiais \\ em Informática I\end{tabular} & \begin{tabular}[c]{@{}c@{}}Segurança e \\ Auditoria de Sistemas\end{tabular} \\ \hline
20h00 & \begin{tabular}[c]{@{}c@{}}Otimização \\ Linear\end{tabular} & \begin{tabular}[c]{@{}c@{}}Comércio \\ Eletrônico\end{tabular} & \begin{tabular}[c]{@{}c@{}}Sistemas de \\ Software Livre\end{tabular} &  &  \\ \hline
21h00 & \begin{tabular}[c]{@{}c@{}}Otimização \\ Linear\end{tabular} & \begin{tabular}[c]{@{}c@{}}Comércio \\ Eletrônico\end{tabular} & \begin{tabular}[c]{@{}c@{}}Sistemas de \\ Software Livre\end{tabular} &  &  \\ \hline
22h00 &  &  &  &  &  \\ \hline
\end{tabular}
\end{adjustbox}
\begin{adjustbox}{width=1\textwidth,center=\textwidth}
\centering
\begin{tabular}{|c|c|c|c|c|c|}
\hline
\textbf{Disciplina} & \textbf{Curso} & \textbf{Período} & \textbf{Professor} & \textbf{Capacidade} & \textbf{Número de Vagas} \\ \hline
Computabilidade e Complexidade & SI & 3 & Juliana Pinheiro Campos & 65 & 40 \\ \hline
Otimização Linear & SI & 5 & Geraldo Regis Mauri & 65 & 20 \\ \hline
Comércio Eletrônico & SI & 7 & Simone Dornelas Costa & 65 & 20 \\ \hline
Segurança e Auditoria de Sistemas & SI & 7 & Simone Dornelas Costa & 65 & 20 \\ \hline
Sistemas de Software Livre & SI & 9 & Jacson Rodrigues Correia Da Silva & 65 & 20 \\ \hline
Tópicos Especiais em Informática I & SI & 9 & Valeria Alves Da Silva & 65 & 20 \\ \hline
\end{tabular}
\end{adjustbox}
\caption{Prédio Novo - Sala 12}
\end{table}

\begin{table}[!h]
\begin{adjustbox}{width=1\textwidth,center=\textwidth}
\centering
\begin{tabular}{|c|c|c|c|c|c|}
\hline
 & Segunda-Feira & Terça-Feira & Quarta-Feira & Quinta-Feira & Sexta-Feira \\ \hline
07h00 &  &  &  &  &  \\ \hline
08h00 &  &  &  &  & \begin{tabular}[c]{@{}c@{}}Matemática \\ Discreta\end{tabular} \\ \hline
09h00 &  &  &  &  & \begin{tabular}[c]{@{}c@{}}Matemática \\ Discreta\end{tabular} \\ \hline
10h00 &  &  &  &  &  \\ \hline
11h00 &  &  &  &  &  \\ \hline
13h00 & Cálculo A & \begin{tabular}[c]{@{}c@{}}Vetores e \\ Geometria Analítica\end{tabular} & Cálculo A & \begin{tabular}[c]{@{}c@{}}Vetores e \\ Geometria Analítica\end{tabular} & Cálculo A \\ \hline
14h00 & Cálculo A & \begin{tabular}[c]{@{}c@{}}Vetores e \\ Geometria Analítica\end{tabular} & Cálculo A & \begin{tabular}[c]{@{}c@{}}Vetores e \\ Geometria Analítica\end{tabular} & Cálculo A \\ \hline
15h00 &  &  &  & \begin{tabular}[c]{@{}c@{}}Introdução a \\ Ciência da Computação\end{tabular} &  \\ \hline
16h00 &  &  &  & \begin{tabular}[c]{@{}c@{}}Introdução a \\ Ciência da Computação\end{tabular} &  \\ \hline
18h00 &  &  &  &  &  \\ \hline
19h00 &  &  &  &  &  \\ \hline
20h00 &  &  &  &  &  \\ \hline
21h00 &  &  &  &  &  \\ \hline
22h00 &  &  &  &  &  \\ \hline
\end{tabular}
\end{adjustbox}
\begin{adjustbox}{width=1\textwidth,center=\textwidth}
\centering
\begin{tabular}{|c|c|c|c|c|c|}
\hline
\textbf{Disciplina} & \textbf{Curso} & \textbf{Período} & \textbf{Professor} & \textbf{Capacidade} & \textbf{Número de Vagas} \\ \hline
Introdução a Ciência da Computação & CC & 1 & Bruno Vilela Oliveira & 90 & 40 \\ \hline
Matemática Discreta & CC & 1 & Edmar Hell Kampke & 90 & 40 \\ \hline
Cálculo A & CC & 1 & Tharso & 90 & 30 \\ \hline
Vetores e Geometria Analítica & CC & 1 & Bernado & 90 & 30 \\ \hline
\end{tabular}
\end{adjustbox}
\caption{Prédio Novo - Sala 09}
\end{table}

\begin{table}[!h]
\begin{adjustbox}{width=1\textwidth,center=\textwidth}
\centering
\begin{tabular}{|c|c|c|c|c|c|}
\hline
 & Segunda-Feira & Terça-Feira & Quarta-Feira & Quinta-Feira & Sexta-Feira \\ \hline
07h00 &  &  &  &  &  \\ \hline
08h00 &  &  &  &  &  \\ \hline
09h00 &  &  &  &  &  \\ \hline
10h00 &  &  &  &  &  \\ \hline
11h00 &  &  &  &  &  \\ \hline
13h00 &  &  &  &  &  \\ \hline
14h00 &  &  &  &  &  \\ \hline
15h00 &  &  &  &  &  \\ \hline
16h00 &  &  &  &  &  \\ \hline
18h00 & \begin{tabular}[c]{@{}c@{}}Algoritmos \\ Numéricos\end{tabular} & \begin{tabular}[c]{@{}c@{}}Sistemas \\ Distribuídos\end{tabular} & Libras & \begin{tabular}[c]{@{}c@{}}Computabilidade \\ e Complexidade\end{tabular} & Libras \\ \hline
19h00 & \begin{tabular}[c]{@{}c@{}}Algoritmos \\ Numéricos\end{tabular} & \begin{tabular}[c]{@{}c@{}}Sistemas \\ Distribuídos\end{tabular} & Libras & \begin{tabular}[c]{@{}c@{}}Computabilidade \\ e Complexidade\end{tabular} & Libras \\ \hline
20h00 & &  & \begin{tabular}[c]{@{}c@{}}Informática \\ e Sociedade\end{tabular} &  &  \\ \hline
21h00 & &  & \begin{tabular}[c]{@{}c@{}}Informática \\ e Sociedade\end{tabular} &  &  \\ \hline
22h00 &  &  &  &  &  \\ \hline
\end{tabular}
\end{adjustbox}
\begin{adjustbox}{width=1\textwidth,center=\textwidth}
\centering
\begin{tabular}{|c|c|c|c|c|c|}
\hline
\textbf{Disciplina} & \textbf{Curso} & \textbf{Período} & \textbf{Professor} & \textbf{Capacidade} & \textbf{Número de Vagas} \\ \hline
Computabilidade e Complexidade & SI & 3 & Juliana Pinheiro Campos & 70 & 40 \\ \hline
Informática e Sociedade & SI & 7 & Larice Nogueira De Andrade & 70 & 30 \\ \hline
Sistemas Distribuídos & SI & 7 & Helder De Amorim Mendes & 70 & 20 \\ \hline
Algoritmos Numéricos & MA & 5 & Thiago Meireles Paixao & 70 & 15 \\ \hline
Libras & SI & 9 & Aline De Menezes & 70 & 20 \\ \hline
\end{tabular}
\end{adjustbox}
\caption{Prédio Antigo - Sala 03}
\end{table}

\begin{table}[!h]
\begin{adjustbox}{width=1\textwidth,center=\textwidth}
\centering
\begin{tabular}{|c|c|c|c|c|c|}
\hline
 & Segunda-Feira & Terça-Feira & Quarta-Feira & Quinta-Feira & Sexta-Feira \\ \hline
07h00 &  &  &  &  &  \\ \hline
08h00 &  &  &  &  &  \\ \hline
09h00 &  &  &  &  &  \\ \hline
10h00 &  &  &  &  &  \\ \hline
11h00 &  &  &  &  &  \\ \hline
13h00 &  &  &  &  &  \\ \hline
14h00 &  &  &  &  &  \\ \hline
15h00 &  &  &  &  &  \\ \hline
16h00 &  &  &  &  &  \\ \hline
18h00 & \begin{tabular}[c]{@{}c@{}}Português \\ Instrumental\end{tabular} & \begin{tabular}[c]{@{}c@{}}Vetores e \\ Geometria Analítica\end{tabular} & \begin{tabular}[c]{@{}c@{}}Lógica \\ Computacional\end{tabular} & \begin{tabular}[c]{@{}c@{}}Vetores e \\ Geometria Analítica\end{tabular} & \begin{tabular}[c]{@{}c@{}}Lógica \\ Computacional\end{tabular} \\ \hline
19h00 & \begin{tabular}[c]{@{}c@{}}Português \\ Instrumental\end{tabular} & \begin{tabular}[c]{@{}c@{}}Vetores e \\ Geometria Analítica\end{tabular} & \begin{tabular}[c]{@{}c@{}}Lógica \\ Computacional\end{tabular} & \begin{tabular}[c]{@{}c@{}}Vetores e \\ Geometria Analítica\end{tabular} & \begin{tabular}[c]{@{}c@{}}Lógica \\ Computacional\end{tabular} \\ \hline
20h00 & \begin{tabular}[c]{@{}c@{}}Introdução aos \\ Sistemas de Informação\end{tabular} & \begin{tabular}[c]{@{}c@{}}Projeto de \\ Sistemas de Software\end{tabular} &  & \begin{tabular}[c]{@{}c@{}}Interface \\ Humano-Computador\end{tabular} &  \\ \hline
21h00 & \begin{tabular}[c]{@{}c@{}}Introdução aos \\ Sistemas de Informação\end{tabular} & \begin{tabular}[c]{@{}c@{}}Projeto de \\ Sistemas de Software\end{tabular} &  & \begin{tabular}[c]{@{}c@{}}Interface \\ Humano-Computador\end{tabular} &  \\ \hline
22h00 &  &  &  &  &  \\ \hline
\end{tabular}
\end{adjustbox}
\begin{adjustbox}{width=1\textwidth,center=\textwidth}
\centering
\begin{tabular}{|c|c|c|c|c|c|}
\hline
\textbf{Disciplina} & \textbf{Curso} & \textbf{Período} & \textbf{Professor} & \textbf{Capacidade} & \textbf{Número de Vagas} \\ \hline
Introdução aos Sistemas de Informação & SI & 1 & Edmar Hell Kampke & 80 & 70 \\ \hline
Lógica Computacional & SI & 1 & Jacson Rodrigues Correia Da Silva & 80 & 70 \\ \hline
Projeto de Sistemas de Software & SI & 5 & Clayton Vieira Fraga Filho & 80 & 30 \\ \hline
Interface Humano-Computador & SI & 5 & Simone Dornelas Costa & 80 & 20 \\ \hline
Português Instrumental & SI & 1 & Alexandre Rosa & 80 & 70 \\ \hline
Vetores e Geometria Analítica & SI & 1 & Paulo Henrique Souza & 80 & 70 \\ \hline
\end{tabular}
\end{adjustbox}
\caption{Prédio Central - Sala 0}
\end{table}

\begin{table}[!h]
\begin{adjustbox}{width=1\textwidth,center=\textwidth}
\centering
\begin{tabular}{|c|c|c|c|c|c|}
\hline
 & Segunda-Feira & Terça-Feira & Quarta-Feira & Quinta-Feira & Sexta-Feira \\ \hline
07h00 &  &  &  &  &  \\ \hline
08h00 &  &  &  &  &  \\ \hline
09h00 &  &  &  &  &  \\ \hline
10h00 &  &  & \begin{tabular}[c]{@{}c@{}}Gerência de \\ Projeto de Software\end{tabular} &  &  \\ \hline
11h00 &  &  & \begin{tabular}[c]{@{}c@{}}Gerência de \\ Projeto de Software\end{tabular} &  &  \\ \hline
13h00 & \begin{tabular}[c]{@{}c@{}}Metodologia de \\ Pesquisa em Informática\end{tabular} &  & \begin{tabular}[c]{@{}c@{}}Inteligência \\ Artificial\end{tabular} &  & Programação I \\ \hline
14h00 & \begin{tabular}[c]{@{}c@{}}Metodologia de \\ Pesquisa em Informática\end{tabular} &  & \begin{tabular}[c]{@{}c@{}}Inteligência \\ Artificial\end{tabular} &  & Programação I \\ \hline
15h00 & \begin{tabular}[c]{@{}c@{}}Desenvolvimento de \\ Sistemas para WEB\end{tabular} &  & \begin{tabular}[c]{@{}c@{}}Gerenciamento de \\ Banco de Dados\end{tabular} &  &  \\ \hline
16h00 & \begin{tabular}[c]{@{}c@{}}Desenvolvimento de \\ Sistemas para WEB\end{tabular} &  & \begin{tabular}[c]{@{}c@{}}Gerenciamento de \\ Banco de Dados\end{tabular} &  &  \\ \hline
18h00 & \begin{tabular}[c]{@{}c@{}}Introdução \\ a Informática\end{tabular} &  &  & Programação I & \begin{tabular}[c]{@{}c@{}}Estrutura \\ de Dados I\end{tabular} \\ \hline
19h00 & \begin{tabular}[c]{@{}c@{}}Introdução \\ a Informática\end{tabular} &  &  & Programação I & \begin{tabular}[c]{@{}c@{}}Estrutura \\ de Dados I\end{tabular} \\ \hline
20h00 &  &  & \begin{tabular}[c]{@{}c@{}}Introdução \\ a Informática\end{tabular} &  &  \\ \hline
21h00 &  &  & \begin{tabular}[c]{@{}c@{}}Introdução \\ a Informática\end{tabular} &  &  \\ \hline
22h00 &  &  &  &  &  \\ \hline
\end{tabular}
\end{adjustbox}
\begin{adjustbox}{width=1\textwidth,center=\textwidth}
\centering
\begin{tabular}{|c|c|c|c|c|c|}
\hline
\textbf{Disciplina} & \textbf{Curso} & \textbf{Período} & \textbf{Professor} & \textbf{Capacidade} & \textbf{Número de Vagas} \\ \hline
Metodologia de Pesquisa em Informática & CC & 3 & Larice Nogueira De Andrade & 40 & 30 \\ \hline
Gerência de Projeto de Software & CC & 7 & Clayton Vieira Fraga Filho & 40 & 20 \\ \hline
Inteligência Artificial & CC & 7 & Jacson Rodrigues Correia Da Silva & 40 & 20 \\ \hline
Desenvolvimento de Sistemas para WEB & CC & 7 & Bruno Vilela Oliveira & 40 & 20 \\ \hline
Gerenciamento de Banco de Dados & CC & 7 & Antonio Almeida De Barros Junior & 40 & 20 \\ \hline
Programação I & SI & 1 & Thiago Meireles Paixao & 40 & 30 \\ \hline
Introdução a Informática & MA & 1 & Bruno Vilela Oliveira & 40 & 35 \\ \hline
Estrutura de Dados I & SI & 3 & Juliana Pinheiro Campos & 40 & 30 \\ \hline
\end{tabular}
\end{adjustbox}
\caption{ChiChiu - Laboratório 03}
\end{table}

\begin{table}[!h]
\begin{adjustbox}{width=1\textwidth,center=\textwidth}
\centering
\begin{tabular}{|c|c|c|c|c|c|}
\hline
 & Segunda-Feira & Terça-Feira & Quarta-Feira & Quinta-Feira & Sexta-Feira \\ \hline
07h00 &  &  &  &  &  \\ \hline
08h00 &  & \begin{tabular}[c]{@{}c@{}}Engenharia de \\ Requisitos de Software\end{tabular} &  &  &  \\ \hline
09h00 &  & \begin{tabular}[c]{@{}c@{}}Engenharia de \\ Requisitos de Software\end{tabular} &  &  &  \\ \hline
10h00 &  & Programação I &  &  &  \\ \hline
11h00 &  & Programação I &  &  &  \\ \hline
13h00 & \begin{tabular}[c]{@{}c@{}}Lógica e Técnicas \\ de Programação\end{tabular} & \begin{tabular}[c]{@{}c@{}}Gerência de \\ Projeto de Software\end{tabular} & \begin{tabular}[c]{@{}c@{}}Redes de \\ Computadores\end{tabular} &  &  \\ \hline
14h00 & \begin{tabular}[c]{@{}c@{}}Lógica e Técnicas \\ de Programação\end{tabular} & \begin{tabular}[c]{@{}c@{}}Gerência de \\ Projeto de Software\end{tabular} & \begin{tabular}[c]{@{}c@{}}Redes de \\ Computadores\end{tabular} &  &  \\ \hline
15h00 &  &  & \begin{tabular}[c]{@{}c@{}}Desenvolvimento de \\ Sistemas para WEB\end{tabular} &  &  \\ \hline
16h00 &  &  & \begin{tabular}[c]{@{}c@{}}Desenvolvimento de \\ Sistemas para WEB\end{tabular} &  &  \\ \hline
18h00 & \begin{tabular}[c]{@{}c@{}}Introdução a \\ Informática\end{tabular} & \begin{tabular}[c]{@{}c@{}}Introdução a \\ Informática\end{tabular} & \begin{tabular}[c]{@{}c@{}}Otimização \\ Linear\end{tabular} &  &  \\ \hline
19h00 & \begin{tabular}[c]{@{}c@{}}Introdução a \\ Informática\end{tabular} & \begin{tabular}[c]{@{}c@{}}Introdução a \\ Informática\end{tabular} & \begin{tabular}[c]{@{}c@{}}Otimização \\ Linear\end{tabular} &  &  \\ \hline
20h00 &  & \begin{tabular}[c]{@{}c@{}}Introdução a \\  Informática\end{tabular} &  & \begin{tabular}[c]{@{}c@{}}Computação \\ Forense\end{tabular} &  \\ \hline
21h00 &  & \begin{tabular}[c]{@{}c@{}}Introdução a \\
Informática\end{tabular} &  & \begin{tabular}[c]{@{}c@{}}Computação \\ Forense\end{tabular} &  \\ \hline
22h00 &  &  &  &  &  \\ \hline
\end{tabular}
\end{adjustbox}
\begin{adjustbox}{width=1\textwidth,center=\textwidth}
\centering
\begin{tabular}{|c|c|c|c|c|c|}
\hline
\textbf{Disciplina} & \textbf{Curso} & \textbf{Período} & \textbf{Professor} & \textbf{Capacidade} & \textbf{Número de Vagas} \\ \hline
Programação I & CC & 1 & Thiago Meireles Paixao & 40 & 37 \\ \hline
Engenharia de Requisitos de Software & CC & 5 & Clayton Vieira Fraga Filho & 40 & 30 \\ \hline
Redes de Computadores & CC & 5 & Rodrigo Freitas Silva & 40 & 20 \\ \hline
Gerência de Projeto de Software & CC & 7 & Clayton Vieira Fraga Filho & 40 & 20 \\ \hline
Desenvolvimento de Sistemas para WEB & CC & 5 & Bruno Vilela Oliveira & 40 & 20 \\ \hline
Introdução a Informática & SI & 1 & Valeria Alves Da Silva & 40 & 30 \\ \hline
Otimização Linear & SI & 5 & Geraldo Regis Mauri & 40 & 20 \\ \hline
Computação Forense & SI & 9 & Jacson Rodrigues Correia Da Silva & 40 & 20 \\ \hline
Lógica e Técnicas de Programação & EM & 3 & Bruno Vilela Oliveira & 40 & 37 \\ \hline
Introdução a Informática & SI & 1 & Larice Nogueira De Andrade & 40 & 30 \\ \hline
\end{tabular}
\end{adjustbox}
\caption{ChiChiu - Laboratório 02}
\end{table}

\begin{table}[!h]
\begin{adjustbox}{width=1\textwidth,center=\textwidth}
\centering
\begin{tabular}{|c|c|c|c|c|c|}
\hline
 & Segunda-Feira & Terça-Feira & Quarta-Feira & Quinta-Feira & Sexta-Feira \\ \hline
07h00 &  &  &  &  &  \\ \hline
08h00 &  &  &  & Programação II &  \\ \hline
09h00 &  &  &  & Programação II &  \\ \hline
10h00 &  & \begin{tabular}[c]{@{}c@{}}Redes de \\ Computadores\end{tabular} & \begin{tabular}[c]{@{}c@{}}Métodos de \\ Otimização\end{tabular} &  &  \\ \hline
11h00 &  & \begin{tabular}[c]{@{}c@{}}Redes de \\ Computadores\end{tabular} & \begin{tabular}[c]{@{}c@{}}Métodos de \\ Otimização\end{tabular} &  &  \\ \hline
13h00 & \begin{tabular}[c]{@{}c@{}}Métodos \\ de Otimização\end{tabular} & Programação I &  &  & \begin{tabular}[c]{@{}c@{}}Inteligência \\ Artificial\end{tabular} \\ \hline
14h00 & \begin{tabular}[c]{@{}c@{}}Métodos \\ de Otimização\end{tabular} & Programação I &  &  & \begin{tabular}[c]{@{}c@{}}Inteligência \\ Artificial\end{tabular} \\ \hline
15h00 & Informática & \begin{tabular}[c]{@{}c@{}}Lógica e Técnicas \\ de Programação\end{tabular} &  &  &  \\ \hline
16h00 & Informática & \begin{tabular}[c]{@{}c@{}}Lógica e Técnicas \\ de Programação\end{tabular} &  &  &  \\ \hline
18h00 & \begin{tabular}[c]{@{}c@{}}Interface \\ Humano-Computador\end{tabular} & \begin{tabular}[c]{@{}c@{}}Introdução a \\ Informática\end{tabular} & Programação I & \begin{tabular}[c]{@{}c@{}}Projeto de \\ Sistemas de Software\end{tabular} & \begin{tabular}[c]{@{}c@{}}Banco \\ de Dados\end{tabular} \\ \hline
19h00 & \begin{tabular}[c]{@{}c@{}}Interface \\ Humano-Computador\end{tabular} & \begin{tabular}[c]{@{}c@{}}Introdução a \\ Informática\end{tabular} & Programação I & \begin{tabular}[c]{@{}c@{}}Projeto de \\ Sistemas de Software\end{tabular} & \begin{tabular}[c]{@{}c@{}}Banco \\ de Dados\end{tabular} \\ \hline
20h00 &  &  &  & \begin{tabular}[c]{@{}c@{}}Introdução a \\ Informática\end{tabular} & \begin{tabular}[c]{@{}c@{}}Comércio \\ Eletrônico\end{tabular} \\ \hline
21h00 &  &  &  & \begin{tabular}[c]{@{}c@{}}Introdução a \\ Informática\end{tabular} & \begin{tabular}[c]{@{}c@{}}Comércio \\ Eletrônico\end{tabular} \\ \hline
22h00 &  &  &  &  & \begin{tabular}[c]{@{}c@{}}Processamento \\ Digital de Imagens\end{tabular} \\ \hline
\end{tabular}
\end{adjustbox}
\begin{adjustbox}{width=1\textwidth,center=\textwidth}
\centering
\begin{tabular}{|c|c|c|c|c|c|}
\hline
\textbf{Disciplina} & \textbf{Curso} & \textbf{Período} & \textbf{Professor} & \textbf{Capacidade} & \textbf{Número de Vagas} \\ \hline
Programação II & CC & 3 & Clayton Vieira Fraga Filho & 40 & 30 \\ \hline
Métodos de Otimização & CC & 5 & Geraldo Regis Mauri & 40 & 20 \\ \hline
Redes de Computadores & CC & 5 & Rodrigo Freitas Silva & 40 & 20 \\ \hline
Inteligência Artificial & CC & 7 & Jacson Rodrigues Correia Da Silva & 40 & 20 \\ \hline
Introdução a Informática & SI & 1 & Valeria Alves Da Silva & 40 & 30 \\ \hline
Banco de Dados & SI & 5 & Antonio Almeida De Barros Junior & 40 & 30 \\ \hline
Projeto de Sistemas de Software & SI & 5 & Clayton Vieira Fraga Filho & 40 & 30 \\ \hline
Interface Humano-Computador & SI & 5 & Simone Dornelas Costa & 40 & 20 \\ \hline
Comércio Eletrônico & SI & 7 & Simone Dornelas Costa & 40 & 20 \\ \hline
Processamento Digital de Imagens & SI & 7 & Thiago Meireles Paixao & 40 & 20 \\ \hline
Informática & AG & 1 & Larice Nogueira De Andrade & 40 & 30 \\ \hline
Lógica e Técnicas de Programação & EM & 3 & Bruno Vilela Oliveira & 40 & 37 \\ \hline
Programação I & SI & 1 & Thiago Meireles Paixao & 40 & 30 \\ \hline
Introdução a Informática & MA & 1 & Bruno Vilela Oliveira & 40 & 35 \\ \hline
\end{tabular}
\end{adjustbox}
\caption{ChiChiu - Laboratório 01}
\end{table}

\begin{table}[!h]
\begin{adjustbox}{width=1\textwidth,center=\textwidth}
\centering
\begin{tabular}{|c|c|c|c|c|c|}
\hline
 & Segunda-Feira & Terça-Feira & Quarta-Feira & Quinta-Feira & Sexta-Feira \\ \hline
07h00 &  &  &  &  &  \\ \hline
08h00 &  &  &  &  &  \\ \hline
09h00 &  &  &  &  &  \\ \hline
10h00 &  &  &  &  &  \\ \hline
11h00 &  &  &  &  &  \\ \hline
13h00 &  &  &  &  &  \\ \hline
14h00 &  &  &  &  &  \\ \hline
15h00 &  &  &  &  & \begin{tabular}[c]{@{}c@{}}Gerenciamento de \\ Banco de Dados\end{tabular} \\ \hline
16h00 &  &  &  &  & \begin{tabular}[c]{@{}c@{}}Gerenciamento de \\ Banco de Dados\end{tabular} \\ \hline
18h00 &  &  &  &  &  \\ \hline
19h00 &  &  &  &  &  \\ \hline
20h00 &  &  &  &  &  \\ \hline
21h00 &  &  &  &  &  \\ \hline
22h00 &  &  &  &  &  \\ \hline
\end{tabular}
\end{adjustbox}
\begin{adjustbox}{width=1\textwidth,center=\textwidth}
\centering
\begin{tabular}{|c|c|c|c|c|c|}
\hline
\textbf{Disciplina} & \textbf{Curso} & \textbf{Período} & \textbf{Professor} & \textbf{Capacidade} & \textbf{Número de Vagas} \\ \hline
Gerenciamento de Banco de Dados & CC & 7 & Antonio Almeida De Barros Junior & 20 & 20 \\ \hline
\end{tabular}
\end{adjustbox}
\caption{Reuni - Laboratório 07}
\end{table}

\begin{table}[!h]
\begin{adjustbox}{width=1\textwidth,center=\textwidth}
\centering
\begin{tabular}{|c|c|c|c|c|c|}
\hline
 & Segunda-Feira & Terça-Feira & Quarta-Feira & Quinta-Feira & Sexta-Feira \\ \hline
07h00 &  &  &  &  &  \\ \hline
08h00 &  &  &  &  &  \\ \hline
09h00 &  &  &  &  &  \\ \hline
10h00 &  &  &  &  &  \\ \hline
11h00 &  &  &  &  &  \\ \hline
13h00 &  &  &  &  &  \\ \hline
14h00 &  &  &  &  &  \\ \hline
15h00 &  &  &  &  &  \\ \hline
16h00 &  &  &  &  &  \\ \hline
18h00 & \begin{tabular}[c]{@{}c@{}}Computação \\ Forense\end{tabular} & \begin{tabular}[c]{@{}c@{}}Sistemas \\ Operacionais\end{tabular} & \begin{tabular}[c]{@{}c@{}}Segurança e \\ Auditoria de Sistemas\end{tabular} &  &  \\ \hline
19h00 & \begin{tabular}[c]{@{}c@{}}Computação \\ Forense\end{tabular} & \begin{tabular}[c]{@{}c@{}}Sistemas \\ Operacionais\end{tabular} & \begin{tabular}[c]{@{}c@{}}Segurança e \\ Auditoria de Sistemas\end{tabular} &  &  \\ \hline
20h00 & \begin{tabular}[c]{@{}c@{}}Processamento \\ Digital de Imagens\end{tabular} &  &  &  &  \\ \hline
21h00 & \begin{tabular}[c]{@{}c@{}}Processamento \\ Digital de Imagens\end{tabular} &  &  &  &  \\ \hline
22h00 & \begin{tabular}[c]{@{}c@{}}Processamento \\ Digital de Imagens\end{tabular} &  &  &  &  \\ \hline
\end{tabular}
\end{adjustbox}
\begin{adjustbox}{width=1\textwidth,center=\textwidth}
\centering
\begin{tabular}{|c|c|c|c|c|c|}
\hline
\textbf{Disciplina} & \textbf{Curso} & \textbf{Período} & \textbf{Professor} & \textbf{Capacidade} & \textbf{Número de Vagas} \\ \hline
Computação Forense & SI & 9 & Jacson Rodrigues Correia Da Silva & 20 & 20 \\ \hline
Sistemas Operacionais & SI & 5 & Jacson Rodrigues Correia Da Silva & 20 & 20 \\ \hline
Segurança e Auditoria de Sistemas & SI & 7 & Simone Dornelas Costa & 20 & 20 \\ \hline
Processamento Digital de Imagens & SI & 7 & Thiago Meireles Paixão & 20 & 20 \\ \hline
\end{tabular}
\end{adjustbox}
\caption{Reuni - Laboratório 05}
\end{table}

% -------------------------------------------------------


\chapter{Tabela de horários obtidas pelo GRASP para Turmas do período 2016/1}

\begin{table}[!h]
\begin{adjustbox}{width=1\textwidth,center=\textwidth}
\centering
\begin{tabular}{|c|c|c|c|c|c|}
\hline
 & Segunda-Feira & Terça-Feira & Quarta-Feira & Quinta-Feira & Sexta-Feira \\ \hline
07h00 &  &  &  &  &  \\ \hline
08h00 &  & Estrutura De Dados I &  &  &  \\ \hline
09h00 &  & Estrutura De Dados I &  &  &  \\ \hline
10h00 & Estrutura De Dados I & Informática e Sociedade &  &  &  \\ \hline
11h00 & Estrutura De Dados I & Informática e Sociedade &  &  &  \\ \hline
13h00 & Português Instrumental & Cálculo B & Lógica Computacional & Cálculo B & Lógica Computacional \\ \hline
14h00 & Português Instrumental & Cálculo B & Lógica Computacional & Cálculo B & Lógica Computacional \\ \hline
15h00 &  & Álgebra Linear & Circuitos Digitais & Álgebra Linear & Circuitos Digitais \\ \hline
16h00 &  & Álgebra Linear & Circuitos Digitais & Álgebra Linear & Circuitos Digitais \\ \hline
18h00 &  &  &  &  &  \\ \hline
19h00 &  &  &  &  &  \\ \hline
20h00 &  &  &  &  &  \\ \hline
21h00 &  &  &  &  &  \\ \hline
22h00 &  &  &  &  &  \\ \hline
\end{tabular}
\end{adjustbox}
\begin{adjustbox}{width=1\textwidth,center=\textwidth}
\centering
\begin{tabular}{|c|c|c|c|c|c|}
\hline
\textbf{Código da disciplina} & \textbf{Disciplina} & \textbf{Sala} & \textbf{Professor} & \textbf{Vagas} \\ \hline
COM06992 & Estrutura De Dados I & CC-3 & Marcelo & 40 \\ \hline
ENG06854 & Português Instrumental & PC-4 & Alexandre Rosa & 30 \\ \hline
COM06996 & Informática e Sociedade & PN-1 & Larice Nogueira De Andrade & 30 \\ \hline
MPA06979 & Cálculo B & PC-4 & Ana Clara & 30 \\ \hline
MPA06855 & Álgebra Linear & PC-4 & Gabriel & 21 \\ \hline
COM06853 & Lógica Computacional & PC-9 & Juliana Pinheiro Campos & 30 \\ \hline
COM06999 & Circuitos Digitais & PC-9 & Valeria Alves Da Silva & 30 \\ \hline
\end{tabular}
\end{adjustbox}
\caption{Ciência da Computação - 2º Período}
\end{table}


\begin{table}[!h]
\begin{adjustbox}{width=1\textwidth,center=\textwidth}
\centering
\begin{tabular}{|c|c|c|c|c|c|}
\hline
 & Segunda-Feira & Terça-Feira & Quarta-Feira & Quinta-Feira & Sexta-Feira \\ \hline
07h00 &  &  &  &  &  \\ \hline
08h00 &  &  &  & \begin{tabular}[c]{@{}c@{}}Sistemas \\ Operacionais\end{tabular} & \begin{tabular}[c]{@{}c@{}}Sistemas \\ Operacionais\end{tabular} \\ \hline
09h00 &  &  &  & \begin{tabular}[c]{@{}c@{}}Sistemas \\ Operacionais\end{tabular} & \begin{tabular}[c]{@{}c@{}}Sistemas \\ Operacionais\end{tabular} \\ \hline
10h00 &  &  &  & \begin{tabular}[c]{@{}c@{}}Algoritmos \\ Numéricos\end{tabular} & \begin{tabular}[c]{@{}c@{}}Algoritmos \\ Numéricos\end{tabular} \\ \hline
11h00 &  &  &  & \begin{tabular}[c]{@{}c@{}}Algoritmos \\ Numéricos\end{tabular} & \begin{tabular}[c]{@{}c@{}}Algoritmos \\ Numéricos\end{tabular} \\ \hline
13h00 & \begin{tabular}[c]{@{}c@{}}Engenharia \\ de Software\end{tabular} & \begin{tabular}[c]{@{}c@{}}Teoria \\ dos Grafos\end{tabular} & \begin{tabular}[c]{@{}c@{}}Teoria \\ dos Grafos\end{tabular} & \begin{tabular}[c]{@{}c@{}}Otimização \\ Linear\end{tabular} & \begin{tabular}[c]{@{}c@{}}Banco \\ de Dados\end{tabular} \\ \hline
14h00 & \begin{tabular}[c]{@{}c@{}}Engenharia \\ de Software\end{tabular} & \begin{tabular}[c]{@{}c@{}}Teoria \\ dos Grafos\end{tabular} & \begin{tabular}[c]{@{}c@{}}Teoria \\ dos Grafos\end{tabular} & \begin{tabular}[c]{@{}c@{}}Otimização \\ Linear\end{tabular} & \begin{tabular}[c]{@{}c@{}}Banco \\ de Dados\end{tabular} \\ \hline
15h00 &  &  &  &  &  \\ \hline
16h00 &  &  &  &  &  \\ \hline
18h00 &  &  &  &  &  \\ \hline
19h00 &  &  &  &  &  \\ \hline
20h00 &  &  &  &  &  \\ \hline
21h00 &  &  &  &  &  \\ \hline
22h00 &  &  &  &  &  \\ \hline

\end{tabular}
\end{adjustbox}
\begin{adjustbox}{width=1\textwidth,center=\textwidth}
\centering
\begin{tabular}{|c|c|c|c|c|c|}
\hline
\textbf{Código da disciplina} & \textbf{Disciplina} & \textbf{Sala} & \textbf{Professor} & \textbf{Vagas} \\ \hline
COM10132 & Sistemas Operacionais & PN-1/PC-9 & Fabrício & 20 \\ \hline
COM10128 & Algoritmos Numéricos & PN-1/PC-9 & Prof. Algoritmos Numéricos & 30 \\ \hline
COM10015 & Engenharia de Software & CC-2 & André & 30 \\ \hline
COM10133 & Teoria dos Grafos & PN-1/PC-4 & Edmar Hell Kampke & 30 \\ \hline
COM10131 & Otimização Linear & CC-1 & Geraldo Régis Mauri & 30 \\ \hline
COM10129 & Banco de Dados & CC-2 & Leandro & 30 \\ \hline

\end{tabular}
\end{adjustbox}
\caption{Ciência da Computação - 4º Período}
\end{table}


\begin{table}[!h]
\begin{adjustbox}{width=1\textwidth,center=\textwidth}
\centering
\begin{tabular}{|c|c|c|c|c|c|}
\hline
 & Segunda-Feira & Terça-Feira & Quarta-Feira & Quinta-Feira & Sexta-Feira \\ \hline
07h00 &  &  &  &  &  \\ \hline
08h00 &  &  &  & \begin{tabular}[c]{@{}c@{}}Projeto de \\ Sistemas de Software\end{tabular} &  \\ \hline
09h00 &  &  &  & \begin{tabular}[c]{@{}c@{}}Projeto de \\ Sistemas de Software\end{tabular} &  \\ \hline
10h00 &  &  &  & \begin{tabular}[c]{@{}c@{}}Computação \\ Gráfica\end{tabular} & \begin{tabular}[c]{@{}c@{}}Sistemas \\ Distribuídos\end{tabular} \\ \hline
11h00 &  &  &  & \begin{tabular}[c]{@{}c@{}}Computação \\ Gráfica\end{tabular} & \begin{tabular}[c]{@{}c@{}}Sistemas \\ Distribuídos\end{tabular} \\ \hline
13h00 & \begin{tabular}[c]{@{}c@{}}Análise e Projeto \\ de Algoritmos\end{tabular} & \begin{tabular}[c]{@{}c@{}}Projeto de \\ Sistemas de Software\end{tabular} & \begin{tabular}[c]{@{}c@{}}Análise e Projeto \\ de Algoritmos\end{tabular} & \begin{tabular}[c]{@{}c@{}}Sistemas \\ Distribuídos\end{tabular} & \begin{tabular}[c]{@{}c@{}}Computação \\ Gráfica\end{tabular} \\ \hline
14h00 & \begin{tabular}[c]{@{}c@{}}Análise e Projeto \\ de Algoritmos\end{tabular} & \begin{tabular}[c]{@{}c@{}}Projeto de \\ Sistemas de Software\end{tabular} & \begin{tabular}[c]{@{}c@{}}Análise e Projeto \\ de Algoritmos\end{tabular} & \begin{tabular}[c]{@{}c@{}}Sistemas \\ Distribuídos\end{tabular} & \begin{tabular}[c]{@{}c@{}}Computação \\ Gráfica\end{tabular} \\ \hline
15h00 & \begin{tabular}[c]{@{}c@{}}Direito e\\  Legislação\end{tabular} & \begin{tabular}[c]{@{}c@{}}Interface \\ Humano-Computador\end{tabular} & \begin{tabular}[c]{@{}c@{}}Tópicos Especiais \\ em Programação\end{tabular} & \begin{tabular}[c]{@{}c@{}}Interface \\ Humano-Computador\end{tabular} & \begin{tabular}[c]{@{}c@{}}Tópicos Especiais \\ em Programação\end{tabular} \\ \hline
16h00 & \begin{tabular}[c]{@{}c@{}}Direito e\\  Legislação\end{tabular} & \begin{tabular}[c]{@{}c@{}}Interface \\ Humano-Computador\end{tabular} & \begin{tabular}[c]{@{}c@{}}Tópicos Especiais \\ em Programação\end{tabular} & \begin{tabular}[c]{@{}c@{}}Interface \\ Humano-Computador\end{tabular} & \begin{tabular}[c]{@{}c@{}}Tópicos Especiais \\ em Programação\end{tabular} \\ \hline
18h00 &  &  &  &  &  \\ \hline
19h00 &  &  &  &  &  \\ \hline
20h00 &  &  &  &  &  \\ \hline
21h00 &  &  &  &  &  \\ \hline
22h00 &  &  &  &  &  \\ \hline

\end{tabular}
\end{adjustbox}
\begin{adjustbox}{width=1\textwidth,center=\textwidth}
\centering
\begin{tabular}{|c|c|c|c|c|c|}
\hline
\textbf{Código da disciplina} & \textbf{Disciplina} & \textbf{Sala} & \textbf{Professor} & \textbf{Vagas} \\ \hline
COM10507 & Interface Humano-Computador & RN-6 & Simone Dornelas Costa & 20 \\ \hline
COM10508 & Projeto de Sistemas de Software & CC-2/CC-3 & André & 30 \\ \hline
COM10602 & Análise e Projeto de Algoritmos & PN-1/PC-9 & Rodrigo Freitas Silva & 30 \\ \hline
COM10603 & Direito Legislação & PN-1 & Larice Nogueira De Andrade & 30 \\ \hline
COM10604 & Computação Gráfica & CC-1/CC-3 & Fabrício & 30 \\ \hline
COM10616 & Sistemas Distribuídos & RN-7 & Fabrício & 20 \\ \hline
COM11013 & Tópicos Especiais em Programação & CC-3 & Jacson Rodrigues Correia Da Silva & 30 \\ \hline

\end{tabular}
\end{adjustbox}
\caption{Ciência da Computação - 6º Período}
\end{table}


\begin{table}[!h]
\begin{adjustbox}{width=1\textwidth,center=\textwidth}
\centering
\begin{tabular}{|c|c|c|c|c|c|}
\hline
 & Segunda-Feira & Terça-Feira & Quarta-Feira & Quinta-Feira & Sexta-Feira \\ \hline
07h00 &  &  &  &  &  \\ \hline
08h00 &  &  &  &  &  \\ \hline
09h00 &  &  &  &  &  \\ \hline
10h00 &  &  &  &  &  \\ \hline
11h00 &  &  &  &  &  \\ \hline
13h00 &  & \begin{tabular}[c]{@{}c@{}}Tópicos Especiais \\ em Otimização\end{tabular} & \begin{tabular}[c]{@{}c@{}}Tópicos Especiais \\ em Inteligência Artificial\end{tabular} &  &  \\ \hline
14h00 &  & \begin{tabular}[c]{@{}c@{}}Tópicos Especiais \\ em Otimização\end{tabular} & \begin{tabular}[c]{@{}c@{}}Tópicos Especiais \\ em Inteligência Artificial\end{tabular} &  &  \\ \hline
15h00 & Empreendedorismo &  & \begin{tabular}[c]{@{}c@{}}Tópicos Especiais \\ em Otimização\end{tabular} & \begin{tabular}[c]{@{}c@{}}Tópicos Especiais \\ em Inteligência Artificial\end{tabular} &  \\ \hline
16h00 & Empreendedorismo &  & \begin{tabular}[c]{@{}c@{}}Tópicos Especiais \\ em Otimização\end{tabular} & \begin{tabular}[c]{@{}c@{}}Tópicos Especiais \\ em Inteligência Artificial\end{tabular} &  \\ \hline
18h00 &  &  &  &  &  \\ \hline
19h00 &  &  &  &  &  \\ \hline
20h00 &  &  &  &  &  \\ \hline
21h00 &  &  &  &  &  \\ \hline
22h00 &  &  &  &  &  \\ \hline

\end{tabular}
\end{adjustbox}
\begin{adjustbox}{width=1\textwidth,center=\textwidth}
\centering
\begin{tabular}{|c|c|c|c|c|c|}
\hline
\textbf{Código da disciplina} & \textbf{Disciplina} & \textbf{Sala} & \textbf{Professor} & \textbf{Vagas} \\ \hline
CFM11061 & Empreendedorismo & PC-9 & Wendel & 30 \\ \hline
COM11013 & Tópicos Especiais em Otimização & CC-1/CC-2 & Dayan & 30 \\ \hline
COM11608 & Tópicos Especiais em Inteligência Artificial & CC-1/CC-3 & Jacson Rodrigues Correia Da Silva & 30 \\ \hline

\end{tabular}
\end{adjustbox}
\caption{Ciência da Computação - 8º Período}
\end{table}


\begin{table}[!h]
\begin{adjustbox}{width=1\textwidth,center=\textwidth}
\centering
\begin{tabular}{|c|c|c|c|c|c|}
\hline
 & Segunda-Feira & Terça-Feira & Quarta-Feira & Quinta-Feira & Sexta-Feira \\ \hline
07h00 &  &  &  &  &  \\ \hline
08h00 &  &  &  &  &  \\ \hline
09h00 &  &  &  &  &  \\ \hline
10h00 &  &  &  &  &  \\ \hline
11h00 &  &  &  &  &  \\ \hline
13h00 &  &  &  &  &  \\ \hline
14h00 &  &  &  &  &  \\ \hline
15h00 &  &  &  &  &  \\ \hline
16h00 &  &  &  &  &  \\ \hline
18h00 & Cálculo A & \begin{tabular}[c]{@{}c@{}}Matemática \\ Discreta\end{tabular} & Cálculo A & \begin{tabular}[c]{@{}c@{}}Matemática \\ Discreta\end{tabular} & Cálculo A \\ \hline
19h00 & Cálculo A & \begin{tabular}[c]{@{}c@{}}Matemática \\ Discreta\end{tabular} & Cálculo A & \begin{tabular}[c]{@{}c@{}}Matemática \\ Discreta\end{tabular} & Cálculo A \\ \hline
20h00 & \begin{tabular}[c]{@{}c@{}}Inglês \\ Instrumental\end{tabular} & \begin{tabular}[c]{@{}c@{}}Teoria Geral \\ Sistemas\end{tabular} & \begin{tabular}[c]{@{}c@{}}Fundamentos de \\ Programação Web\end{tabular} & \begin{tabular}[c]{@{}c@{}}Fundamentos de \\ Programação Web\end{tabular} & \begin{tabular}[c]{@{}c@{}}Teoria Geral \\ Sistemas\end{tabular} \\ \hline
21h00 & \begin{tabular}[c]{@{}c@{}}Inglês \\ Instrumental\end{tabular} & \begin{tabular}[c]{@{}c@{}}Teoria Geral \\ Sistemas\end{tabular} & \begin{tabular}[c]{@{}c@{}}Fundamentos de \\ Programação Web\end{tabular} & \begin{tabular}[c]{@{}c@{}}Fundamentos de \\ Programação Web\end{tabular} & \begin{tabular}[c]{@{}c@{}}Teoria Geral \\ Sistemas\end{tabular} \\ \hline
22h00 &  &  &  &  &  \\ \hline

\end{tabular}
\end{adjustbox}
\begin{adjustbox}{width=1\textwidth,center=\textwidth}
\centering
\begin{tabular}{|c|c|c|c|c|c|}
\hline
\textbf{Código da disciplina} & \textbf{Disciplina} & \textbf{Sala} & \textbf{Professor} & \textbf{Vagas} \\ \hline
COM06851 & Matemática Discreta & MA-0/PN-1 & Edmar Hell Kampke & 40 \\ \hline
COM06984 & Fundamentos de Programação Web & CC-1/CC-3 & Bruno Vilela Oliveira & 40 \\ \hline
COM06985 & Teoria Geral Sistemas & MA-0 & Simone Dornelas Costa & 40 \\ \hline
ENG06849 & Inglês Instrumental & MA-0 & Prof Inglês Instrumental & 40 \\ \hline
MPA06839 & Cálculo A & MA-0 & Prof Calculo A & 40 \\ \hline

\end{tabular}
\end{adjustbox}
\caption{Sistemas de Informação - 2º Período}
\end{table}


\begin{table}[!h]
\begin{adjustbox}{width=1\textwidth,center=\textwidth}
\centering
\begin{tabular}{|c|c|c|c|c|c|}
\hline
 & Segunda-Feira & Terça-Feira & Quarta-Feira & Quinta-Feira & Sexta-Feira \\ \hline
07h00 &  &  &  &  &  \\ \hline
08h00 &  &  &  &  &  \\ \hline
09h00 &  &  &  &  &  \\ \hline
10h00 &  &  &  &  &  \\ \hline
11h00 &  &  &  &  &  \\ \hline
13h00 &  &  &  &  &  \\ \hline
14h00 &  &  &  &  &  \\ \hline
15h00 &  &  &  &  &  \\ \hline
16h00 &  &  &  &  &  \\ \hline
18h00 & \begin{tabular}[c]{@{}c@{}}Engenharia de \\ Requisitos de Software\end{tabular} & Estatística & \begin{tabular}[c]{@{}c@{}}Estrutura \\ de Dados II\end{tabular} & Estatística & Programação II \\ \hline
19h00 & \begin{tabular}[c]{@{}c@{}}Engenharia de \\ Requisitos de Software\end{tabular} & Estatística & \begin{tabular}[c]{@{}c@{}}Estrutura \\ de Dados II\end{tabular} & Estatística & Programação II \\ \hline
20h00 & \begin{tabular}[c]{@{}c@{}}Arquitetura \\ de Computadores\end{tabular} & \begin{tabular}[c]{@{}c@{}}Estrutura \\ de Dados II\end{tabular} & \begin{tabular}[c]{@{}c@{}}Arquitetura \\ de Computadores\end{tabular} & \begin{tabular}[c]{@{}c@{}}Engenharia de \\ Requisitos de Software\end{tabular} &  \\ \hline
21h00 & \begin{tabular}[c]{@{}c@{}}Arquitetura \\ de Computadores\end{tabular} & \begin{tabular}[c]{@{}c@{}}Estrutura \\ de Dados II\end{tabular} & \begin{tabular}[c]{@{}c@{}}Arquitetura \\ de Computadores\end{tabular} & \begin{tabular}[c]{@{}c@{}}Engenharia de \\ Requisitos de Software\end{tabular} &  \\ \hline
22h00 &  &  &  &  &  \\ \hline

\end{tabular}
\end{adjustbox}
\begin{adjustbox}{width=1\textwidth,center=\textwidth}
\centering
\begin{tabular}{|c|c|c|c|c|c|}
\hline
\textbf{Código da disciplina} & \textbf{Disciplina} & \textbf{Sala} & \textbf{Professor} & \textbf{Vagas} \\ \hline
COM10076 & Arquitetura de Computadores & PN-1/PN-12 & Valéria Alves Da Silva & 50 \\ \hline
COM10078 & Estrutura de Dados II & CC-1/CC-3 & Marcelo & 30 \\ \hline
COM10082 & Programação II & CC-1 & Bruno Vilela Oliveira & 30 \\ \hline
COM10275 & Engenharia de Requisitos de Software & CC-2/CC-3 & André & 40 \\ \hline
ENG05510 & Estatística & PN-12 & Prof Estatística Básica & 40 \\ \hline

\end{tabular}
\end{adjustbox}
\caption{Sistemas de Informação - 4º Período}
\end{table}


\begin{table}[!h]
\begin{adjustbox}{width=1\textwidth,center=\textwidth}
\centering
\begin{tabular}{|c|c|c|c|c|c|}
\hline
 & Segunda-Feira & Terça-Feira & Quarta-Feira & Quinta-Feira & Sexta-Feira \\ \hline
07h00 &  &  &  &  &  \\ \hline
08h00 &  &  &  &  &  \\ \hline
09h00 &  &  &  &  &  \\ \hline
10h00 &  &  &  &  &  \\ \hline
11h00 &  &  &  &  &  \\ \hline
13h00 &  &  &  &  &  \\ \hline
14h00 &  &  &  &  &  \\ \hline
15h00 &  &  &  &  &  \\ \hline
16h00 &  &  &  &  &  \\ \hline
18h00 & \begin{tabular}[c]{@{}c@{}}Metodologia de \\ Pesquisa em Informática\end{tabular} & \begin{tabular}[c]{@{}c@{}}Métodos de \\ Otimização\end{tabular} & \begin{tabular}[c]{@{}c@{}}Redes de \\ Computadores\end{tabular} & \begin{tabular}[c]{@{}c@{}}Métodos de \\ Otimização\end{tabular} & \begin{tabular}[c]{@{}c@{}}Gerência de \\ Projeto de Software\end{tabular} \\ \hline
19h00 & \begin{tabular}[c]{@{}c@{}}Metodologia de \\ Pesquisa em Informática\end{tabular} & \begin{tabular}[c]{@{}c@{}}Métodos de \\ Otimização\end{tabular} & \begin{tabular}[c]{@{}c@{}}Redes de \\ Computadores\end{tabular} & \begin{tabular}[c]{@{}c@{}}Métodos de \\ Otimização\end{tabular} & \begin{tabular}[c]{@{}c@{}}Gerência de \\ Projeto de Software\end{tabular} \\ \hline
20h00 & \begin{tabular}[c]{@{}c@{}}Redes de \\ Computadores\end{tabular} & \begin{tabular}[c]{@{}c@{}}Direito \\ Legislação\end{tabular} &  & \begin{tabular}[c]{@{}c@{}}Gerenciamento de \\ Banco de Dados\end{tabular} & \begin{tabular}[c]{@{}c@{}}Gerenciamento de \\ Banco de Dados\end{tabular} \\ \hline
21h00 & \begin{tabular}[c]{@{}c@{}}Redes de \\ Computadores\end{tabular} & \begin{tabular}[c]{@{}c@{}}Direito \\ Legislação\end{tabular} &  & \begin{tabular}[c]{@{}c@{}}Gerenciamento de \\ Banco de Dados\end{tabular} & \begin{tabular}[c]{@{}c@{}}Gerenciamento de \\ Banco de Dados\end{tabular} \\ \hline
22h00 &  &  &  &  &  \\ \hline

\end{tabular}
\end{adjustbox}
\begin{adjustbox}{width=1\textwidth,center=\textwidth}
\centering
\begin{tabular}{|c|c|c|c|c|c|}
\hline
\textbf{Código da disciplina} & \textbf{Disciplina} & \textbf{Sala} & \textbf{Professor} & \textbf{Vagas} \\ \hline
COM10081 & Metodologia de Pesquisa em Informática & PN-12 & Larice Nogueira De Andrade & 40 \\ \hline
COM10393 & Métodos de Otimização & CC-1/CC-3 & Geraldo Régis Mauri & 20 \\ \hline
COM10394 & Redes de Computadores & CC-1/RN-7 & Rodrigo Freitas Silva & 20 \\ \hline
COM10603 & Direito Legislação & PN-12 & Larice Nogueira De Andrade & 40 \\ \hline
COM10733 & Gerência de Projeto de Software & RN-6 & André & 20 \\ \hline
COM11014 & Gerenciamento de Banco de Dados & CC-2/RN-6 & Leandro & 20 \\ \hline

\end{tabular}
\end{adjustbox}
\caption{Sistemas de Informação - 6º Período}
\end{table}


\begin{table}[!h]
\begin{adjustbox}{width=1\textwidth,center=\textwidth}
\centering
\begin{tabular}{|c|c|c|c|c|c|}
\hline
 & Segunda-Feira & Terça-Feira & Quarta-Feira & Quinta-Feira & Sexta-Feira \\ \hline
07h00 &  &  &  &  &  \\ \hline
08h00 &  &  &  &  &  \\ \hline
09h00 &  &  &  &  &  \\ \hline
10h00 &  &  &  &  &  \\ \hline
11h00 &  &  &  &  &  \\ \hline
13h00 &  &  &  &  &  \\ \hline
14h00 &  &  &  &  &  \\ \hline
15h00 &  &  &  &  &  \\ \hline
16h00 &  &  &  &  &  \\ \hline
18h00 & \begin{tabular}[c]{@{}c@{}}Segurança \\ em Redes\end{tabular} & \begin{tabular}[c]{@{}c@{}}Gestão Qualidade \\ de Software\end{tabular} & \begin{tabular}[c]{@{}c@{}}Gestão Qualidade \\ de Software\end{tabular} & Empreendedorismo & \begin{tabular}[c]{@{}c@{}}Desenvolvimento \\ de Sistemas para WEB\end{tabular} \\ \hline
19h00 & \begin{tabular}[c]{@{}c@{}}Segurança \\ em Redes\end{tabular} & \begin{tabular}[c]{@{}c@{}}Gestão Qualidade \\ de Software\end{tabular} & \begin{tabular}[c]{@{}c@{}}Gestão Qualidade \\ de Software\end{tabular} & Empreendedorismo & \begin{tabular}[c]{@{}c@{}}Desenvolvimento \\ de Sistemas para WEB\end{tabular} \\ \hline
20h00 & \begin{tabular}[c]{@{}c@{}}Gerenciamento \\ Administração Redes\end{tabular} & \begin{tabular}[c]{@{}c@{}}Gerenciamento \\ Administração Redes\end{tabular} & \begin{tabular}[c]{@{}c@{}}Segurança \\ em Redes\end{tabular} & \begin{tabular}[c]{@{}c@{}}Desenvolvimento \\ de Sistemas para WEB\end{tabular} &  \\ \hline
21h00 & \begin{tabular}[c]{@{}c@{}}Gerenciamento \\ Administração Redes\end{tabular} & \begin{tabular}[c]{@{}c@{}}Gerenciamento \\ Administração Redes\end{tabular} & \begin{tabular}[c]{@{}c@{}}Segurança \\ em Redes\end{tabular} & \begin{tabular}[c]{@{}c@{}}Desenvolvimento \\ de Sistemas para WEB\end{tabular} &  \\ \hline
22h00 &  &  &  &  &  \\ \hline

\end{tabular}
\end{adjustbox}
\begin{adjustbox}{width=1\textwidth,center=\textwidth}
\centering
\begin{tabular}{|c|c|c|c|c|c|}
\hline
\textbf{Código da disciplina} & \textbf{Disciplina} & \textbf{Sala} & \textbf{Professor} & \textbf{Vagas} \\ \hline
CFM11061 & Empreendedorismo & PN-1 & Wendel & 40 \\ \hline
COM10396 & Desenvolvimento de Sistemas para WEB & CC-2/RN-7 & Jacson Rodrigues Correia Da Silva & 20 \\ \hline
COM10608 & Gerenciamento Administração Redes & CC-3/RN-7 & Helder De Amorim Mendes & 20 \\ \hline
COM11211 & Gestão Qualidade de Software & RN-6 & André & 20 \\ \hline
ENG10792 & Segurança em Redes & CC-3/RN-7 & Helder De Amorim Mendes & 20 \\ \hline

\end{tabular}
\end{adjustbox}
\caption{Sistemas de Informação - 8º Período}
\end{table}


\begin{table}[!h]
\begin{adjustbox}{width=1\textwidth,center=\textwidth}
\centering
\begin{tabular}{|c|c|c|c|c|c|}
\hline
 & Segunda-Feira & Terça-Feira & Quarta-Feira & Quinta-Feira & Sexta-Feira \\ \hline
07h00 &  &  &  &  &  \\ \hline
08h00 &  &  &  &  &  \\ \hline
09h00 &  &  &  &  &  \\ \hline
10h00 &  &  &  &  &  \\ \hline
11h00 &  &  &  &  &  \\ \hline
13h00 &  &  &  &  &  \\ \hline
14h00 &  &  &  &  &  \\ \hline
15h00 &  &  &  &  &  \\ \hline
16h00 &  &  &  &  &  \\ \hline
18h00 &  & Programação I & Programação I &  &  \\ \hline
19h00 &  & Programação I & Programação I &  &  \\ \hline
20h00 &  &  &  &  &  \\ \hline
21h00 &  &  &  &  &  \\ \hline
22h00 &  &  &  &  &  \\ \hline

\end{tabular}
\end{adjustbox}
\begin{adjustbox}{width=1\textwidth,center=\textwidth}
\centering
\begin{tabular}{|c|c|c|c|c|c|}
\hline
\textbf{Código da disciplina} & \textbf{Disciplina} & \textbf{Sala} & \textbf{Professor} & \textbf{Vagas} \\ \hline
COM06842 & Programação I & CC-1/CC-3 & Rômulo & 28 \\ \hline

\end{tabular}
\end{adjustbox}
\caption{Matemática - 1º Período}
\end{table}


\begin{table}[!h]
\begin{adjustbox}{width=1\textwidth,center=\textwidth}
\centering
\begin{tabular}{|c|c|c|c|c|c|}
\hline
 & Segunda-Feira & Terça-Feira & Quarta-Feira & Quinta-Feira & Sexta-Feira \\ \hline
07h00 &  &  &  &  & Informática \\ \hline
08h00 &  &  &  &  & Informática \\ \hline
09h00 &  &  &  &  & Informática \\ \hline
10h00 &  &  &  &  &  \\ \hline
11h00 &  &  &  &  &  \\ \hline
13h00 &  &  &  &  &  \\ \hline
14h00 &  &  &  &  &  \\ \hline
15h00 &  &  &  &  &  \\ \hline
16h00 &  &  &  &  &  \\ \hline
18h00 &  &  &  &  &  \\ \hline
19h00 &  &  &  &  &  \\ \hline
20h00 &  &  &  &  &  \\ \hline
21h00 &  &  &  &  &  \\ \hline
22h00 &  &  &  &  &  \\ \hline

\end{tabular}
\end{adjustbox}
\begin{adjustbox}{width=1\textwidth,center=\textwidth}
\centering
\begin{tabular}{|c|c|c|c|c|c|}
\hline
\textbf{Código da disciplina} & \textbf{Disciplina} & \textbf{Sala} & \textbf{Professor} & \textbf{Vagas} \\ \hline
COM05207 & Informática & CC-3 & Rômulo & 40 \\ \hline

\end{tabular}
\end{adjustbox}
\caption{Agronomia - 1º Período}
\end{table}


\begin{table}[!h]
\begin{adjustbox}{width=1\textwidth,center=\textwidth}
\centering
\begin{tabular}{|c|c|c|c|c|c|}
\hline
 & Segunda-Feira & Terça-Feira & Quarta-Feira & Quinta-Feira & Sexta-Feira \\ \hline
07h00 &  &  &  &  &  \\ \hline
08h00 &  &  &  &  &  \\ \hline
09h00 &  &  &  &  &  \\ \hline
10h00 &  &  &  &  &  \\ \hline
11h00 &  &  &  &  &  \\ \hline
13h00 &  &  & \begin{tabular}[c]{@{}c@{}}Lógica e Técnicas \\ de Programação\end{tabular} & \begin{tabular}[c]{@{}c@{}}Lógica e Técnicas \\ de Programação\end{tabular} &  \\ \hline
14h00 &  &  & \begin{tabular}[c]{@{}c@{}}Lógica e Técnicas \\ de Programação\end{tabular} & \begin{tabular}[c]{@{}c@{}}Lógica e Técnicas \\ de Programação\end{tabular} &  \\ \hline
15h00 &  &  &  &  &  \\ \hline
16h00 &  &  &  &  &  \\ \hline
18h00 &  &  &  &  &  \\ \hline
19h00 &  &  &  &  &  \\ \hline
20h00 &  &  &  &  &  \\ \hline
21h00 &  &  &  &  &  \\ \hline
22h00 &  &  &  &  &  \\ \hline

\end{tabular}
\end{adjustbox}
\begin{adjustbox}{width=1\textwidth,center=\textwidth}
\centering
\begin{tabular}{|c|c|c|c|c|c|}
\hline
\textbf{Código da disciplina} & \textbf{Disciplina} & \textbf{Sala} & \textbf{Professor} & \textbf{Vagas} \\ \hline
COM06039 & Lógica e Técnicas de Programação & CC-2/CC-3 & Leandro & 40 \\ \hline
\end{tabular}
\end{adjustbox}
\caption{Engenharia de Alimentos - 3º Período}
\end{table}

\begin{table}[!h]
\begin{adjustbox}{width=1\textwidth,center=\textwidth}
\centering
\begin{tabular}{|c|c|c|c|c|c|}
\hline
 & Segunda-Feira & Terça-Feira & Quarta-Feira & Quinta-Feira & Sexta-Feira \\ \hline
07h00 &  &  &  &  &  \\ \hline
08h00 &  &  &  &  &  \\ \hline
09h00 &  &  &  &  &  \\ \hline
10h00 &  &  &  &  &  \\ \hline
11h00 &  &  &  &  &  \\ \hline
13h00 &  &  &  &  &  \\ \hline
14h00 &  &  & \begin{tabular}[c]{@{}c@{}}Lógica e Técnicas \\ de Programação\end{tabular} & \begin{tabular}[c]{@{}c@{}}Lógica e Técnicas \\ de Programação\end{tabular} &  \\ \hline
15h00 &  &  & \begin{tabular}[c]{@{}c@{}}Lógica e Técnicas \\ de Programação\end{tabular} & \begin{tabular}[c]{@{}c@{}}Lógica e Técnicas \\ de Programação\end{tabular} &  \\ \hline
16h00 &  &  &  &  &  \\ \hline
18h00 &  &  &  &  &  \\ \hline
19h00 &  &  &  &  &  \\ \hline
20h00 &  &  &  &  &  \\ \hline
21h00 &  &  &  &  &  \\ \hline
22h00 &  &  &  &  &  \\ \hline

\end{tabular}
\end{adjustbox}
\begin{adjustbox}{width=1\textwidth,center=\textwidth}
\centering
\begin{tabular}{|c|c|c|c|c|c|}
\hline
\textbf{Código da disciplina} & \textbf{Disciplina} & \textbf{Sala} & \textbf{Professor} & \textbf{Vagas} \\ \hline
COM06039 & Lógica e Técnicas de Programação & CC-1/CC-2 & Leandro & 40 \\ \hline
\end{tabular}
\end{adjustbox}
\caption{Geologia - 7º Período}
\end{table}


\begin{table}[!h]
\begin{adjustbox}{width=1\textwidth,center=\textwidth}
\centering
\begin{tabular}{|c|c|c|c|c|c|}
\hline
 & Segunda-Feira & Terça-Feira & Quarta-Feira & Quinta-Feira & Sexta-Feira \\ \hline
07h00 &  &  & Informática &  &  \\ \hline
08h00 &  &  & Informática &  &  \\ \hline
09h00 &  &  & Informática &  &  \\ \hline
10h00 &  &  &  &  &  \\ \hline
11h00 &  &  &  &  &  \\ \hline
13h00 &  &  &  &  &  \\ \hline
14h00 &  &  &  &  &  \\ \hline
15h00 &  &  &  &  &  \\ \hline
16h00 &  &  &  &  &  \\ \hline
18h00 &  &  &  &  &  \\ \hline
19h00 &  &  &  &  &  \\ \hline
20h00 &  &  &  &  &  \\ \hline
21h00 &  &  &  &  &  \\ \hline
22h00 &  &  &  &  &  \\ \hline

\end{tabular}
\end{adjustbox}
\begin{adjustbox}{width=1\textwidth,center=\textwidth}
\centering
\begin{tabular}{|c|c|c|c|c|c|}
\hline
\textbf{Código da disciplina} & \textbf{Disciplina} & \textbf{Sala} & \textbf{Professor} & \textbf{Vagas} \\ \hline
COM05207 & Informática & CC-1 & Rômulo & 40 \\ \hline

\end{tabular}
\end{adjustbox}
\caption{Engenharia Florestal - 1º Período}
\end{table}



\begin{table}[!h]
\begin{adjustbox}{width=1\textwidth,center=\textwidth}
\centering
\begin{tabular}{|c|c|c|c|c|c|}
\hline
 & Segunda-Feira & Terça-Feira & Quarta-Feira & Quinta-Feira & Sexta-Feira \\ \hline
07h00 &  &  &  &  &  \\ \hline
08h00 &  &  &  &  &  \\ \hline
09h00 &  &  &  &  &  \\ \hline
10h00 &  &  &  &  &  \\ \hline
11h00 &  &  &  &  &  \\ \hline
13h00 &  &  &  &  &  \\ \hline
14h00 &  &  &  &  &  \\ \hline
15h00 &  &  &  &  &  \\ \hline
16h00 &  &  &  &  &  \\ \hline
18h00 &  &  &  & Programação I & Programação I \\ \hline
19h00 &  &  &  & Programação I & Programação I \\ \hline
20h00 &  &  &  &  &  \\ \hline
21h00 &  &  &  &  &  \\ \hline
22h00 &  &  &  &  &  \\ \hline

\end{tabular}
\end{adjustbox}
\begin{adjustbox}{width=1\textwidth,center=\textwidth}
\centering
\begin{tabular}{|c|c|c|c|c|c|}
\hline
\textbf{Código da disciplina} & \textbf{Disciplina} & \textbf{Sala} & \textbf{Professor} & \textbf{Vagas} \\ \hline
COM06842 & Programação I & CC-2/CC-3 & Rômulo & 30 \\ \hline


\end{tabular}
\end{adjustbox}
\caption{Sistemas de Informação - 1º Período}
\end{table}



\end{apendices}



% (*) Incluir como apêndice a documentação técnica produzida durante o PG (especificação de requisitos,
% projeto arquitetural, etc.). Utilizar o exemplo \includepdf caso o documento seja produzido em outro
% editor de texto (Microsoft Word, LibreOffice Writer) e transformado em PDF. Utilizar o exemplo \include
% caso os documentos tenham sido também escritos em LaTeX.
% \includepdf[pages={1-}]{apendices/apendice01-requisitos.pdf}
% \includepdf[pages={1-}]{apendices/apendice02-projeto.pdf}
% \include{ap1-requisitos}
% \include{ap2-projeto}
\end{apendicesenv}


% Índice remissivo.
\phantompart
\printindex

% Fim do documento.
\end{document}
